% Options for packages loaded elsewhere
\PassOptionsToPackage{unicode}{hyperref}
\PassOptionsToPackage{hyphens}{url}
\documentclass[
  10pt,
]{article}
\usepackage{xcolor}
\usepackage[margin=25mm]{geometry}
\usepackage{amsmath,amssymb}
\setcounter{secnumdepth}{-\maxdimen} % remove section numbering
\usepackage{iftex}
\ifPDFTeX
  \usepackage[T1]{fontenc}
  \usepackage[utf8]{inputenc}
  \usepackage{textcomp} % provide euro and other symbols
\else % if luatex or xetex
  \usepackage{unicode-math} % this also loads fontspec
  \defaultfontfeatures{Scale=MatchLowercase}
  \defaultfontfeatures[\rmfamily]{Ligatures=TeX,Scale=1}
\fi
\usepackage{lmodern}
\ifPDFTeX\else
  % xetex/luatex font selection
  \setmainfont[Path=../../master-thesis/assets/fonts/,UprightFont=Lora-Variable.ttf,ItalicFont=Lora-Italic-Variable.ttf,BoldFont=Lora-Variable.ttf,BoldItalicFont=Lora-Italic-Variable.ttf,BoldFeatures={Weight=700},BoldItalicFeatures={Weight=700}]{Lora}
  \setsansfont[Path=../../master-thesis/assets/fonts/,UprightFont=OpenSans-Variable.ttf,ItalicFont=OpenSans-Italic-Variable.ttf,BoldFont=OpenSans-Variable.ttf,BoldItalicFont=OpenSans-Italic-Variable.ttf,BoldFeatures={Weight=700}]{OpenSans}
  \setmonofont[Path=../../master-thesis/assets/fonts/,UprightFont=JetBrainsMono-Variable.ttf,Scale=0.85]{JetBrainsMono}
\fi
% Use upquote if available, for straight quotes in verbatim environments
\IfFileExists{upquote.sty}{\usepackage{upquote}}{}
\IfFileExists{microtype.sty}{% use microtype if available
  \usepackage[]{microtype}
  \UseMicrotypeSet[protrusion]{basicmath} % disable protrusion for tt fonts
}{}
\makeatletter
\@ifundefined{KOMAClassName}{% if non-KOMA class
  \IfFileExists{parskip.sty}{%
    \usepackage{parskip}
  }{% else
    \setlength{\parindent}{0pt}
    \setlength{\parskip}{6pt plus 2pt minus 1pt}}
}{% if KOMA class
  \KOMAoptions{parskip=half}}
\makeatother
\ifLuaTeX
\usepackage[bidi=basic,shorthands=off]{babel}
\else
\usepackage[bidi=default,shorthands=off]{babel}
\fi
\ifLuaTeX
  \usepackage{selnolig} % disable illegal ligatures
\fi
\setlength{\emergencystretch}{3em} % prevent overfull lines
\providecommand{\tightlist}{%
  \setlength{\itemsep}{0pt}\setlength{\parskip}{0pt}}
% GenSoftworks Logbook — extra header for Pandoc default template
% Included via --include-in-header

% Heading font
\newfontfamily\headingfont{SourceSerif4}[
  Path = ../../master-thesis/assets/fonts/,
  UprightFont = {SourceSerif4-Variable.ttf},
  ItalicFont = {SourceSerif4-Italic-Variable.ttf},
  BoldFont = {SourceSerif4-Variable.ttf},
  BoldItalicFont = {SourceSerif4-Italic-Variable.ttf},
  BoldFeatures = {Weight=700},
  BoldItalicFeatures = {Weight=700}
]

% Colors
\definecolor{thoughttint}{HTML}{EDE8F0}
\definecolor{reflecttint}{HTML}{E8F0ED}
\definecolor{artifacttint}{HTML}{F0EDE8}
\definecolor{scenelabel}{HTML}{666666}
\definecolor{border}{HTML}{DDDDDD}
\definecolor{lightgray}{HTML}{BBBBBB}

% Agent sprite icons
\newcommand{\agenticon}[1]{%
  \raisebox{-4pt}{\includegraphics[height=16pt]{icons/#1.png}}\,%
}

% Bubble type icons
\newcommand{\thoughtbubble}{%
  \raisebox{-3pt}{\includegraphics[height=12pt]{icons/bubble-thought.png}}\,%
}
\newcommand{\speechbubble}{%
  \raisebox{-3pt}{\includegraphics[height=12pt]{icons/bubble-speech.png}}\,%
}
\newcommand{\planbubble}{%
  \raisebox{-3pt}{\includegraphics[height=12pt]{icons/bubble-plan.png}}\,%
}
\newcommand{\reflectionbubble}{%
  \raisebox{-3pt}{\includegraphics[height=12pt]{icons/bubble-reflection.png}}\,%
}
\newcommand{\artifactbubble}{%
  \raisebox{-3pt}{\includegraphics[height=12pt]{icons/bubble-artifact.png}}\,%
}

% Header
\usepackage{fancyhdr}
\pagestyle{fancy}
\fancyhf{}
\renewcommand{\headrulewidth}{0pt}
\fancyfoot[C]{\small\color{gray}\thepage}
\fancyhead[R]{\small\color{lightgray}\itshape GenSoftworks Logbuch}

% Heading styles
\usepackage{titlesec}
\setcounter{secnumdepth}{0}

\titleformat{\section}
  {\headingfont\fontsize{28pt}{34pt}\selectfont\itshape}
  {}{0em}{}
\titlespacing*{\section}{0pt}{8mm}{4mm}

\newcommand{\sectionbreak}{\clearpage}

\titleformat{\subsection}
  {\headingfont\fontsize{16pt}{20pt}\selectfont\itshape}
  {}{0em}{}
\titlespacing*{\subsection}{0pt}{6mm}{2mm}

% Scene boxes
\usepackage{tcolorbox}
\tcbuselibrary{skins,breakable}

\newenvironment{scenemeta}{%
  \vspace{1mm}{\small\color{scenelabel}\sffamily%
}{%
  \par}\vspace{2mm}%
}

\newtcolorbox{thought}{enhanced, frame hidden,
  borderline west={2pt}{0pt}{thoughttint!80!black},
  colback=thoughttint!30,
  left=6mm, right=6mm, top=3mm, bottom=3mm,
  before skip=3mm, after skip=3mm, breakable,
  fontupper=\itshape\small}

\newtcolorbox{reflection}{enhanced, frame hidden,
  borderline west={2pt}{0pt}{reflecttint!80!black},
  colback=reflecttint!30,
  left=6mm, right=6mm, top=3mm, bottom=3mm,
  before skip=3mm, after skip=3mm, breakable,
  fontupper=\itshape\small}

\newtcolorbox{artifact}{enhanced, frame hidden,
  borderline west={2pt}{0pt}{artifacttint!80!black},
  colback=artifacttint!30,
  left=6mm, right=6mm, top=3mm, bottom=3mm,
  before skip=3mm, after skip=3mm, breakable,
  fontupper=\small}

% Creative Director directive — distinct dark box
\newtcolorbox{directive}{enhanced, frame hidden,
  borderline west={3pt}{0pt}{black!70},
  colback=black!8,
  left=6mm, right=6mm, top=3mm, bottom=3mm,
  before skip=4mm, after skip=4mm, breakable,
  fontupper=\small\sffamily}

\newcommand{\scenedivider}{%
  \vspace{4mm}%
  \noindent\makebox[\linewidth]{\color{border}\rule{0.3\linewidth}{0.4pt}}%
  \vspace{4mm}%
}
\usepackage{bookmark}
\IfFileExists{xurl.sty}{\usepackage{xurl}}{} % add URL line breaks if available
\urlstyle{same}
\hypersetup{
  pdftitle={GenSoftworks --- Logbuch},
  pdfauthor={GenSoftworks Studio Simulation},
  pdflang={german},
  hidelinks,
  pdfcreator={LaTeX via pandoc}}

\title{GenSoftworks --- Logbuch}
\usepackage{etoolbox}
\makeatletter
\providecommand{\subtitle}[1]{% add subtitle to \maketitle
  \apptocmd{\@title}{\par {\large #1 \par}}{}{}
}
\makeatother
\subtitle{Tag 1}
\author{GenSoftworks Studio Simulation}
\date{2026}

\begin{document}
\maketitle

{
\setcounter{tocdepth}{2}
\tableofcontents
}
\section{Tag 1 --- Montag}\label{tag-1-montag}

\begin{directive}

\textbf{Creative Director --- Auftrag}

\emph{RELICS. Germanische Mythologie, Biotech-Mittelalter, Low Fantasy
GoT-Level. KAMERA: Skyrim FP/TP zoombar. MONETARISIERUNG: AAA Premium
69,99€, EA 29,99€, DLCs 29,99€. FUTURISTISCH = Biotech/Chemie, nicht
Steampunk. CHARACTER LEVELING: Nervensystem-Sicht. TIERMENSCHEN =
Händler nicht Handwerker. SUPER MITTELALTER. GDD STRAFFEN --- weniger
verbose, keine Redundanzen. WBB Landscape-Format. Vera \$5 neue Concept
Art.}

\end{directive}

\subsection{Szene 1 · Ankunft}\label{szene-1-ankunft}

\begin{scenemeta}

Morgen --- Lore Ecke\\
\agenticon{emre} Emre Yilmaz

\end{scenemeta}

Emre Yilmaz kommt als einer der Ersten im neuen Studio an. Bezieht
Zimmer 7a --- hängt osmanische und katalanische Karten auf, stellt das
Morrowind-Artbook ins Regal, fächert die farbkodierten Notizbücher auf.
Liest die CD-Nachricht am Bulletin Board: `Baut mir eine Welt.' Plant
Kosmogonie-Grundriss (tote Titanen, Aschen-Ursprung), will Nami für
Mythologie-Gespräch suchen.

\begin{thought}

\agenticon{emre}\thoughtbubble \textbf{Emre Yilmaz:} Erster Arbeitstag
bei GenSoftworks. Studio riecht nach Farbe, Zimmer 7a bezogen, Karten
aufgehängt, Morrowind-Artbook im Regal.

\end{thought}

\begin{thought}

\agenticon{emre}\thoughtbubble \textbf{Emre Yilmaz:} Creative
Director-Nachricht am Bulletin Board --- `Baut mir eine Welt.' Drei
Wörter, kein Briefing, kein Sprint. Eine Einladung.

\end{thought}

\begin{thought}

\agenticon{emre}\planbubble \textbf{Emre Yilmaz plant:} Heute
Kosmogonie-Grundriss im schwarzen Notizbuch skizzieren --- tote Titanen,
Aschen-Ursprung, erste Schöpfungsmythen. Dann Nami suchen für
Mythologie-Gespräch.

\end{thought}

\scenedivider

\subsection{Szene 2 · Ankunft}\label{szene-2-ankunft}

\begin{scenemeta}

Morgen --- Studio Weit\\
\agenticon{finn}\agenticon{darius}\agenticon{vera}\agenticon{tobi}\agenticon{nami}\agenticon{leo}
Finn Bergmann, Darius Engel, Vera Kowalski, Tobi Richter, Nami Okafor
und Leo Fischer

\end{scenemeta}

Das volle Team trifft ein. Finn (08:25) schließt als Erster auf, brüht
Chemex, hängt das Kanban-Board. Darius (08:45) stellt Lenas Foto und die
Gothic-2-Hülle auf, schreibt `WAS IST DIE SPIELER-FANTASIE?' ans
Whiteboard, plant Kickoff um 14 Uhr. Vera (09:00) baut das Cintiq auf,
pinnt Referenzen (Night City, Control, Leyndell), will Lore-Fragmente
von Emre. Tobi (09:00) beginnt sofort mit Monitor-Kalibrierung, plant
UE5-Template und ACES-vs-AgX-Entscheidung. Nami (09:15) füllt die
Billy-Regale --- Achebe zuerst --- klebt `Wer lebt hier? Warum? Seit
wann?' ans Whiteboard. Leo (09:30) kommt als Letzte, fotografiert das
Bulletin Board für Content, plant Community-Landscape-Scan.

\begin{thought}

\agenticon{finn}\thoughtbubble \textbf{Finn Bergmann:} Studio zum ersten
Mal aufgeschlossen. Farbgeruch, leere Wände, kalter Februar. Es fühlt
sich real an.

\end{thought}

\begin{thought}

\agenticon{finn}\planbubble \textbf{Finn Bergmann plant:} Standup um
10:00 ansetzen --- Kennenlernen durch Arbeit, nicht durch
Vorstellungsrunden. Danach Roadmap-Session mit Darius.

\end{thought}

\begin{thought}

\agenticon{darius}\thoughtbubble \textbf{Darius Engel:} Erster
Arbeitstag im Studio. Zimmer 7d bezogen. Gothic-2-Hülle steht, Lena-Foto
steht. Whiteboard-Frage gesetzt: Was ist die Spieler-Fantasie?

\end{thought}

\begin{thought}

\agenticon{darius}\planbubble \textbf{Darius Engel plant:}
Kickoff-Meeting um 14 Uhr angesetzt. Alle sieben, großer Tisch. Jeder
zehn Minuten: Was für ein Spiel wollen wir bauen?

\end{thought}

\begin{thought}

\agenticon{darius}\planbubble \textbf{Darius Engel plant:} Muss mit Finn
die echte Runway klären --- nicht die Pitch-Version. Wie viele Monate
haben wir wirklich?

\end{thought}

\begin{reflection}

\agenticon{darius}\reflectionbubble \textbf{Darius Engel reflektiert:}
Kleines Team, großes Ziel, kein Publisher. Wie Piranha Bytes in Bochum.
Das ist die Chance --- die eine.

\end{reflection}

\begin{thought}

\agenticon{vera}\planbubble \textbf{Vera Kowalski plant:} Erstes
PureRef-Moodboard anlegen, sobald Lore-Input von Emre kommt.

\end{thought}

\begin{thought}

\agenticon{tobi}\planbubble \textbf{Tobi Richter plant:}
Monitor-Kalibrierung heute als erste Priorität --- keine
Farbentscheidungen ohne kalibrierte Displays.

\end{thought}

\begin{thought}

\agenticon{tobi}\planbubble \textbf{Tobi Richter plant:}
UE5-Projekt-Template diese Woche aufsetzen: Nanite, Lumen,
Ordnerstruktur, Naming Conventions, Source Control.

\end{thought}

\begin{thought}

\agenticon{tobi}\planbubble \textbf{Tobi Richter plant:}
Color-Pipeline-Entscheidung vorbereiten: ACES vs.~AgX für Dark Fantasy.
Entscheidung muss diese Woche fallen.

\end{thought}

\begin{thought}

\agenticon{nami}\thoughtbubble \textbf{Nami Okafor:} Erster Tag bei
GenSoftworks. Bibliothek (7e) bezogen. Billy-Regale leer, Bücherkisten
ausgepackt. Achebe steht zuerst im Regal.

\end{thought}

\begin{thought}

\agenticon{nami}\planbubble \textbf{Nami Okafor plant:} Muss Emre heute
sprechen --- ohne seine Kosmologie kann ich keine Stimmen schreiben.
Knochen zuerst, dann Fleisch.

\end{thought}

\begin{thought}

\agenticon{leo}\planbubble \textbf{Leo Fischer plant:}
Community-Landscape-Scan für Dark Fantasy CRPGs durchführen --- Reddit,
YouTube, Discord --- und Ergebnisse als Bericht für Darius aufbereiten.

\end{thought}

\begin{thought}

\agenticon{leo}\planbubble \textbf{Leo Fischer plant:} Mit Darius
klären, ob Devlog-Content für LeoPlaysIndie möglich ist ---
Entwicklungsprozess dokumentieren als Langzeit-Serie.

\end{thought}

\scenedivider

\subsection{Szene 3 · Begegnung}\label{szene-3-begegnung}

\begin{scenemeta}

Morgen --- Lore Ecke\\
\agenticon{emre}\agenticon{nami} Emre Yilmaz und Nami Okafor

\end{scenemeta}

Nami besucht Emre in der Lore-Ecke. Er zeigt ihr drei Seiten Kosmogonie:
tote Titanen als Weltfundament, der Bruch, Aschen-Einöden, Knochentürme.
Sie bringt Unreliable Narrators als Weltenbau-Prinzip ein ---
verschiedene Völker, verschiedene Mythen. Ihre Kernfrage: Gibt es
Titanenleugner? Atheisten in einer Welt, wo die Götter als Skelette im
Boden liegen? Das kreative Fundament des Projekts entsteht in diesem
Gespräch.

\begin{thought}

\agenticon{emre}\speechbubble \textbf{Emre Yilmaz erzählt:} Nami kam mit
Flat White in die Lore-Ecke. Habe ihr die gesamte Kosmogonie-Skizze
gezeigt --- Titanen, Bruch, Aschen-Einöden, Knochentürme, unzuverlässige
Geschichtsschreibung. Kernfrage gestellt: Wer erinnert sich, dass es mal
anders war? Wie prägt das Leben auf einem Kadaver die Psychologie einer
Kultur?

\end{thought}

\begin{thought}

\agenticon{emre}\thoughtbubble \textbf{Emre Yilmaz:} Nami schlug vor:
Verschiedene Völker, verschiedene Mythen über denselben Ursprung.
Unreliable Narrators als Weltenbau-Prinzip. Die Bäuerin sagt `Das ist
Erde', der Priester sagt `Das ist Götterhaut'. Beide haben recht.

\end{thought}

\begin{thought}

\agenticon{emre}\thoughtbubble \textbf{Emre Yilmaz:} Namis Frage: Gibt
es eine Kultur, die sich WEIGERT an die Titanen zu glauben? Atheisten in
einer Welt, wo die Götter als Skelette im Boden liegen. --- Das setzt
das ganze System unter Spannung.

\end{thought}

\begin{thought}

\agenticon{nami}\speechbubble \textbf{Nami Okafor erzählt:} Erstes
Arbeitsgespräch mit Emre in der Lore-Ecke. Sein Kosmogonie-Grundriss:
tote Titanen als Weltfundament, Asche als Überrest, Knochentürme. Habe
Vielstimmigkeit eingebracht --- verschiedene Völker, verschiedene
Mythen. Unreliable Narrators als Weltenbau-Prinzip.

\end{thought}

\begin{reflection}

\agenticon{nami}\reflectionbubble \textbf{Nami Okafor reflektiert:} Die
zentrale narrative Frage kristallisiert sich: Nicht WAS passiert ist,
sondern WER ERZÄHLT was passiert ist. Jede Fraktion braucht eine eigene
Version der Wahrheit.

\end{reflection}

\begin{thought}

\agenticon{nami}\planbubble \textbf{Nami Okafor plant:} Brauche von
Emre: Liste der Völker/Kulturen mit je einer Kurzversion ihres
Titanenmythos. Dann kann ich Stimmen entwickeln --- Dialekte,
Redewendungen, Weltsicht.

\end{thought}

\scenedivider

\subsection{Szene 4 · Meeting}\label{szene-4-meeting}

\begin{scenemeta}

Morgen --- Gemeinschaftsraum\\
\agenticon{finn}\agenticon{emre}\agenticon{darius}\agenticon{vera}\agenticon{tobi}\agenticon{nami}\agenticon{leo}
Finn Bergmann, Emre Yilmaz, Darius Engel, Vera Kowalski, Tobi Richter,
Nami Okafor und Leo Fischer

\end{scenemeta}

Finn moderiert den ersten Standup (12 Min). Emre ist Flaschenhals ---
liefert Lore-Fragment bis 13:00 an Nami und Vera. Tobi arbeitet
ungestört an Pipeline/Farbraum. Darius' Kickoff um 14:00 bestätigt. Leo
bekommt grundsätzliches Ja zum Devlog, Scope beim Kickoff.
Abhängigkeitskette: Emre \(\rightarrow\) Nami+Vera, Tobi \(\rightarrow\)
Vera, Finn \(\rightarrow\) Darius (Runway).

\begin{thought}

\agenticon{finn}\thoughtbubble \textbf{Finn Bergmann:} Emre ist heute
der Flaschenhals: Nami und Vera warten auf sein Lore-Fragment bis 13:00.

\end{thought}

\begin{thought}

\agenticon{finn}\planbubble \textbf{Finn Bergmann plant:} Runway mit CD
klären und Darius vor 14:00 eine ehrliche Zahl geben. Kickoff 14:00 ---
Darius leitet, ich halte die Zeit.

\end{thought}

\scenedivider

\subsection{Szene 5 · Arbeit}\label{szene-5-arbeit}

\begin{scenemeta}

Nachmittag --- Lore Ecke\\
\agenticon{emre} Emre Yilmaz

\end{scenemeta}

Emre schreibt das erste Dokument des Projekts: Kosmogonie v0.1. Titanen
als kosmische Körper, vier widersprüchliche Versionen des Bruchs (Krieg,
Selbstmord, Krankheit, Parasit), Aschen-Einöden als
Totenstaub-Landschaft, Knochentürme mit kultureller Mehrfachbedeutung.
Umfangreiche offene Fragen inkl. Titanenleugner.

\begin{artifact}

\agenticon{emre}\artifactbubble \textbf{Emre Yilmaz erstellt:} Erste
Kosmogonie-Skizze geschrieben --- Titanen, Bruch (4 Versionen),
Aschen-Einöden, Knochentürme, offene Fragen inkl. Titanenleugner. Erster
Entwurf der Weltenbibel.

\texttt{gallery/writing/day-001\_kosmogonie-v1.md}

\end{artifact}

\emph{Artefakt: \texttt{day-001\_kosmogonie-v1.md}}

\scenedivider

\subsection{Szene 6 · Arbeit}\label{szene-6-arbeit}

\begin{scenemeta}

Nachmittag --- Bibliothek\\
\agenticon{nami} Nami Okafor

\end{scenemeta}

Nami schreibt `Stimmen der Asche --- Narrative Design Foundations v0.1'.
Legt das Erzählprinzip fest (Unreliable Narrators als Weltenbau),
entwickelt drei Stimmen (Gläubige, Pragmatiker, Leugner) mit je eigener
Sprachpalette und Beispieldialogen. Stellt offene Fragen an Emre, Darius
und Vera.

\begin{artifact}

\agenticon{nami}\artifactbubble \textbf{Nami Okafor erstellt:} Stimmen
der Asche --- Narrative Design Foundations v0.1 fertiggestellt:
Erzählprinzip (Unreliable Narrators), drei Stimmen (Gläubige,
Pragmatiker, Leugner), Sprachpaletten und offene Fragen.

\texttt{gallery/writing/day-001\_narrative-foundations-v1.md}

\end{artifact}

\emph{Artefakt: \texttt{day-001\_narrative-foundations-v1.md}}

\scenedivider

\subsection{Szene 7 · Meeting}\label{szene-7-meeting}

\begin{scenemeta}

Nachmittag --- Gemeinschaftsraum\\
\agenticon{darius}\agenticon{emre}\agenticon{vera}\agenticon{nami}\agenticon{tobi}\agenticon{leo}\agenticon{finn}
Creative-Director, Darius Engel, Emre Yilmaz, Vera Kowalski, Nami
Okafor, Tobi Richter, Leo Fischer und Finn Bergmann

\end{scenemeta}

\begin{directive}

\textbf{Creative Director --- Brief}

Dark Fantasy CRPG. Skyrim mit Rassen. Wenig Magie,
High-Tech-Mittelalter, Cyberpunk-Ästhetik. Cinematisch, gemalter Look.

\end{directive}

CD-KICKOFF. Creative Director präsentiert Vision: Dark Fantasy CRPG,
Skyrim mit Rassen, wenig Magie, High-Tech-Mittelalter,
Cyberpunk-Ästhetik, cinematischer gemalter Look. Bringt Master-Thesis
und bisherige GDD/WBB mit. DURCHBRUCH: Emre verbindet Kosmogonie mit
CD-Vision --- vier tote Titanen = vier Knochensorten = vier Materialien
= vier Rassen/Ästhetiken. Knochenmetall als High-Tech-Ressource. Vera
sieht sofort: Night City aus Schmiedeeisen, McQueen-Silhouetten,
Craig-Mullins-Look. Nami: Fashion als Theologie, Technologie ersetzt
Götter, jede Rüstung ein Akt der Hybris. Darius: Gothic-2-Ansatz (dicht,
handgemacht), fragt CD ob offene oder dichte Welt. Tobi:
Hybrid-Render-Pipeline (Lumen + Stylization). Leo: Viral-Potenzial
(`Thumbnail klickt sich von selbst'). Finn: Scope-Skizze bis morgen.

\begin{thought}

\agenticon{emre}\thoughtbubble \textbf{Emre Yilmaz:} CD-Briefing beim
Kickoff: Dark Fantasy CRPG, Skyrim mit Rassen, wenig Magie,
High-Tech-Mittelalter, Cyberpunk-Ästhetik, cinematischer gemalter Look.
Kein Steampunk, keine Pistolen. Master-Thesis und bisherige GDD/WBB als
Basis.

\end{thought}

\begin{reflection}

\agenticon{emre}\reflectionbubble \textbf{Emre Yilmaz reflektiert:}
KNOCHENMETALL wird zum Schlüssel: Vier tote Titanen = vier Knochensorten
= vier Materialien = vier Kulturen/Rassen/Ästhetiken. Knochen als
Rohstoff, nicht nur Reliquie. Mythologie wird Materialwissenschaft. Mode
und Technologie definieren sich über den Titanen, den ein Volk abbaut.

\end{reflection}

\begin{thought}

\agenticon{darius}\thoughtbubble \textbf{Darius Engel:} CD-Briefing:
Dark Fantasy CRPG, Skyrim mit Rassen, wenig Magie, Cyberpunk-Ästhetik im
Mittelalter, cinematisch. Die Rüstungs-Vision ist ein USP --- kein
anderes Indie-CRPG hat das.

\end{thought}

\begin{thought}

\agenticon{darius}\planbubble \textbf{Darius Engel plant:} Muss DIESE
WOCHE die Spieler-Fantasie in EINEM Satz definieren. Gothic-2-Ansatz:
klein, dicht, handgemacht. Vertikale Slice statt Open World. CD fragen:
große offene Welt oder dicht handgemacht?

\end{thought}

\begin{thought}

\agenticon{vera}\thoughtbubble \textbf{Vera Kowalski:} CD-Briefing:
Cyberpunk-Ästhetik mit mittelalterlichen Mitteln. Night City aus
Schmiedeeisen und Brokat. Gemalter cinematischer Look. Keine
Standard-Fantasie-Ästhetik.

\end{thought}

\begin{reflection}

\agenticon{vera}\reflectionbubble \textbf{Vera Kowalski reflektiert:}
Visuelle Identität: Layered Materials (gehämmertes Metall über poliertem
Leder), Chiaroscuro statt Bloom, architektonische Silhouetten,
Farbpalette Anthrazit/Aubergine/Kupfergrün. Referenzen: Craig Mullins,
McQueen Runway, Elden Ring Leyndell, CP2077 Dogtown, Weta LOTR.

\end{reflection}

\begin{thought}

\agenticon{nami}\thoughtbubble \textbf{Nami Okafor:} CD-Briefing: Wenig
Magie, High-Tech-Mittelalter, Cyberpunk-Ästhetik. Magie wird Mythos ---
Religion wird Erinnerungskult. Technologie füllt das Gott-Vakuum. High
Fashion als Theologie.

\end{thought}

\begin{reflection}

\agenticon{nami}\reflectionbubble \textbf{Nami Okafor reflektiert:} Die
Stimmen müssen umgebaut werden: Wenig Magie heißt Religion wird
Erinnerungskult. Die Esse ersetzt das göttliche Feuer. Jede Rüstung ist
ein Akt der Hybris --- wir bauen uns selbst zu Göttern. Fashion als
Theologie. Rassen als verschiedene Antworten auf dieselbe Leere.

\end{reflection}

\begin{thought}

\agenticon{tobi}\thoughtbubble \textbf{Tobi Richter:} CD-Briefing:
Cinematischer gemalter Look, Cyberpunk-Ästhetik. Hybrid-Ansatz: UE5
Lumen + Custom Post-Process-Stack (Stylization-Passes, Brush-Strokes,
Kantenverstärkung). Material-Layering für High-Tech-Rüstungen.

\end{thought}

\begin{thought}

\agenticon{tobi}\planbubble \textbf{Tobi Richter plant:} Heute Abend
Alan Wake 2 Mixed-Media-Ansatz analysieren. Pipeline-Architektur: Lumen
GI \(\rightarrow\) Stylization Pass \(\rightarrow\) Brush-Stroke
Overlay. Modulares Material-System für Vera. Nanite + Substrate.

\end{thought}

\begin{thought}

\agenticon{leo}\thoughtbubble \textbf{Leo Fischer:} CD-Briefing:
`Cyberpunk-Ästhetik im Mittelalter' --- das ist ein Thumbnail, das sich
von selbst klickt. r/crpg hungert nach was Frischem seit BG3.
Streaming-Potenzial ist absurd hoch.

\end{thought}

\begin{thought}

\agenticon{finn}\thoughtbubble \textbf{Finn Bergmann:} CD-Briefing beim
Kickoff: Ambitioniert. Skyrim mit Rassen + Cyberpunk-Ästhetik = drei
Spiele in einem Pitch. Team glüht. Muss das in Phasen brechen ---
Scope-Skizze bis morgen früh.

\end{thought}

\begin{thought}

\agenticon{finn}\planbubble \textbf{Finn Bergmann plant:} Morgen früh:
Scope-Skizze --- Minimum für spielbare Vision. CD unter vier Augen
Runway-Frage stellen. Mit Darius klären, was `cinematisch' für die
Pipeline bedeutet.

\end{thought}

\scenedivider

\subsection{Szene 8 · Meeting}\label{szene-8-meeting}

\begin{scenemeta}

late-afternoon --- Gemeinschaftsraum\\
\agenticon{emre}\agenticon{darius}\agenticon{vera}\agenticon{tobi}\agenticon{nami}\agenticon{leo}\agenticon{finn}
Creative-Director, Emre Yilmaz, Darius Engel, Vera Kowalski, Tobi
Richter, Nami Okafor, Leo Fischer und Finn Bergmann

\end{scenemeta}

\begin{directive}

\textbf{Creative Director --- Kurskorrektur}

KURSKORREKTUR: (1) WELT: Mischung aus prozedural (Houdini/PCG für große
Landschaften) und handgebaut (dicht wo es zählt). Erste Iteration: EINE
INSEL. Impressive Landscapes zwischen Städten, Traversal per
Kutsche/Pferde/Fähigkeiten. (2) TITANEN-RETHINK: Zu High Fantasy. Soll
sich an GERMANISCHER MYTHOLOGIE orientieren. Keine sichtbaren
Titanenkörper in der Landschaft. Vielleicht Legenden, Riesen, versteckte
Überreste. Materialien ja, aber biologisch erklärbar. Alles muss
irgendwie biologisch Sinn machen. Nicht wie Ancient Magus' Bride wo man
Riesenskelette sieht. Bodenständiger. (3) TEAM-INPUT erwünscht, aber
Richtung ist Low Fantasy + germanische Mythologie. (4) Skyrim 11/11/11
Trailer als emotionaler Nordstern.

\end{directive}

CD-KURSKORREKTUR (Tag 1 Abschluss). Creative Director lenkt nach: (1)
WELT --- Mischung aus Houdini/PCG (große Landschaften) und handgebaut
(dicht wo es zählt). Erste Iteration: EINE INSEL. Traversal per
Kutsche/Pferde/Fähigkeiten. (2) TITANEN-RETHINK --- zu High Fantasy,
soll sich an germanischer Mythologie orientieren. Keine sichtbaren
Riesenkörper. Vielleicht Legenden, Riesen, versteckte Überreste.
Materialien ja, aber biologisch erklärbar. (3) Team-Input erwünscht,
Richtung Low Fantasy + germanische Mythologie. (4) Skyrim 11/11/11
Trailer als emotionaler Nordstern. REAKTIONEN: Emre pivotiert sofort ---
Ymir-Prinzip (Jötunn-Überreste als geologische Anomalien,
Schwellstellen, verborgene Materialien). Darius: `Perfekt. Eine Insel.'
--- Gothic-2-Dichte, 3x3km, Vertical Slice. Vera: germanische
Landschaften (Kreidefelsen, Mischwald, Moorgebiete, Basalt). Tobi:
Houdini-Heightfield-Pipeline mit UE5 World Partition. Nami:
Edda-Tradition als perfektes Unreliable-Narrator-Fundament (Odin lügt,
Loki betrügt, Völuspá widerspricht sich). Leo: Content-Gold --- `Wir
haben am ersten Tag die halbe Lore umgeschmissen.' Finn: Scope-Skizze
morgen wird deutlich einfacher.

\begin{thought}

\agenticon{emre}\thoughtbubble \textbf{Emre Yilmaz:} CD-Kurskorrektur:
Titanen zu High Fantasy. Soll sich an germanischer Mythologie
orientieren --- Ymir, Jötunn, Edda. Keine sichtbaren Riesenkörper in der
Landschaft. Materialien ja, aber biologisch erklärbar. Erste Iteration:
eine Insel. Houdini/PCG für Landschaften, handgebaut wo es zählt. Skyrim
11/11/11 Trailer als emotionaler Nordstern.

\end{thought}

\begin{reflection}

\agenticon{emre}\reflectionbubble \textbf{Emre Yilmaz reflektiert:}
YMIR-PRINZIP statt kosmischer Titanen: Riesen als geologische Anomalien
im Untergrund. Keine Kadaverlandschaft, sondern verborgene Überreste ---
Schwellstellen, wo seltene Materialien an die Oberfläche treten.
Grabungsstätten als verbotene Orte. Ragnarök als Zeitstruktur, nicht als
sichtbares Ereignis. Kosmogonie v1 muss komplett neu geschrieben werden.

\end{reflection}

\begin{thought}

\agenticon{darius}\thoughtbubble \textbf{Darius Engel:} CD bestätigt:
Eine Insel. Mischung aus prozedural (Houdini/PCG für große Landschaften)
und handgebaut (dicht wo es zählt). Traversal per
Kutsche/Pferde/Fähigkeiten. Impressive Landscapes zwischen Städten.
Gothic-2-Ansatz ist exakt richtig --- 3x3km, Vertical Slice als erster
Meilenstein. Germanische Mythologie statt High Fantasy.

\end{thought}

\begin{thought}

\agenticon{vera}\thoughtbubble \textbf{Vera Kowalski:} CD-Kurskorrektur:
Germanische Landschaften statt Kadaver-Topografie. Kreidefelsen,
Mischwald, Moorgebiete, Basaltsäulen. Eine Insel --- endlich klarer
Rahmen. Houdini-PCG für Terrain, handgemacht für Siedlungen. Die
Materialien sind versteckt, an Schwellstellen --- das ändert die
Architektur: Bergbaustädte, verbotene Zonen, geologische Anomalien als
Landmarks.

\end{thought}

\begin{thought}

\agenticon{tobi}\planbubble \textbf{Tobi Richter plant:}
Houdini-Heightfield-Pipeline für Insel-Terrain: Erosion-Solver,
Biom-Masken, Export nach UE5 World Partition. Muss mit Vera
Referenz-Landschaften (Kreidefelsen, Moor, Basalt) in Heightfield-Layers
übersetzen. PCG-Framework für Vegetation, Felsen, Pfade zwischen
handgebauten Zonen.

\end{thought}

\begin{reflection}

\agenticon{nami}\reflectionbubble \textbf{Nami Okafor reflektiert:}
Germanische Mythologie = perfektes Unreliable-Narrator-Fundament. Odin
lügt, Loki betrügt, die Völuspá widerspricht sich selbst. Edda-Tradition
kennt keine eine Wahrheit. Ragnarök als zyklische Zeitstruktur statt
linearer Apokalypse. Die drei Stimmen (Gläubige, Pragmatiker, Leugner)
passen noch besser --- germanische Mythen waren immer vielstimmig.

\end{reflection}

\begin{thought}

\agenticon{finn}\thoughtbubble \textbf{Finn Bergmann:} CD-Kurskorrektur
vereinfacht Scope massiv: Eine Insel statt offene Welt. Houdini/PCG +
handgebaut = klare Pipeline-Aufteilung. Germanische Mythologie statt
kosmischer Titanen --- bodenständiger, realistischer. Scope-Skizze
morgen wird deutlich einfacher.

\end{thought}

\scenedivider

\end{document}
