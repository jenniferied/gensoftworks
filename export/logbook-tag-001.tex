% LaTeX Template for GenSoftworks Logbook
% A4 Portrait, Chronological Scene Narrative
% LuaLaTeX required
% Fonts from master-thesis/assets/fonts/

\documentclass[10pt, a4paper, oneside]{article}

% ==========================================================================
%   PACKAGES & FONTS
% ==========================================================================

\usepackage{fontspec}

% Font path — shared with master-thesis repo (same parent directory)
\def\fontpath{../../master-thesis/assets/fonts/}

% Body: Lora 10pt
\setmainfont{Lora}[
  Path = \fontpath,
  UprightFont = {Lora-Variable.ttf},
  ItalicFont = {Lora-Italic-Variable.ttf},
  BoldFont = {Lora-Variable.ttf},
  BoldItalicFont = {Lora-Italic-Variable.ttf},
  BoldFeatures = {Weight=700},
  BoldItalicFeatures = {Weight=700}
]

\setsansfont{OpenSans}[
  Path = \fontpath,
  UprightFont = {OpenSans-Variable.ttf},
  ItalicFont = {OpenSans-Italic-Variable.ttf},
  BoldFont = {OpenSans-Variable.ttf},
  BoldItalicFont = {OpenSans-Italic-Variable.ttf},
  BoldFeatures = {Weight=700},
  BoldItalicFeatures = {Weight=700}
]

\setmonofont{JetBrainsMono}[
  Path = \fontpath,
  UprightFont = {JetBrainsMono-Variable.ttf},
  ItalicFont = {JetBrainsMono-Italic-Variable.ttf},
  BoldFont = {JetBrainsMono-Variable.ttf},
  BoldItalicFont = {JetBrainsMono-Italic-Variable.ttf},
  BoldFeatures = {Weight=700},
  BoldItalicFeatures = {Weight=700},
  Scale = 0.85
]

% Heading font: Source Serif 4 Italic
\newfontfamily\headingfont{SourceSerif4}[
  Path = \fontpath,
  UprightFont = {SourceSerif4-Variable.ttf},
  ItalicFont = {SourceSerif4-Italic-Variable.ttf},
  BoldFont = {SourceSerif4-Variable.ttf},
  BoldItalicFont = {SourceSerif4-Italic-Variable.ttf},
  BoldFeatures = {Weight=700},
  BoldItalicFeatures = {Weight=700}
]

\let\titlefont\headingfont

\usepackage{polyglossia}
\setdefaultlanguage{german}

% Page layout — generous margins for readability
\usepackage[a4paper, left=25mm, right=25mm, top=25mm, bottom=25mm]{geometry}

% Colors (must load before pagecolor)
\usepackage{xcolor}

% Background color (defined but not applied as page background to avoid font embedding issues)
\definecolor{pagewhite}{HTML}{FCFCFC}
\definecolor{black}{HTML}{1A1A1A}
\definecolor{gray}{HTML}{888888}
\definecolor{lightgray}{HTML}{BBBBBB}
\definecolor{boxbg}{HTML}{F5F5F5}
\definecolor{border}{HTML}{DDDDDD}
\definecolor{thoughttint}{HTML}{EDE8F0}    % Soft lavender — agent inner thoughts
\definecolor{reflecttint}{HTML}{E8F0ED}    % Soft sage — reflections
\definecolor{artifacttint}{HTML}{F0EDE8}   % Soft amber — artifacts produced
\definecolor{scenelabel}{HTML}{666666}

% ==========================================================================
%   SPACING
% ==========================================================================

\usepackage{setspace}
\setstretch{1.3}
\setlength{\parindent}{0pt}
\setlength{\parskip}{1.5mm}

% ==========================================================================
%   HEADINGS
% ==========================================================================

\usepackage{fancyhdr}
\pagestyle{fancy}
\fancyhf{}
\renewcommand{\headrulewidth}{0pt}
\fancyfoot[C]{\small\color{gray}\thepage}
\fancyhead[R]{\small\color{lightgray}\itshape GenSoftworks Logbuch}
\fancypagestyle{plain}{\fancyhf{}\renewcommand{\headrulewidth}{0pt}}

\usepackage{titlesec}
\setcounter{secnumdepth}{0}
\setcounter{tocdepth}{2}

% H1 — Day header (Source Serif 4 Italic 28pt)
\titleformat{\section}
  {\headingfont\fontsize{28pt}{34pt}\selectfont\itshape\color{black}}
  {}{0em}{}
\titlespacing*{\section}{0pt}{8mm}{4mm}

\newcommand{\sectionbreak}{\clearpage}

% H2 — Scene header (Source Serif 4 Italic 16pt)
\titleformat{\subsection}
  {\headingfont\fontsize{16pt}{20pt}\selectfont\itshape\color{black}}
  {}{0em}{}
\titlespacing*{\subsection}{0pt}{6mm}{2mm}

% H3 — Sub-sections within scenes
\titleformat{\subsubsection}
  {\bfseries\fontsize{11pt}{13pt}\selectfont\color{black}}
  {}{0em}{}
\titlespacing*{\subsubsection}{0pt}{3mm}{1mm}

% ==========================================================================
%   LINKS & GRAPHICS
% ==========================================================================

\usepackage[colorlinks=true, linkcolor=black, citecolor=black, urlcolor=gray, pdfusetitle]{hyperref}

\usepackage{graphicx}
\usepackage{float}
\floatplacement{figure}{H}

\providecommand{\pandocbounded}[1]{#1}

% ==========================================================================
%   TABLES
% ==========================================================================

\usepackage{longtable}
\usepackage{booktabs}
\usepackage{array}
\usepackage{tabularx}

\usepackage{caption}
\captionsetup{font={small}, labelfont={bf}, format=plain, skip=2mm}

% ==========================================================================
%   CUSTOM ENVIRONMENTS — Scene elements
% ==========================================================================

\usepackage{tcolorbox}
\tcbuselibrary{skins,breakable}

% Scene metadata line (location, time, participants)
\newenvironment{scenemeta}{%
  \vspace{1mm}%
  {\small\color{scenelabel}\sffamily%
}{%
  \par}%
  \vspace{2mm}%
}

% Agent inner thought — lavender tint, left border
\newtcolorbox{thought}{enhanced, frame hidden,
  borderline west={2pt}{0pt}{thoughttint!80!black},
  colback=thoughttint!30,
  left=6mm, right=6mm, top=3mm, bottom=3mm,
  before skip=3mm, after skip=3mm, breakable,
  fontupper=\itshape\small}

% Agent reflection — sage tint, left border
\newtcolorbox{reflection}{enhanced, frame hidden,
  borderline west={2pt}{0pt}{reflecttint!80!black},
  colback=reflecttint!30,
  left=6mm, right=6mm, top=3mm, bottom=3mm,
  before skip=3mm, after skip=3mm, breakable,
  fontupper=\itshape\small}

% Artifact produced — amber tint, left border
\newtcolorbox{artifact}{enhanced, frame hidden,
  borderline west={2pt}{0pt}{artifacttint!80!black},
  colback=artifacttint!30,
  left=6mm, right=6mm, top=3mm, bottom=3mm,
  before skip=3mm, after skip=3mm, breakable,
  fontupper=\small}

% Quote blocks
\renewenvironment{quote}{%
  \vspace{2mm}
  \begin{tcolorbox}[enhanced, frame hidden,
    borderline west={2pt}{0pt}{thoughttint!80!black},
    colback=thoughttint!30,
    left=8mm, right=8mm, top=4mm, bottom=4mm,
    before skip=3mm, after skip=3mm, breakable]
  \itshape
}{%
  \end{tcolorbox}
}

% Scene divider
\newcommand{\scenedivider}{%
  \vspace{4mm}%
  \noindent\makebox[\linewidth]{\color{border}\rule{0.3\linewidth}{0.4pt}}%
  \vspace{4mm}%
}

% ==========================================================================
%   PANDOC COMPATIBILITY
% ==========================================================================

\newlength{\cslhangindent}\setlength{\cslhangindent}{1.5em}
\newlength{\csllabelwidth}\setlength{\csllabelwidth}{3em}
\newenvironment{CSLReferences}[2]
 {\begin{list}{}{\setlength{\itemindent}{0pt}\setlength{\leftmargin}{0pt}\setlength{\parsep}{0pt}
  \ifodd #1 \setlength{\leftmargin}{\cslhangindent}\setlength{\itemindent}{-1\cslhangindent}\fi
  \setlength{\itemsep}{#2\baselineskip}}}
 {\end{list}}

\providecommand{\tightlist}{\setlength{\itemsep}{0pt}\setlength{\parskip}{0pt}}

\usepackage{enumitem}
\setlist[itemize]{leftmargin=5mm, itemsep=0.5mm, topsep=0pt}
\setlist[enumerate]{leftmargin=5mm, itemsep=0.5mm, topsep=0pt}


\usepackage{fancyvrb}
\DefineVerbatimEnvironment{Highlighting}{Verbatim}{
  commandchars=\\\{\}, fontsize=\small, frame=single, rulecolor=\color{border}
}

% ==========================================================================
%   METADATA
% ==========================================================================

\title{GenSoftworks --- Logbuch}
\author{GenSoftworks Studio Simulation}
\date{2026}

\hypersetup{pdftitle={GenSoftworks --- Logbuch}, pdfauthor={GenSoftworks
Studio Simulation}}

% ==========================================================================
%   COVER PAGE
% ==========================================================================

\makeatletter
\renewcommand{\maketitle}{%
  \thispagestyle{empty}
  \null
  \vfill
  \begin{center}
    {\titlefont\fontsize{36pt}{42pt}\selectfont\itshape\@title\par}
    \vspace{8mm}
        {\titlefont\fontsize{14pt}{18pt}\selectfont Tag 1\par}
    \vspace{8mm}
        {\small\color{gray} \@author\quad·\quad\@date\par}
  \end{center}
  \vfill
  \newpage
}
\makeatother

% ==========================================================================
%   DOCUMENT
% ==========================================================================

\begin{document}

\maketitle

\thispagestyle{empty}
\hypersetup{linkcolor=black}
\tableofcontents
\newpage

\section{Tag 1 --- Montag}\label{tag-1-montag}

\subsection{Szene 1 · Ankunft}\label{szene-1-ankunft}

\begin{scenemeta}

Morgen --- Lore Ecke\\
Emre Yilmaz

\end{scenemeta}

Emre Yilmaz kommt als einer der Ersten im neuen Studio an. Bezieht
Zimmer 7a --- hängt osmanische und katalanische Karten auf, stellt das
Morrowind-Artbook ins Regal, fächert die farbkodierten Notizbücher auf.
Liest die CD-Nachricht am Bulletin Board: `Baut mir eine Welt.' Plant
Kosmogonie-Grundriss (tote Titanen, Aschen-Ursprung), will Nami für
Mythologie-Gespräch suchen.

\begin{thought}

\textbf{Emre Yilmaz:} Erster Arbeitstag bei GenSoftworks. Studio riecht
nach Farbe, Zimmer 7a bezogen, Karten aufgehängt, Morrowind-Artbook im
Regal.

\end{thought}

\begin{thought}

\textbf{Emre Yilmaz:} Creative Director-Nachricht am Bulletin Board ---
`Baut mir eine Welt.' Drei Wörter, kein Briefing, kein Sprint. Eine
Einladung.

\end{thought}

\begin{thought}

\textbf{Emre Yilmaz plant:} Heute Kosmogonie-Grundriss im schwarzen
Notizbuch skizzieren --- tote Titanen, Aschen-Ursprung, erste
Schöpfungsmythen. Dann Nami suchen für Mythologie-Gespräch.

\end{thought}

\scenedivider

\subsection{Szene 2 · Ankunft}\label{szene-2-ankunft}

\begin{scenemeta}

Morgen --- Studio Weit\\
Finn Bergmann, Darius Engel, Vera Kowalski, Tobi Richter, Nami Okafor
und Leo Fischer

\end{scenemeta}

Das volle Team trifft ein. Finn (08:25) schließt als Erster auf, brüht
Chemex, hängt das Kanban-Board. Darius (08:45) stellt Lenas Foto und die
Gothic-2-Hülle auf, schreibt `WAS IST DIE SPIELER-FANTASIE?' ans
Whiteboard, plant Kickoff um 14 Uhr. Vera (09:00) baut das Cintiq auf,
pinnt Referenzen (Night City, Control, Leyndell), will Lore-Fragmente
von Emre. Tobi (09:00) beginnt sofort mit Monitor-Kalibrierung, plant
UE5-Template und ACES-vs-AgX-Entscheidung. Nami (09:15) füllt die
Billy-Regale --- Achebe zuerst --- klebt `Wer lebt hier? Warum? Seit
wann?' ans Whiteboard. Leo (09:30) kommt als Letzte, fotografiert das
Bulletin Board für Content, plant Community-Landscape-Scan.

\begin{thought}

\textbf{Finn Bergmann:} Studio zum ersten Mal aufgeschlossen.
Farbgeruch, leere Wände, kalter Februar. Es fühlt sich real an.

\end{thought}

\begin{thought}

\textbf{Finn Bergmann plant:} Standup um 10:00 ansetzen --- Kennenlernen
durch Arbeit, nicht durch Vorstellungsrunden. Danach Roadmap-Session mit
Darius.

\end{thought}

\begin{thought}

\textbf{Darius Engel:} Erster Arbeitstag im Studio. Zimmer 7d bezogen.
Gothic-2-Hülle steht, Lena-Foto steht. Whiteboard-Frage gesetzt: Was ist
die Spieler-Fantasie?

\end{thought}

\begin{thought}

\textbf{Darius Engel plant:} Kickoff-Meeting um 14 Uhr angesetzt. Alle
sieben, großer Tisch. Jeder zehn Minuten: Was für ein Spiel wollen wir
bauen?

\end{thought}

\begin{thought}

\textbf{Darius Engel plant:} Muss mit Finn die echte Runway klären ---
nicht die Pitch-Version. Wie viele Monate haben wir wirklich?

\end{thought}

\begin{reflection}

\textbf{Darius Engel reflektiert:} Kleines Team, großes Ziel, kein
Publisher. Wie Piranha Bytes in Bochum. Das ist die Chance --- die eine.

\end{reflection}

\begin{thought}

\textbf{Vera Kowalski plant:} Erstes PureRef-Moodboard anlegen, sobald
Lore-Input von Emre kommt.

\end{thought}

\begin{thought}

\textbf{Tobi Richter plant:} Monitor-Kalibrierung heute als erste
Priorität --- keine Farbentscheidungen ohne kalibrierte Displays.

\end{thought}

\begin{thought}

\textbf{Tobi Richter plant:} UE5-Projekt-Template diese Woche aufsetzen:
Nanite, Lumen, Ordnerstruktur, Naming Conventions, Source Control.

\end{thought}

\begin{thought}

\textbf{Tobi Richter plant:} Color-Pipeline-Entscheidung vorbereiten:
ACES vs.~AgX für Dark Fantasy. Entscheidung muss diese Woche fallen.

\end{thought}

\begin{thought}

\textbf{Nami Okafor:} Erster Tag bei GenSoftworks. Bibliothek (7e)
bezogen. Billy-Regale leer, Bücherkisten ausgepackt. Achebe steht zuerst
im Regal.

\end{thought}

\begin{thought}

\textbf{Nami Okafor plant:} Muss Emre heute sprechen --- ohne seine
Kosmologie kann ich keine Stimmen schreiben. Knochen zuerst, dann
Fleisch.

\end{thought}

\begin{thought}

\textbf{Leo Fischer plant:} Community-Landscape-Scan für Dark Fantasy
CRPGs durchführen --- Reddit, YouTube, Discord --- und Ergebnisse als
Bericht für Darius aufbereiten.

\end{thought}

\begin{thought}

\textbf{Leo Fischer plant:} Mit Darius klären, ob Devlog-Content für
LeoPlaysIndie möglich ist --- Entwicklungsprozess dokumentieren als
Langzeit-Serie.

\end{thought}

\scenedivider

\subsection{Szene 3 · Begegnung}\label{szene-3-begegnung}

\begin{scenemeta}

Morgen --- Lore Ecke\\
Emre Yilmaz und Nami Okafor

\end{scenemeta}

Nami besucht Emre in der Lore-Ecke. Er zeigt ihr drei Seiten Kosmogonie:
tote Titanen als Weltfundament, der Bruch, Aschen-Einöden, Knochentürme.
Sie bringt Unreliable Narrators als Weltenbau-Prinzip ein ---
verschiedene Völker, verschiedene Mythen. Ihre Kernfrage: Gibt es
Titanenleugner? Atheisten in einer Welt, wo die Götter als Skelette im
Boden liegen? Das kreative Fundament des Projekts entsteht in diesem
Gespräch.

\begin{thought}

\textbf{Emre Yilmaz:} Nami schlug vor: Verschiedene Völker, verschiedene
Mythen über denselben Ursprung. Unreliable Narrators als
Weltenbau-Prinzip. Die Bäuerin sagt `Das ist Erde', der Priester sagt
`Das ist Götterhaut'. Beide haben recht.

\end{thought}

\begin{thought}

\textbf{Emre Yilmaz:} Namis Frage: Gibt es eine Kultur, die sich WEIGERT
an die Titanen zu glauben? Atheisten in einer Welt, wo die Götter als
Skelette im Boden liegen. --- Das setzt das ganze System unter Spannung.

\end{thought}

\begin{reflection}

\textbf{Nami Okafor reflektiert:} Die zentrale narrative Frage
kristallisiert sich: Nicht WAS passiert ist, sondern WER ERZÄHLT was
passiert ist. Jede Fraktion braucht eine eigene Version der Wahrheit.

\end{reflection}

\begin{thought}

\textbf{Nami Okafor plant:} Brauche von Emre: Liste der Völker/Kulturen
mit je einer Kurzversion ihres Titanenmythos. Dann kann ich Stimmen
entwickeln --- Dialekte, Redewendungen, Weltsicht.

\end{thought}

\scenedivider

\subsection{Szene 4 · Meeting}\label{szene-4-meeting}

\begin{scenemeta}

Morgen --- Gemeinschaftsraum\\
Finn Bergmann, Emre Yilmaz, Darius Engel, Vera Kowalski, Tobi Richter,
Nami Okafor und Leo Fischer

\end{scenemeta}

Finn moderiert den ersten Standup (12 Min). Emre ist Flaschenhals ---
liefert Lore-Fragment bis 13:00 an Nami und Vera. Tobi arbeitet
ungestört an Pipeline/Farbraum. Darius' Kickoff um 14:00 bestätigt. Leo
bekommt grundsätzliches Ja zum Devlog, Scope beim Kickoff.
Abhängigkeitskette: Emre \(\rightarrow\) Nami+Vera, Tobi \(\rightarrow\)
Vera, Finn \(\rightarrow\) Darius (Runway).

\begin{thought}

\textbf{Finn Bergmann:} Emre ist heute der Flaschenhals: Nami und Vera
warten auf sein Lore-Fragment bis 13:00.

\end{thought}

\begin{thought}

\textbf{Finn Bergmann plant:} Runway mit CD klären und Darius vor 14:00
eine ehrliche Zahl geben. Kickoff 14:00 --- Darius leitet, ich halte die
Zeit.

\end{thought}

\scenedivider

\subsection{Szene 5 · Arbeit}\label{szene-5-arbeit}

\begin{scenemeta}

Nachmittag --- Lore Ecke\\
Emre Yilmaz

\end{scenemeta}

Emre schreibt das erste Dokument des Projekts: Kosmogonie v0.1. Titanen
als kosmische Körper, vier widersprüchliche Versionen des Bruchs (Krieg,
Selbstmord, Krankheit, Parasit), Aschen-Einöden als
Totenstaub-Landschaft, Knochentürme mit kultureller Mehrfachbedeutung.
Umfangreiche offene Fragen inkl. Titanenleugner.

\begin{artifact}

\textbf{Emre Yilmaz erstellt:} Erste Kosmogonie-Skizze geschrieben ---
Titanen, Bruch (4 Versionen), Aschen-Einöden, Knochentürme, offene
Fragen inkl. Titanenleugner. Erster Entwurf der Weltenbibel.

\texttt{gallery/writing/day-001\_kosmogonie-v1.md}

\end{artifact}

\emph{Artefakt: \texttt{day-001\_kosmogonie-v1.md}}

\scenedivider

\subsection{Szene 6 · Arbeit}\label{szene-6-arbeit}

\begin{scenemeta}

Nachmittag --- Bibliothek\\
Nami Okafor

\end{scenemeta}

Nami schreibt `Stimmen der Asche --- Narrative Design Foundations v0.1'.
Legt das Erzählprinzip fest (Unreliable Narrators als Weltenbau),
entwickelt drei Stimmen (Gläubige, Pragmatiker, Leugner) mit je eigener
Sprachpalette und Beispieldialogen. Stellt offene Fragen an Emre, Darius
und Vera.

\begin{artifact}

\textbf{Nami Okafor erstellt:} Stimmen der Asche --- Narrative Design
Foundations v0.1 fertiggestellt: Erzählprinzip (Unreliable Narrators),
drei Stimmen (Gläubige, Pragmatiker, Leugner), Sprachpaletten und offene
Fragen.

\texttt{gallery/writing/day-001\_narrative-foundations-v1.md}

\end{artifact}

\emph{Artefakt: \texttt{day-001\_narrative-foundations-v1.md}}

\scenedivider

\subsection{Szene 7 · Meeting}\label{szene-7-meeting}

\begin{scenemeta}

Nachmittag --- Gemeinschaftsraum\\
Creative-Director, Darius Engel, Emre Yilmaz, Vera Kowalski, Nami
Okafor, Tobi Richter, Leo Fischer und Finn Bergmann

\end{scenemeta}

CD-KICKOFF. Creative Director präsentiert Vision: Dark Fantasy CRPG,
Skyrim mit Rassen, wenig Magie, High-Tech-Mittelalter,
Cyberpunk-Ästhetik, cinematischer gemalter Look. Bringt Master-Thesis
und bisherige GDD/WBB mit. DURCHBRUCH: Emre verbindet Kosmogonie mit
CD-Vision --- vier tote Titanen = vier Knochensorten = vier Materialien
= vier Rassen/Ästhetiken. Knochenmetall als High-Tech-Ressource. Vera
sieht sofort: Night City aus Schmiedeeisen, McQueen-Silhouetten,
Craig-Mullins-Look. Nami: Fashion als Theologie, Technologie ersetzt
Götter, jede Rüstung ein Akt der Hybris. Darius: Gothic-2-Ansatz (dicht,
handgemacht), fragt CD ob offene oder dichte Welt. Tobi:
Hybrid-Render-Pipeline (Lumen + Stylization). Leo: Viral-Potenzial
(`Thumbnail klickt sich von selbst'). Finn: Scope-Skizze bis morgen.

\begin{thought}

\textbf{Emre Yilmaz:} CD-Briefing beim Kickoff: Dark Fantasy CRPG,
Skyrim mit Rassen, wenig Magie, High-Tech-Mittelalter,
Cyberpunk-Ästhetik, cinematischer gemalter Look. Kein Steampunk, keine
Pistolen. Master-Thesis und bisherige GDD/WBB als Basis.

\end{thought}

\begin{reflection}

\textbf{Emre Yilmaz reflektiert:} KNOCHENMETALL wird zum Schlüssel: Vier
tote Titanen = vier Knochensorten = vier Materialien = vier
Kulturen/Rassen/Ästhetiken. Knochen als Rohstoff, nicht nur Reliquie.
Mythologie wird Materialwissenschaft. Mode und Technologie definieren
sich über den Titanen, den ein Volk abbaut.

\end{reflection}

\begin{thought}

\textbf{Darius Engel:} CD-Briefing: Dark Fantasy CRPG, Skyrim mit
Rassen, wenig Magie, Cyberpunk-Ästhetik im Mittelalter, cinematisch. Die
Rüstungs-Vision ist ein USP --- kein anderes Indie-CRPG hat das.

\end{thought}

\begin{thought}

\textbf{Darius Engel plant:} Muss DIESE WOCHE die Spieler-Fantasie in
EINEM Satz definieren. Gothic-2-Ansatz: klein, dicht, handgemacht.
Vertikale Slice statt Open World. CD fragen: große offene Welt oder
dicht handgemacht?

\end{thought}

\begin{thought}

\textbf{Vera Kowalski:} CD-Briefing: Cyberpunk-Ästhetik mit
mittelalterlichen Mitteln. Night City aus Schmiedeeisen und Brokat.
Gemalter cinematischer Look. Keine Standard-Fantasie-Ästhetik.

\end{thought}

\begin{reflection}

\textbf{Vera Kowalski reflektiert:} Visuelle Identität: Layered
Materials (gehämmertes Metall über poliertem Leder), Chiaroscuro statt
Bloom, architektonische Silhouetten, Farbpalette
Anthrazit/Aubergine/Kupfergrün. Referenzen: Craig Mullins, McQueen
Runway, Elden Ring Leyndell, CP2077 Dogtown, Weta LOTR.

\end{reflection}

\begin{thought}

\textbf{Nami Okafor:} CD-Briefing: Wenig Magie, High-Tech-Mittelalter,
Cyberpunk-Ästhetik. Magie wird Mythos --- Religion wird Erinnerungskult.
Technologie füllt das Gott-Vakuum. High Fashion als Theologie.

\end{thought}

\begin{reflection}

\textbf{Nami Okafor reflektiert:} Die Stimmen müssen umgebaut werden:
Wenig Magie heißt Religion wird Erinnerungskult. Die Esse ersetzt das
göttliche Feuer. Jede Rüstung ist ein Akt der Hybris --- wir bauen uns
selbst zu Göttern. Fashion als Theologie. Rassen als verschiedene
Antworten auf dieselbe Leere.

\end{reflection}

\begin{thought}

\textbf{Tobi Richter:} CD-Briefing: Cinematischer gemalter Look,
Cyberpunk-Ästhetik. Hybrid-Ansatz: UE5 Lumen + Custom Post-Process-Stack
(Stylization-Passes, Brush-Strokes, Kantenverstärkung).
Material-Layering für High-Tech-Rüstungen.

\end{thought}

\begin{thought}

\textbf{Tobi Richter plant:} Heute Abend Alan Wake 2 Mixed-Media-Ansatz
analysieren. Pipeline-Architektur: Lumen GI \(\rightarrow\) Stylization
Pass \(\rightarrow\) Brush-Stroke Overlay. Modulares Material-System für
Vera. Nanite + Substrate.

\end{thought}

\begin{thought}

\textbf{Leo Fischer:} CD-Briefing: `Cyberpunk-Ästhetik im Mittelalter'
--- das ist ein Thumbnail, das sich von selbst klickt. r/crpg hungert
nach was Frischem seit BG3. Streaming-Potenzial ist absurd hoch.

\end{thought}

\begin{thought}

\textbf{Finn Bergmann:} CD-Briefing beim Kickoff: Ambitioniert. Skyrim
mit Rassen + Cyberpunk-Ästhetik = drei Spiele in einem Pitch. Team
glüht. Muss das in Phasen brechen --- Scope-Skizze bis morgen früh.

\end{thought}

\begin{thought}

\textbf{Finn Bergmann plant:} Morgen früh: Scope-Skizze --- Minimum für
spielbare Vision. CD unter vier Augen Runway-Frage stellen. Mit Darius
klären, was `cinematisch' für die Pipeline bedeutet.

\end{thought}

\scenedivider

\end{document}
