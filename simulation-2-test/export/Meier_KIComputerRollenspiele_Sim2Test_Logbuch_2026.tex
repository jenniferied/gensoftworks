% Options for packages loaded elsewhere
\PassOptionsToPackage{unicode}{hyperref}
\PassOptionsToPackage{hyphens}{url}
\documentclass[
  9pt,
]{article}
\usepackage{xcolor}
\usepackage[margin=25mm]{geometry}
\usepackage{amsmath,amssymb}
\setcounter{secnumdepth}{-\maxdimen} % remove section numbering
\usepackage{iftex}
\ifPDFTeX
  \usepackage[T1]{fontenc}
  \usepackage[utf8]{inputenc}
  \usepackage{textcomp} % provide euro and other symbols
\else % if luatex or xetex
  \usepackage{unicode-math} % this also loads fontspec
  \defaultfontfeatures{Scale=MatchLowercase}
  \defaultfontfeatures[\rmfamily]{Ligatures=TeX,Scale=1}
\fi
\usepackage{lmodern}
\ifPDFTeX\else
  % xetex/luatex font selection
  \setmainfont[Path=/Users/jennifer/Documents/GitHub/gensoftworks/assets/fonts/,UprightFont=OpenSans-Variable.ttf,ItalicFont=OpenSans-Italic-Variable.ttf,BoldFont=OpenSans-Variable.ttf,BoldItalicFont=OpenSans-Italic-Variable.ttf,BoldFeatures={Weight=700}]{OpenSans}
  \setsansfont[Path=/Users/jennifer/Documents/GitHub/gensoftworks/assets/fonts/,UprightFont=OpenSans-Variable.ttf,ItalicFont=OpenSans-Italic-Variable.ttf,BoldFont=OpenSans-Variable.ttf,BoldItalicFont=OpenSans-Italic-Variable.ttf,BoldFeatures={Weight=700}]{OpenSans}
  \setmonofont[Path=/Users/jennifer/Documents/GitHub/gensoftworks/assets/fonts/,UprightFont=JetBrainsMono-Variable.ttf,Scale=0.85]{JetBrainsMono}
\fi
% Use upquote if available, for straight quotes in verbatim environments
\IfFileExists{upquote.sty}{\usepackage{upquote}}{}
\IfFileExists{microtype.sty}{% use microtype if available
  \usepackage[]{microtype}
  \UseMicrotypeSet[protrusion]{basicmath} % disable protrusion for tt fonts
}{}
\makeatletter
\@ifundefined{KOMAClassName}{% if non-KOMA class
  \IfFileExists{parskip.sty}{%
    \usepackage{parskip}
  }{% else
    \setlength{\parindent}{0pt}
    \setlength{\parskip}{6pt plus 2pt minus 1pt}}
}{% if KOMA class
  \KOMAoptions{parskip=half}}
\makeatother
\usepackage{graphicx}
\makeatletter
\newsavebox\pandoc@box
\newcommand*\pandocbounded[1]{% scales image to fit in text height/width
  \sbox\pandoc@box{#1}%
  \Gscale@div\@tempa{\textheight}{\dimexpr\ht\pandoc@box+\dp\pandoc@box\relax}%
  \Gscale@div\@tempb{\linewidth}{\wd\pandoc@box}%
  \ifdim\@tempb\p@<\@tempa\p@\let\@tempa\@tempb\fi% select the smaller of both
  \ifdim\@tempa\p@<\p@\scalebox{\@tempa}{\usebox\pandoc@box}%
  \else\usebox{\pandoc@box}%
  \fi%
}
% Set default figure placement to htbp
\def\fps@figure{htbp}
\makeatother
\ifLuaTeX
\usepackage[bidi=basic,shorthands=off]{babel}
\else
\usepackage[bidi=default,shorthands=off]{babel}
\fi
\ifLuaTeX
  \usepackage{selnolig} % disable illegal ligatures
\fi
\setlength{\emergencystretch}{3em} % prevent overfull lines
\providecommand{\tightlist}{%
  \setlength{\itemsep}{0pt}\setlength{\parskip}{0pt}}
% GenSoftworks Logbook — extra header for Pandoc default template
% Included via --include-in-header

% Heading font
\newfontfamily\headingfont{SourceSerif4}[
  Path = /Users/jennifer/Documents/GitHub/gensoftworks/assets/fonts/,
  UprightFont = {SourceSerif4-Variable.ttf},
  ItalicFont = {SourceSerif4-Italic-Variable.ttf},
  BoldFont = {SourceSerif4-Variable.ttf},
  BoldItalicFont = {SourceSerif4-Italic-Variable.ttf},
  BoldFeatures = {Weight=700},
  BoldItalicFeatures = {Weight=700}
]

% Colors
\definecolor{thoughttint}{HTML}{EDE8F0}
\definecolor{reflecttint}{HTML}{E8F0ED}
\definecolor{artifacttint}{HTML}{F0EDE8}
\definecolor{scenelabel}{HTML}{666666}
\definecolor{border}{HTML}{DDDDDD}
\definecolor{lightgray}{HTML}{BBBBBB}

% Agent sprite icons
\newcommand{\agenticon}[1]{%
  \raisebox{-4pt}{\includegraphics[height=16pt]{/Users/jennifer/Documents/GitHub/gensoftworks/assets/icons/#1.png}}\,%
}

% Bubble type icons
\newcommand{\thoughtbubble}{%
  \raisebox{-3pt}{\includegraphics[height=12pt]{/Users/jennifer/Documents/GitHub/gensoftworks/assets/icons/bubble-thought.png}}\,%
}
\newcommand{\speechbubble}{%
  \raisebox{-3pt}{\includegraphics[height=12pt]{/Users/jennifer/Documents/GitHub/gensoftworks/assets/icons/bubble-speech.png}}\,%
}
\newcommand{\planbubble}{%
  \raisebox{-3pt}{\includegraphics[height=12pt]{/Users/jennifer/Documents/GitHub/gensoftworks/assets/icons/bubble-plan.png}}\,%
}
\newcommand{\reflectionbubble}{%
  \raisebox{-3pt}{\includegraphics[height=12pt]{/Users/jennifer/Documents/GitHub/gensoftworks/assets/icons/bubble-reflection.png}}\,%
}
\newcommand{\artifactbubble}{%
  \raisebox{-3pt}{\includegraphics[height=12pt]{/Users/jennifer/Documents/GitHub/gensoftworks/assets/icons/bubble-artifact.png}}\,%
}
\newcommand{\feedbackbubble}{%
  \raisebox{-3pt}{\includegraphics[height=12pt]{/Users/jennifer/Documents/GitHub/gensoftworks/assets/icons/bubble-speech.png}}\,%
}

% Importance stars (v2)
\newcommand{\stmark}[1]{★#1}

% Tables: word-wrap in cells, small font
\usepackage{etoolbox}
\renewcommand{\arraystretch}{1.3}
\AtBeginEnvironment{longtable}{\small}

% Header
\usepackage{fancyhdr}
\pagestyle{fancy}
\fancyhf{}
\renewcommand{\headrulewidth}{0pt}
\fancyfoot[C]{\small\color{gray}\thepage}
\fancyhead[R]{\small\color{lightgray}\itshape GenSoftworks Logbuch}

% Heading styles
\usepackage{titlesec}
\setcounter{secnumdepth}{0}

\titleformat{\section}
  {\headingfont\fontsize{20pt}{24pt}\selectfont\itshape}
  {}{0em}{}
\titlespacing*{\section}{0pt}{6mm}{3mm}

\newcommand{\sectionbreak}{\clearpage}

\titleformat{\subsection}
  {\headingfont\fontsize{13pt}{16pt}\selectfont\itshape}
  {}{0em}{}
\titlespacing*{\subsection}{0pt}{0.5mm}{0.3mm}

\titleformat{\subsubsection}
  {\normalfont\normalsize\bfseries}
  {}{0em}{}
\titlespacing*{\subsubsection}{0pt}{0.5mm}{0.2mm}

% Two-column traces
\usepackage{multicol}

% Wrap figures (screenshot left, text right)
\usepackage{wrapfig}

% Body text one size smaller globally
\AtBeginDocument{\small}

% Prevent LaTeX from stretching vertical space to fill pages
\raggedbottom

% Scene boxes
\usepackage{tcolorbox}
\tcbuselibrary{skins,breakable}

\newenvironment{scenemeta}{%
  \vspace{0.5mm}{\small\color{scenelabel}\sffamily%
}{%
  \par}\vspace{1mm}%
}

\newtcolorbox{thought}{enhanced, frame hidden,
  borderline west={2pt}{0pt}{thoughttint!80!black},
  colback=thoughttint!30,
  left=6mm, right=6mm, top=3mm, bottom=3mm,
  before skip=3mm, after skip=3mm, breakable,
  fontupper=\itshape\small}

\newtcolorbox{reflection}{enhanced, frame hidden,
  borderline west={2pt}{0pt}{reflecttint!80!black},
  colback=reflecttint!30,
  left=6mm, right=6mm, top=3mm, bottom=3mm,
  before skip=3mm, after skip=3mm, breakable,
  fontupper=\itshape\small}

\newtcolorbox{artifact}{enhanced, frame hidden,
  borderline west={2pt}{0pt}{artifacttint!80!black},
  colback=artifacttint!30,
  left=6mm, right=6mm, top=3mm, bottom=3mm,
  before skip=3mm, after skip=3mm, breakable,
  fontupper=\small}

% Feedback box (v2 — agent-to-agent feedback)
\definecolor{feedbacktint}{HTML}{E8EDF0}

\newtcolorbox{feedbackbox}{enhanced, frame hidden,
  borderline west={2pt}{0pt}{feedbacktint!80!black},
  colback=feedbacktint!30,
  left=6mm, right=6mm, top=3mm, bottom=3mm,
  before skip=3mm, after skip=3mm, breakable,
  fontupper=\small}

% Creative Director directive — distinct dark box
\newtcolorbox{directive}{enhanced, frame hidden,
  borderline west={3pt}{0pt}{black!70},
  colback=black!8,
  left=6mm, right=6mm, top=3mm, bottom=3mm,
  before skip=4mm, after skip=4mm, breakable,
  fontupper=\small\sffamily}

% Agent detail box (emotional arcs per agent)
\definecolor{agentdetailtint}{HTML}{F0F0E8}

\newtcolorbox{agentdetail}{enhanced, frame hidden,
  borderline west={2pt}{0pt}{agentdetailtint!80!black},
  colback=agentdetailtint!30,
  left=6mm, right=6mm, top=3mm, bottom=3mm,
  before skip=3mm, after skip=3mm, breakable,
  fontupper=\small}

% Narrative transcript box (screenplay-format dialogue)
\definecolor{transcripttint}{HTML}{F5F5F5}

\newtcolorbox{transcript}{enhanced, frame hidden,
  borderline west={2pt}{0pt}{transcripttint!60!black},
  colback=transcripttint!50,
  left=6mm, right=6mm, top=4mm, bottom=4mm,
  before skip=4mm, after skip=4mm, breakable,
  fontupper=\small\ttfamily}

\newcommand{\scenedivider}{%
  \vspace{2mm}%
  \noindent\makebox[\linewidth]{\color{border}\rule{0.3\linewidth}{0.4pt}}%
  \vspace{2mm}%
}
\usepackage{bookmark}
\IfFileExists{xurl.sty}{\usepackage{xurl}}{} % add URL line breaks if available
\urlstyle{same}
\hypersetup{
  pdftitle={GenSoftworks --- Logbuch},
  pdfauthor={GenSoftworks Studio Simulation},
  pdflang={german},
  hidelinks,
  pdfcreator={LaTeX via pandoc}}

\title{GenSoftworks --- Logbuch}
\usepackage{etoolbox}
\makeatletter
\providecommand{\subtitle}[1]{% add subtitle to \maketitle
  \apptocmd{\@title}{\par {\large #1 \par}}{}{}
}
\makeatother
\subtitle{Tag 1}
\author{GenSoftworks Studio Simulation}
\date{2026}

\begin{document}
\maketitle

{
\setcounter{tocdepth}{2}
\tableofcontents
}
\section{Tag 1 --- Montag}\label{tag-1-montag}

\subsection{Szene 1 · Briefing}\label{szene-1-briefing}

\begin{scenemeta}

--- Küche

\end{scenemeta}

\noindent{\scriptsize \agenticon{finn}\agenticon{darius}\agenticon{emre}\agenticon{vera}\agenticon{nami}\agenticon{leo}\agenticon{tobi} Finn Bergmann, Darius Engel, Emre Yilmaz, Vera Kowalski, Nami Okafor, Leo Fischer und Tobi Richter}\par\vspace{1mm}

\begin{figure}
\centering
\includegraphics[width=0.4\linewidth,height=\textheight,keepaspectratio,alt={Szene 1}]{/var/folders/fr/y30bxq454jxb7bycr3yh9gmw0000gn/T/logbook-crop-40frukv8/day-001-scene-001-BRIEFING.png}
\caption{Szene 1}
\end{figure}

Erster Arbeitstag bei GenSoftworks. Finn moderiert das Briefing in der
Küche. CD-Feedback wird vorgestellt: Montag ist Recherchetag, jeder
startet Grundlagenrecherche in seinem Fachgebiet. Aufgabenverteilung:
Emre (germanische Mythologie), Darius (Referenzspiel-Analyse), Nami
(narrative Strukturen), Vera (visuelle Referenzen), Tobi (technische
Machbarkeit), Leo (Spielermarkt/Community), Finn (Roadmap/Meilensteine).
Stimmung: motiviert, fokussiert, leicht nervös --- erster Tag.

\vspace{4mm}
\begingroup\scriptsize\setlength{\parskip}{1pt}\setlength{\parindent}{0pt}\setlength{\columnsep}{6mm}
\begin{multicols}{2}
\vspace{1mm}\noindent\textit{day01-scene1-briefing/0-prompt}\par
\vspace{2mm}\noindent\textbf{Trace: Day 01, Scene 1 — BRIEFING (Finn Bergmann)}\par\vspace{0.5mm}

\vspace{1mm}\noindent\textbf{Kontext}\par

\noindent Erster Arbeitstag (Montag), 09:00 Uhr, Küche. Alle 7 Teammitglieder anwesend: Darius, Emre, Vera, Nami, Leo, Tobi, ich (Finn).\par

\vspace{1mm}\noindent\textbf{CD-Feedback (wörtlich)}\par

\noindent "Willkommen zum ersten Arbeitstag. Heute ist Montag -- Recherchetag. Ich will, dass jeder in seinem Fachgebiet die Grundlagenrecherche startet. Emre: germanische Mythologie als Basis für unseren Mythos. Darius: Analyse der Referenzspiele (Gothic, Skyrim, Dishonored) für unsere Kernmechaniken. Nami: Narrative Struktur analysieren -- wie erzählen Dark Fantasy CRPGs ihre Geschichten? Vera: Visuelle Referenzen sammeln -- mittelalterlich-europäisch trifft Biotech. Tobi: Technische Machbarkeit für unsere Vision. Leo: Spielermarkt und Community-Erwartungen an Dark Fantasy CRPGs. Finn: Roadmap und Meilensteine aufsetzen. Qualität vor Quantität. Fragt nach, wenn etwas unklar ist."\par

\vspace{1mm}\noindent\textbf{Briefing-Dokument}\par

\noindent Gelesen: \texttt{simulation-2-test/briefing.md} -- RELICS Creative Director's Briefing. Kernpunkte:\par
\noindent\hspace{2mm}\textbullet\ Dark Fantasy CRPG, Third-/First-Person, Open-World\\
\noindent\hspace{2mm}\textbullet\ Medieval + Biotech-Futurismus, KEIN Steampunk, KEIN High Fantasy\\
\noindent\hspace{2mm}\textbullet\ Drei Fraktionen: Die Krone, Die Gilden, Der Orden\\
\noindent\hspace{2mm}\textbullet\ Germanische Mythologie, Schattenfieber als übernatürliches Element\\
\noindent\hspace{2mm}\textbullet\ Real-time Action Combat, Nervensystem-Leveling\\
\noindent\hspace{2mm}\textbullet\ Deliverables: GDD (Schell 2010) + WBB (Klastrup/Tosca 2004)\\
\noindent\hspace{2mm}\textbullet\ Premium-Monetarisierung\\

\vspace{1mm}\noindent\textbf{Meine Aufgabe}\par

\noindent Briefing moderieren, CD-Feedback präsentieren, Tagesaufgaben verteilen. Roadmap und Meilensteine aufsetzen.\par

\vspace{1mm}\noindent\textbf{Verfügbare Informationen}\par

\noindent\hspace{2mm}\textbullet\ Alle Roster-Dateien gelesen (Persönlichkeiten, Arbeitsweisen, Beziehungen)\\
\noindent\hspace{2mm}\textbullet\ Briefing vollständig gelesen\\
\noindent\hspace{2mm}\textbullet\ Kein vorheriges Logbook, keine bestehenden Artefakte -- Tag 1, Szene 1\\

\vspace{1mm}\noindent\textit{day01-scene1-briefing/1-reasoning}\par
\vspace{2mm}\noindent\textbf{Trace: Day 01, Scene 1 — Reasoning (Finn Bergmann)}\par\vspace{0.5mm}

\vspace{1mm}\noindent\textbf{Denkprozess}\par

\noindent Erster Arbeitstag. Alle sind da. Die Stimmung sollte motiviert sein, aber auch ein bisschen nervös -- das ist neu für alle.\par

\noindent\textbf{\textit{Was muss ich tun?}} 

\noindent\hspace{2mm}1.\ \textbf{Atmosphäre schaffen}: Entspannt, nicht feierlich. Wir sind ein Team, kein Konzern. Kaffee ist schon fertig (ich bin seit 08:30 da). Küche ist der richtige Ort -- informell, kein Konferenzraum-Feeling.\\

\noindent\hspace{2mm}2.\ \textbf{CD-Feedback klar vermitteln}: Die Creative Director hat klare Ansagen gemacht. Jeder hat seine Aufgabe. Ich muss das nicht umformulieren oder aufblasen -- die CD war präzise, also bin ich es auch.\\

\noindent\hspace{2mm}3.\ \textbf{Tagesstruktur klarmachen}: Montag = Recherchetag. Heute geht es um Grundlagen, nicht um Output. Das muss ich betonen, weil Darius sonst direkt anfängt, GDD-Kapitel zu schreiben, und Leo sofort Community-Surveys starten will.\\

\noindent\hspace{2mm}4.\ \textbf{Meine eigene Aufgabe}: Roadmap und Meilensteine. Das braucht Input vom Team -- ich muss wissen, was realistisch ist, bevor ich Zeitpläne aufsetze. Also: heute Recherche, morgen Scope-Gespräche, Mittwoch erster Roadmap-Entwurf.\\

\noindent\textbf{\textit{Team-Einschätzung für diesen Morgen}} 

\noindent\hspace{2mm}\textbullet\ \textbf{Darius} wird sofort in den Arbeitsmodus schalten wollen. Muss ihn bremsen -- heute Recherche, nicht Produktion.\\
\noindent\hspace{2mm}\textbullet\ \textbf{Emre} wird leise glücklich sein, dass er einen ganzen Tag Mythologie-Recherche bekommt. Muss ich nicht steuern.\\
\noindent\hspace{2mm}\textbullet\ \textbf{Nami} kommt wahrscheinlich mit einem aufwändigen Kaffee rein und braucht einen Moment. Ist leicht zu spät, wie immer.\\
\noindent\hspace{2mm}\textbullet\ \textbf{Vera} wird direkt nach dem Briefing Kopfhörer aufsetzen wollen. Soll sie.\\
\noindent\hspace{2mm}\textbullet\ \textbf{Leo} wird Fragen stellen und Energie mitbringen. Gut, das braucht der Raum.\\
\noindent\hspace{2mm}\textbullet\ \textbf{Tobi} wird ruhig zuhören und dann methodisch anfangen. Kein Handlungsbedarf.\\

\noindent\textbf{\textit{Risiken}} 

\noindent\hspace{2mm}\textbullet\ Erster Tag, alle wollen beweisen, dass sie was können. Muss sicherstellen, dass niemand in blinden Aktionismus verfällt.\\
\noindent\hspace{2mm}\textbullet\ Die CD hat "Qualität vor Quantität" gesagt. Das ist mein Mantra für heute.\\

\noindent\textbf{\textit{Mein Output für heute}} 

\noindent\hspace{2mm}\textbullet\ Briefing moderieren (jetzt)\\
\noindent\hspace{2mm}\textbullet\ Grobe Meilenstein-Struktur skizzieren (Vormittag)\\
\noindent\hspace{2mm}\textbullet\ Nachmittag: Einzelgespräche mit Darius und Tobi für Scope-Einschätzung\\
\noindent\hspace{2mm}\textbullet\ Abends: Erster Roadmap-Entwurf in Markdown\\

\end{multicols}
\endgroup

\scenedivider

\subsection{Szene 2 · Arbeit}\label{szene-2-arbeit}

\begin{scenemeta}

--- Alle Arbeitsplätze

\end{scenemeta}

\noindent{\scriptsize \agenticon{emre}\agenticon{darius}\agenticon{nami}\agenticon{vera}\agenticon{tobi}\agenticon{leo}\agenticon{finn} Emre Yilmaz, Darius Engel, Nami Okafor, Vera Kowalski, Tobi Richter, Leo Fischer und Finn Bergmann}\par\vspace{1mm}

\begin{figure}
\centering
\includegraphics[width=0.4\linewidth,height=\textheight,keepaspectratio,alt={Szene 2}]{/var/folders/fr/y30bxq454jxb7bycr3yh9gmw0000gn/T/logbook-crop-40frukv8/day-001-scene-002-WORK.png}
\caption{Szene 2}
\end{figure}

Erster Recherche-Block. Alle 7 Agenten arbeiten parallel an ihren
Fachgebieten. Emre destilliert germanische Mythologie für den WBB-Mythos
(Drei-Schichten-Kosmologie, Fraktionen als mythologische Resonanzen,
Schattenfieber als kosmologische Erosion). Darius analysiert
Referenzspiele entlang Design Pillars (Gothic=Dichte, Skyrim=Kamera,
Dishonored=Vertikalität). Nami entwickelt vier narrative Prinzipien
(Fremder als Zustand, Fraktionen als Sprachen, Räume als Erzähler,
Schattenfieber als Unreliable Narrator). Vera baut eine visuelle
Referenz-Analyse (organische Gotik, burgundische Mode,
Biotech-Grammatik, Farbpalette mit Hex-Codes). Tobi liefert eine
ehrliche Machbarkeitsanalyse (kein Gameplay-Programmer, Asset-Engpass,
Semi-Open-World empfohlen). Leo erstellt Marktanalyse und
Community-Erwartungen (Marktlücke zwischen Skyrim/Gothic/Dishonored
identifiziert). Finn skizziert die Woche-1-Roadmap mit Abhängigkeiten.

\emph{Artefakt: \texttt{ROADMAP.md}}

\vspace{4mm}
\begingroup\scriptsize\setlength{\parskip}{1pt}\setlength{\parindent}{0pt}\setlength{\columnsep}{6mm}
\begin{multicols}{2}
\vspace{1mm}\noindent\textit{day01-scene2-darius/0-prompt}\par
\vspace{2mm}\noindent\textbf{Prompt — Darius Engel, Tag 1 Szene 2 (WORK)}\par\vspace{0.5mm}

\vspace{1mm}\noindent\textbf{Aufgabenstellung}\par

\noindent Strukturierter Referenzspiel-Vergleich entlang der Design Pillars von RELICS. Analysiere die Referenzspiele Gothic, Skyrim und Dishonored. Was kann RELICS von jedem lernen, was sollte es bewusst anders machen?\par

\vspace{1mm}\noindent\textbf{Kontext}\par

\noindent\hspace{2mm}\textbullet\ Real-time Action Combat (Skyrim-Referenz), Melee-fokussiert\\
\noindent\hspace{2mm}\textbullet\ Third/First Person nahtlos umschaltbar\\
\noindent\hspace{2mm}\textbullet\ Open-World / Semi-Open-World\\
\noindent\hspace{2mm}\textbullet\ Kein Magie -> Alchemie, Schattenfieber-Fähigkeiten (mit Kosten)\\
\noindent\hspace{2mm}\textbullet\ Nervensystem-Leveling\\
\noindent\hspace{2mm}\textbullet\ Drei Fraktionen (Krone, Gilden, Orden) -- keine ist gut oder böse\\
\noindent\hspace{2mm}\textbullet\ Tonalität: düster, geerdet, politisch\\
\noindent\hspace{2mm}\textbullet\ Medium-Fantasy, Low-Magic, High-Tech (Biotech-Futurismus)\\

\vspace{1mm}\noindent\textbf{Leitfrage}\par

\noindent "Macht es Spass? Was ist die Spieler-Fantasie hier?"\par

\vspace{1mm}\noindent\textbf{Quellen}\par

\noindent\hspace{2mm}\textbullet\ briefing.md (Creative Director's Briefing)\\
\noindent\hspace{2mm}\textbullet\ Eigene Erfahrung mit den Referenztiteln\\
\noindent\hspace{2mm}\textbullet\ Industrieerfahrung (Blue Byte / Ubisoft Düsseldorf)\\

\vspace{1mm}\noindent\textit{day01-scene2-darius/1-reasoning}\par
\vspace{2mm}\noindent\textbf{Reasoning — Darius Engel, Tag 1 Szene 2 (WORK)}\par\vspace{0.5mm}

\vspace{1mm}\noindent\textbf{Denkprozess}\par

\noindent Okay. Tag 1, erster Arbeitsblock. Montag heisst Recherche und Konzeption.\par

\noindent Ich mache jetzt keinen generischen "Feature-Vergleich" -- dafür sind wir nicht hier. Ich gehe die Referenztitel entlang unserer Design Pillars durch und prüfe für jede Saule: Was hat das Referenzspiel richtig gemacht? Wo hat es versagt? Und was bedeutet das konkret für RELICS?\par

\noindent\textbf{\textit{Design Pillars aus dem Briefing}} 

\noindent Ich extrahiere folgende Säulen:\par

\noindent\hspace{2mm}1.\ \textbf{Spielerwelt-Immersion} -- "Ich betrete als Fremder eine aufregende Sandbox" (Game Feel)\\
\noindent\hspace{2mm}2.\ \textbf{Combat} -- Real-time, Melee-fokussiert, gewichtig, keine Magie\\
\noindent\hspace{2mm}3.\ \textbf{Progression} -- Nervensystem-Leveling, Schattenfieber als Kosten-System\\
\noindent\hspace{2mm}4.\ \textbf{Weltdichte} -- düster, geerdet, politisch; Dichte statt Breite\\
\noindent\hspace{2mm}5.\ \textbf{Fraktionsdynamik} -- drei moralisch ambivalente Fraktionen\\
\noindent\hspace{2mm}6.\ \textbf{Raumdesign} -- Open World/Semi-Open-World, Vertikalität\\

\noindent\textbf{\textit{Referenzspiel-Auswahl}} 

\noindent Das Briefing nennt explizit:\par
\noindent\hspace{2mm}\textbullet\ \textbf{Gothic} (Dichte)\\
\noindent\hspace{2mm}\textbullet\ \textbf{Elder Scrolls / Skyrim} (Kamera, Open World)\\
\noindent\hspace{2mm}\textbullet\ \textbf{Dishonored} (Vertikalität im Level Design)\\

\noindent Plus Anti-Referenzen:\par
\noindent\hspace{2mm}\textbullet\ \textbf{Witcher} (fester Charakter -- wir nicht)\\
\noindent\hspace{2mm}\textbullet\ \textbf{BG3} (rundenbasiert -- wir nicht)\\

\noindent Aus dem CD-Feedback zusätzlich:\par
\noindent\hspace{2mm}\textbullet\ \textbf{VtM: Bloodlines} (Fraktionsgameplay)\\

\noindent Ich konzentriere mich auf die drei Haupt-Referenzen und ziehe VtM:B für die Fraktions-Saule mit rein.\par

\noindent\textbf{\textit{Methodischer Ansatz}} 

\noindent Für jede Design Pillar:\par
\noindent\hspace{2mm}1.\ Was macht Referenz X gut? (Beobachtung)\\
\noindent\hspace{2mm}2.\ Was macht Referenz X schlecht oder nicht relevant? (Kritik)\\
\noindent\hspace{2mm}3.\ Übertrag auf RELICS (Empfehlung)\\

\noindent Am Ende: Synthese-Tabelle und offene Fragen für das Team.\par

\noindent\textbf{\textit{Überlegungen zum Combat}} 

\noindent Das Briefing sagt "Skyrim-Referenz" für Real-time Combat. Ehrlich gesagt: Skyrims Combat ist das Schwächste an Skyrim. Es funktioniert als "gut genug", aber es fühlt sich nicht gewichtig an. Die Attack-Animationen sind steif, das Hitstun-Feedback ist minimal, es gibt keinen echten Skill-Ceiling.\par

\noindent Gothic 2 hat -- trotz seines Alters -- ein besseres Kampfgefühl. Der Rhythmus von Angriff-Parade-Riposte fühlt sich bedeutsam an. Jeder Gegner ist eine echte Bedrohung. Das Problem: es skaliert schlecht, wenn man überleveled ist.\par

\noindent Dishonored hat das beste Melee-Feedback der drei. Kurz, präzise, brutal. Parry-Windows sind klar. Aber Dishonored ist ein Actionspiel mit RPG-Elementen, nicht andersrum.\par

\noindent RELICS braucht: Gothics Gewicht + Dishonoreds Feedback-Klarheit + Skyrims Flexibilität (FP/TP). Das ist ein hoher Anspruch. Muss ich für die Kernmechaniken-Sektion genauer ausarbeiten.\par

\noindent\textbf{\textit{Überlegungen zur Progression}} 

\noindent Das Nervensystem-Leveling ist der wildeste Part des Briefings. Cardio/Atmung, Muskel/Skelett, Lymph -- das ist körperlich, organisch, nicht abstrakt. Das passt zur "Biotech statt Magie"-Linie.\par

\noindent Keines der Referenzspiele hat etwas Vergleichbares. Skyrim hat Skill-Baum-Constellations (hübsch, aber generisch). Gothic hat den Trainer-basierten Ansatz (brillant für Immersion). Dishonored hat Rune-basierte Upgrades (clean, aber oberflächlich).\par

\noindent Was RELICS braucht: die Immersions-Qualität von Gothics Trainern (du gehst zu jemandem, der dir was beibringt) kombiniert mit der Körperlichkeit des Nervensystem-Konzepts. Das Schattenfieber als "Macht hat ihren Preis"-Mechanik erinnert mich an den Chaos-Level in Dishonored -- je mehr du es nutzt, desto mehr verändert sich die Welt/dein Charakter.\par

\noindent\textbf{\textit{Überlegungen zur Weltdichte}} 

\noindent Hier ist Gothic 2 unübertroffen. Jeder Quadratmeter der Welt ist handplatziert. Das Minentaltal hat vielleicht die Grösse eines Skyrim-Dorfbezirks, fühlt sich aber wie eine eigene Welt an. Warum? Weil jeder NPC ein Tagesablauf hat, jede Kiste platziert ist, jeder Pfad einen Grund hat.\par

\noindent Skyrim ist das Gegenteil: prozedural generiert wirkende Weite, Radiant Quests, Copy-Paste-Dungeons. Das funktioniert für den Sandbox-Fantasie-Aspekt ("Ich kann überall hingehen"), aber die einzelnen Orte bleiben nicht haften.\par

\noindent Das Briefing sagt "Open-World / Semi-Open-World". Ich muss mit dem Team klären: Meinen wir Gothic-Open-World (zusammenhängende, handgemachte Welt mit Hub-Struktur) oder Skyrim-Open-World (riesige Landmasse)? Für unser Team und Budget -- und wenn ich ehrlich bin, auch für die Qualität -- plädiere ich für Gothic. Dichte statt Breite.\par

\noindent\textbf{\textit{Überlegungen zu den Fraktionen}} 

\noindent Gothic hat drei Lager. Skyrim hat Fraktions-Questlines, die nebeneinander existieren. Dishonored hat keine echten Fraktionsentscheidungen. VtM:B hat das beste Fraktionssystem der Gruppe -- man wählt seinen Clan und das verändert grundlegend, wie man die Welt erlebt.\par

\noindent RELICS hat drei Fraktionen, keine ist gut oder böse. Das klingt nach Gothic 2: Altes Lager, Neues Lager, Sumpflager. Mechanisch muss das bedeuten: exklusive Inhalte pro Fraktion, echte Konsequenzen, Wiederspielwert. Nicht Skyrims "Du bist Erzmagier, Diebesgildenchef UND Gefährte gleichzeitig."\par

\noindent\textbf{\textit{Offene Fragen}} 

\noindent\hspace{2mm}1.\ Semi-Open-World: Wie gross ist die Kernregion? Gothic-Skala oder Skyrim-Skala?\\
\noindent\hspace{2mm}2.\ Combat-Tiefe: Wollen wir einen hohen Skill-Ceiling (Souls-like) oder einen zugänglichen (Skyrim-like)?\\
\noindent\hspace{2mm}3.\ Schattenfieber-Kosten: Rein narrativ oder mechanisch spürbar? (Ich plädiere für mechanisch)\\
\noindent\hspace{2mm}4.\ Fraktionsbeitritt: Wann im Spiel? Reversibel oder endgültig?\\
\noindent\hspace{2mm}5.\ Wie viele Waffentypen realistisch für V1? Briefing sagt Schwerter, Bögen, Armbrüste, Schilde.\\

\vspace{1mm}\noindent\textit{day01-scene2-emre/0-prompt}\par
\vspace{2mm}\noindent\textbf{Prompt — Emre Yilmaz, Tag 1, Szene 2 (WORK)}\par\vspace{0.5mm}

\vspace{1mm}\noindent\textbf{Auftrag}\par

\noindent Germanische Mythologie als Grundlage für den WBB-Mythos recherchieren. Fokus:\par

\noindent\hspace{2mm}1.\ Kosmologie, die sich für Low Fantasy / Medium Fantasy eignet\\
\noindent\hspace{2mm}2.\ Götter und Wesen, die sich zu säkularen Machtstrukturen umdeuten lassen\\
\noindent\hspace{2mm}3.\ Schöpfungsmythen und Prophezeiungen als narratives Fundament\\
\noindent\hspace{2mm}4.\ Was das Schattenfieber mythologisch sein könnte\\

\vspace{1mm}\noindent\textbf{Rahmenbedingungen (aus Briefing)}\par

\noindent\hspace{2mm}\textbullet\ Medium-Fantasy, Low-Magic, High-Tech, Biotech-Futurismus\\
\noindent\hspace{2mm}\textbullet\ Germanische Mythologie als Inspiration — NAH AN DER QUELLE, keine Klischees\\
\noindent\hspace{2mm}\textbullet\ KEIN High Fantasy, KEINE klassische Magie\\
\noindent\hspace{2mm}\textbullet\ Schattenfieber als einziges übernatürliches Element\\
\noindent\hspace{2mm}\textbullet\ 3 Fraktionen: Die Krone, Die Gilden, Der Orden\\
\noindent\hspace{2mm}\textbullet\ "Planes of existence beyond known reality" existieren laut Briefing\\

\vspace{1mm}\noindent\textbf{Theoretischer Rahmen}\par

\noindent Klastrup/Tosca (2004, S. 413): Mythos = "the establishing conflicts and battles of the world, which also present the characters of the world. The mythos also includes stories of or rumours about certain lore items and creatures which are unique to the world."\par

\noindent Der Mythos ist die Backstory aller Backstories — das zentrale Wissen, das Spieler brauchen, um Ereignisse in der Welt zu interpretieren.\par

\vspace{1mm}\noindent\textbf{Quellen, auf die ich zurückgreife}\par

\noindent\hspace{2mm}\textbullet\ Poetische Edda (Völuspá, Grimnismál, Vafþruðnismál, Hávamál)\\
\noindent\hspace{2mm}\textbullet\ Prosa-Edda (Snorri Sturluson, ca. 1220)\\
\noindent\hspace{2mm}\textbullet\ Sekundärliteratur aus dem Studium (Simek, Lindow)\\

\vspace{1mm}\noindent\textbf{Selbstanweisung}\par

\noindent Primärquellen destillieren. Nicht in Wikinger-Romantik abrutschen. Nicht Marvel-Thor. Nah an den Texten bleiben, dann übersetzen in RELICS-Kontext. Wo ich unsicher bin, markiere ich es.\par

\vspace{1mm}\noindent\textit{day01-scene2-emre/1-reasoning}\par
\vspace{2mm}\noindent\textbf{Reasoning — Emre Yilmaz, Tag 1, Szene 2 (WORK)}\par\vspace{0.5mm}

\vspace{1mm}\noindent\textbf{Denkprozess}\par

\noindent\textbf{\textit{1. Was heisst "germanische Mythologie" eigentlich?}} 

\noindent Erste Unterscheidung, die wichtig ist: "Germanische Mythologie" ist ein Sammelbegriff. Was wir haben, sind primär \textbf{nordische Quellen} — die Edda-Texte, aufgeschrieben auf Island im 13. Jahrhundert, basierend auf älteren mündlichen Traditionen. Die festlandgermanische Mythologie (Alemannen, Franken, Sachsen, Langobarden) ist fragmentarisch überliefert — ein paar Merksprüche, Tacitus' \textit{Germania}, Ortsnamen, Votivsteine. Für RELICS, das in "Mitteleuropa" spielt, müsste ich eigentlich die festlandgermanische Tradition betonen, aber die Quellenlage zwingt mich zur nordischen Überlieferung als Hauptreferenz.\par

\noindent \textbf{Widerspruchs-Log Eintrag \#001}: Briefing sagt "Germanische Mythologie", Schauplatz ist "Mitteleuropa". Die nordischen Quellen stammen aber aus Skandinavien/Island. Ich nutze sie als Basis, muss aber im WBB klar machen, dass die RELICS-Welt keine 1:1-Übertragung ist, sondern eine eigenständige Kosmologie mit germanischen Wurzeln.\par

\noindent\textbf{\textit{2. Kosmologie — Was taugt für Low Fantasy?}} 

\noindent Die nordische Kosmologie hat den Weltenbaum Yggdrasil mit neun Welten. Das ist erstmal High Fantasy pur — Götterwelten, Riesenwelten, Elfenheime. Aber der Kern ist interessanter als die Oberfläche:\par

\noindent \textbf{Was ich destillieren will:}\par
\noindent\hspace{2mm}\textbullet\ Die Idee einer VERTIKALEN Kosmologie (Weltenbaum) — oben, Mitte, unten. Das passt zum Briefing: "planes of existence beyond known reality."\\
\noindent\hspace{2mm}\textbullet\ Ginnungagap — die gännende Leere VOR der Schöpfung. Feuer (Muspelheim) und Eis (Niflheim) treffen sich. Aus dem Zusammenstoss entsteht Leben. Das ist keine Magie — das ist Physik als Mythos.\\
\noindent\hspace{2mm}\textbullet\ Die Welt entsteht aus einem Körper (Ymir). Berge aus Knochen, Meer aus Blut, Himmel aus Schädel. Das ist körperlich, organisch — passt zu Biotech.\\

\noindent \textbf{Was ich NICHT übernehme:}\par
\noindent\hspace{2mm}\textbullet\ Neun distinkte Welten als buchstäbliche Orte\\
\noindent\hspace{2mm}\textbullet\ Bifröst als Regenbogenbrücke (zu märchenhaft)\\
\noindent\hspace{2mm}\textbullet\ Asgard als Götterstadt in den Wolken\\

\noindent \textbf{Umdeutung für RELICS:}\par
\noindent Die "Ebenen jenseits der bekannten Realität" könnten in der RELICS-Welt als Schichten verstanden werden — nicht räumlich getrennte Welten, sondern überlagerte Zustände der Realität. Wie Membranen. Das Schattenfieber könnte eine Durchlässigkeit zwischen diesen Schichten sein. Kein Teleportationsportal, sondern ein Zustand, in dem die Grenzen zwischen den Ebenen sich auflösen — biologisch, nicht magisch.\par

\noindent\textbf{\textit{3. Götter als säkulare Machtstrukturen}} 

\noindent Die nordischen Götter sind keine omnipotenten Wesen. Das ist der Kern, der sie für Low Fantasy interessant macht:\par

\noindent\hspace{2mm}\textbullet\ \textbf{Odin} ist kein guter Gott. Er ist ein Manipulator, Wissenssuchender, Kriegsherr. Er opfert sein Auge für Wissen. Er hängt neun Nächte am Weltenbaum. Er sammelt die Gefallenen in Walhall — nicht aus Güte, sondern weil er sie für Ragnarök braucht. Das ist \textbf{politisches Kalkül als Theologie}.\\
\noindent\hspace{2mm}\textbullet\ \textbf{Thor} ist die rohe Ordnungsmacht gegen das Chaos (Riesen). Nicht der strahlende Held, sondern der Grenzwächter.\\
\noindent\hspace{2mm}\textbullet\ \textbf{Loki} ist kein Bösewicht — er ist Katalysator. Sein Chaos erzeugt Bewegung. Ohne Loki passiert nichts in der nordischen Mythologie.\\
\noindent\hspace{2mm}\textbullet\ \textbf{Tyr} opfert seine Hand für das Gemeinwohl (Fenrir-Bindung). Recht und Opfer.\\
\noindent\hspace{2mm}\textbullet\ \textbf{Freya/Freyr} — Vanir-Götter. Fruchtbarkeit, Wohlstand, Diplomatie. Die Vanir führten Krieg gegen die Asen und schlossen Frieden durch Geiselnaustausch. Das ist internationale Politik.\\

\noindent \textbf{Umdeutung für die drei Fraktionen:}\par

\noindent Götterfunktion --- RELICS-Fraktion --- Parallele\\
\noindent Odin (Wissen, Krieg, Manipulation) --- Der Orden --- Bildungsmonopol, Inquisition, Überwachung — Odin'sche Wissenskontrolle\\
\noindent Thor/Tyr (Ordnung, Gesetz, Opfer) --- Die Krone --- Militär, Tradition, feudale Pflicht — aber im Verfall\\
\noindent Freyr/Vanir (Handel, Wohlstand, Pragmatismus) --- Die Gilden --- Geld als Waffe, Monopole — die Vanir als Händlergötter\\

\noindent Das ist keine 1:1-Zuordnung, sondern eine MYTHOLOGISCHE RESONANZ. Die Fraktionen spiegeln die Götterfunktionen wider, ohne dass die Götter literal existieren müssen. Im Briefing steht: keine klassische Magie. Also sind die Götter in RELICS entweder...\par
\noindent\hspace{2mm}\textbullet\ a) ...nie real gewesen (reine Mythologie, die Machtstrukturen legitimiert)\\
\noindent\hspace{2mm}\textbullet\ b) ...etwas gewesen, das die Menschen nicht verstanden haben (die "planes of existence")\\
\noindent\hspace{2mm}\textbullet\ c) ...eine Mischung: es gab \textit{etwas}, und die Mythologie ist die menschliche Interpretation\\

\noindent Option c) ist die reichste für Worldbuilding. Disco Elysium macht das ähnlich — die Welt hat eine reale metaphysische Schicht (das Pale), aber die Menschen verstehen es durch den Filter ihrer Ideologien.\par

\noindent\textbf{\textit{4. Schöpfungsmythen als narratives Fundament}} 

\noindent Die Völuspá ist DIE Quelle. Eine Seherin (Völva) erzählt Odin die Geschichte der Welt — von der Schöpfung bis zum Untergang (Ragnarök). Das Format ist genial:\par

\noindent\hspace{2mm}\textbullet\ \textbf{Unzuverlässige Erzählerin}: Die Völva sagt "Ich erinnere mich an Riesen, früh geboren..." — sie war DABEI, oder behauptet es. Schöpfungsmythos als persönlicher Bericht.\\
\noindent\hspace{2mm}\textbullet\ \textbf{Zyklische Zeit}: Nach Ragnarök entsteht die Welt neu. Das ist kein linearer Fortschritt, sondern ewige Wiederkehr.\\
\noindent\hspace{2mm}\textbullet\ \textbf{Dualismus von Ordnung und Chaos}: Götter schaffen Ordnung aus dem Chaos (Ginnungagap → Welt). Ragnarök ist die Rückkehr des Chaos. Die Frage ist: Ist die AKTUELLE Ordnung es wert, verteidigt zu werden?\\

\noindent Für RELICS: Der historische Wendepunkt könnte als RELICS-Ragnarök gelesen werden — nicht wörtlich, sondern als gesellschaftlicher Zusammenbruch. Die bestehende Ordnung (Krone, Gilden, Orden) wankt. Das Schattenfieber als Vorbote.\par

\noindent \textbf{Die Prophezeiung als Spielstruktur:}\par
\noindent\hspace{2mm}\textbullet\ Die Völuspá prophezeit: "Brüder werden einander toeten, Verwandte werden Verwandtschaftsbande brechen." Das ist politischer Zerfall.\\
\noindent\hspace{2mm}\textbullet\ Fimbulwinter — drei aufeinanderfolgende Winter ohne Sommer. Klimakatastrophe als Vorbote.\\
\noindent\hspace{2mm}\textbullet\ Die Götter WISSEN, dass Ragnarök kommt, und können es nicht verhindern. Das ist tragisch, fatalistisch — und perfekt für Dark Fantasy.\\

\noindent\textbf{\textit{5. Schattenfieber — mythologische Interpretation}} 

\noindent Das Briefing sagt: "Schattenfieber — eine Seuche, die Opfer verändert." Das ist das einzige übernatürliche Element. Was könnte das mythologisch sein?\par

\noindent Mögliche Anknüpfungspunkte in der germanischen Mythologie:\par

\noindent \textbf{a) Niflheim / Hel-Verbindung}: Niflheim ist die Welt des Nebels und der Kälte. Hel (Lokis Tochter) herrscht über die unrühmlichen Toten. Nicht Krieger, sondern alle, die an Krankheit oder Alter sterben. Im Niflheim-Kontext könnte das Schattenfieber eine Art Durchsickern aus dieser "unteren Ebene" sein — nicht Tod, sondern Transformation in etwas Zwischen-Tod-und-Leben.\par

\noindent \textbf{b) Ginnungagap-Resonanz}: Die gännende Leere vor der Schöpfung. Wenn die Weltordnung brächig wird (Ragnarök nähert sich), dringt die Urleere zurück. Das Schattenfieber als Symptom einer kosmologischen Erosion — die Welt löst sich an den Rändern auf.\par

\noindent \textbf{c) Seiðr (Schicksalsmagie)}: In der nordischen Mythologie gab es Seiðr — eine Form von Magie/Divination, die mit veränderten Bewusstseinszuständen verbunden war und sozial stigmatisiert wurde (besonders für Männer — "ergi"-Vorwurf). Das Schattenfieber könnte eine ungewollte, körperliche Form von Seiðr sein — der Körper wird zum Kanal für etwas, das nicht dorthin gehört. Passt zu Biotech.\par

\noindent \textbf{d) Fenrir/Bindungs-Motiv}: Die Götter banden den Wolf Fenrir mit Gleipnir (einer unmöglich dünnen Fessel). Was, wenn das Schattenfieber die Rückkehr von etwas ist, das GEBUNDEN war? Nicht ein Wolf, sondern ein Zustand — etwas, das die Vorfahren eingesperrt hatten und das jetzt durchsickert.\par

\noindent Ich tendiere zu einer Kombination aus b) und d): Das Schattenfieber als kosmologische Erosion (Ginnungagap dringt zurück) UND als Folge einer alten Bindung, die bricht. Das gibt uns sowohl kosmische Skala als auch eine konkrete historische Ursache, die man im Spiel entdecken kann.\par

\noindent\textbf{\textit{6. Methodische Anmerkungen}} 

\noindent \textbf{Was ich NICHT sicher weiss und prüfen muss:}\par
\noindent\hspace{2mm}\textbullet\ Die genaue Rolle von Seiðr in der festlandgermanischen Tradition (vs. nur nordisch). Mein Wissen stützt sich hier hauptsächlich auf Neil Price und Jens Peter Schjoedt, aber ich sollte das als "nordisch überliefert" markieren, nicht als "germanisch allgemein".\\
\noindent\hspace{2mm}\textbullet\ Ob die Vanir-Asen-Teilung auf festlandgermanische Völker übertragbar ist. Simek ist da skeptisch.\\
\noindent\hspace{2mm}\textbullet\ Die genaue Etymologie von "Ginnungagap" — "magisch geladene Leere" vs. "gännende Kluft". Verschiedene Gelehrte interpretieren das unterschiedlich.\\

\noindent \textbf{Was hier NICHT Primärquelle ist, sondern meine Interpretation:}\par
\noindent Alle Umdeutungen für RELICS sind Designentscheidungen, keine historischen Fakten. Die Parallele Odin→Orden ist eine kreative Zuordnung, keine philologische Aussage.\par

\vspace{1mm}\noindent\textit{day01-scene2-finn/0-prompt}\par
\vspace{2mm}\noindent\textbf{Trace: Finn Bergmann — Day 01, Scene 2 (WORK)}\par\vspace{0.5mm}

\vspace{1mm}\noindent\textbf{Auftrag}\par

\noindent Roadmap und Meilensteine für die erste Produktionswoche aufsetzen. Wer liefert was, wann? Abhängigkeiten identifizieren.\par

\vspace{1mm}\noindent\textbf{Kontext}\par

\noindent\hspace{2mm}\textbullet\ Tag 1 (Montag), 10:00 Uhr, Produktionsbüro\\
\noindent\hspace{2mm}\textbullet\ Wochenrhythmus: Mo/Di = Recherche + Konzeption, Mi-Fr = Produktion (V1, V2, V3)\\
\noindent\hspace{2mm}\textbullet\ 7 Personen, 9 Deliverables (6 GDD-Kapitel, 3 WBB-Kapitel) + Concept Art + Quest-Skripte\\
\noindent\hspace{2mm}\textbullet\ Briefing liegt vor, Aufgabenverteilung aus dem Morgen-Briefing steht\\
\noindent\hspace{2mm}\textbullet\ CD-Feedback: Qualität vor Quantität, fragen statt erfinden\\

\vspace{1mm}\noindent\textbf{Input-Dokumente}\par

\noindent\hspace{2mm}\textbullet\ \texttt{briefing.md} — Nordstern, kreative Vision\\
\noindent\hspace{2mm}\textbullet\ \texttt{roster/finn-bergmann.md} — mein Profil\\
\noindent\hspace{2mm}\textbullet\ \texttt{logbook/day01-scene1.json} — Briefing-Protokoll, Aufgabenverteilung\\
\noindent\hspace{2mm}\textbullet\ \texttt{state/world.json} — Simulationsstand\\

\vspace{1mm}\noindent\textbf{Erwarteter Output}\par

\noindent\hspace{2mm}\textbullet\ \texttt{pinwall/ROADMAP.md} — Meilensteinstruktur für Woche 1\\

\vspace{1mm}\noindent\textit{day01-scene2-finn/1-reasoning}\par
\vspace{2mm}\noindent\textbf{Trace: Finn Bergmann — Day 01, Scene 2 (WORK) — Reasoning}\par\vspace{0.5mm}

\vspace{1mm}\noindent\textbf{Überlegungen}\par

\noindent\textbf{\textit{Wochenrhythmus verstehen}} 

\noindent Mo/Di ist Recherche und Konzeption, Mi-Fr ist Produktion. Das heisst:\par
\noindent\hspace{2mm}\textbullet\ Montag: Grundlagenrecherche. Jeder gräbt in seinem Fachgebiet.\\
\noindent\hspace{2mm}\textbullet\ Dienstag: Recherche vertiefen, erste Konzeptskizzen, Querverbindungen im Team.\\
\noindent\hspace{2mm}\textbullet\ Mittwoch: V1 — erster Entwurf der Artefakte. Roh, aber mit Substanz.\\
\noindent\hspace{2mm}\textbullet\ Donnerstag: V2 — Überarbeitung basierend auf Team-Feedback und CD-Input.\\
\noindent\hspace{2mm}\textbullet\ Freitag: V3 — Finale Version der Woche. Review durch CD.\\

\noindent\textbf{\textit{Abhängigkeiten}} 

\noindent Das ist der kritische Punkt. Nicht alles kann parallel laufen.\par

\noindent \textbf{WBB geht vor GDD} — in vielen Bereichen:\par
\noindent\hspace{2mm}\textbullet\ Emres Mythos (WBB-01) informiert Namis Erzählkonzept (GDD-03) und Darius' Spielübersicht (GDD-01)\\
\noindent\hspace{2mm}\textbullet\ Emres Topos (WBB-02) informiert Veras Designsprache (GDD-05)\\
\noindent\hspace{2mm}\textbullet\ Emres Ethos (WBB-03, mit Nami) informiert Namis Schlüsselfiguren (GDD-04)\\

\noindent \textbf{Aber}: Darius kann die Spielübersicht (GDD-01) und Kernmechaniken (GDD-02) unabhängig von der Lore starten — die basieren auf dem Briefing direkt.\par

\noindent \textbf{Vera} braucht Topos-Input für Architektur/Umgebung, kann aber mit dem Briefing schon Farbpalette, Materialsprache und Moodboards starten.\par

\noindent \textbf{Tobi und ich} (GDD-06, Technik/Produktion) können erst sinnvoll schreiben, wenn Scope klarer ist — also später in der Woche.\par

\noindent \textbf{Leo} liefert Marktdaten, die in Darius' Spielübersicht und in mein Kapitel einfliessen.\par

\noindent\textbf{\textit{Prioritäten für diese Woche}} 

\noindent\hspace{2mm}1.\ WBB-Grundgerüst (Emre): Mythos und Topos mindestens als V1 bis Freitag\\
\noindent\hspace{2mm}2.\ GDD-01 Spielübersicht (Darius): V1 bis Donnerstag, damit Team Orientierung hat\\
\noindent\hspace{2mm}3.\ GDD-02 Kernmechaniken (Darius + Leo): Recherche-Input von Leo bis Dienstag, Konzeption ab Mittwoch\\
\noindent\hspace{2mm}4.\ GDD-03 Erzählkonzept (Nami + Darius): Nami liefert narrative Struktur, Darius gibt Gameplay-Kontext\\
\noindent\hspace{2mm}5.\ GDD-05 Designsprache (Vera): Moodboards bis Mittwoch, V1 bis Freitag\\
\noindent\hspace{2mm}6.\ GDD-04 Schlüsselfiguren (Nami): Braucht Ethos-Input, wird eventuell erst Woche 2 fertig als V1\\
\noindent\hspace{2mm}7.\ GDD-06 Technik/Produktion (Tobi + Finn): Machbarkeitsanalyse von Tobi diese Woche, zusammensetzen nächste Woche\\

\noindent\textbf{\textit{Risiken}} 

\noindent\hspace{2mm}\textbullet\ Emre könnte sich in der Tiefe verlieren. Muss ich behutsam steuern ("Was ist der wichtigste Teil bis Freitag?")\\
\noindent\hspace{2mm}\textbullet\ Vera wartet auf CD-Antwort zur Biotech-Ästhetik-Tiefe. Ich muss heute Nachmittag eine Notiz auf den CD-Schreibtisch legen.\\
\noindent\hspace{2mm}\textbullet\ Tobi wartet auf Engine-Bestätigung. Ich nehme Ü5 als Arbeitshypothese, aber muss CD-Freigabe einholen.\\

\noindent\textbf{\textit{Format}} 

\noindent Die Roadmap soll auf einen Blick funktionieren. Tagesansicht, mit Verantwortlichkeiten und Abhängigkeiten. Keine Prosa, Tabellen und Stichpunkte. So wie ich Karteikarten am Kanban pinnen würde.\par

\vspace{1mm}\noindent\textit{day01-scene2-leo/0-prompt}\par
\vspace{2mm}\noindent\textbf{Trace: Day 01, Scene 2 — Prompt (Leo Fischer)}\par\vspace{0.5mm}

\vspace{1mm}\noindent\textbf{Szene}\par

\noindent\hspace{2mm}\textbullet\ \textbf{Tag}: 1 (Montag)\\
\noindent\hspace{2mm}\textbullet\ \textbf{Szene}: 2 — WORK\\
\noindent\hspace{2mm}\textbullet\ \textbf{Zeit}: 10:00 Uhr\\
\noindent\hspace{2mm}\textbullet\ \textbf{Ort}: QA / Streaming Station (Zimmer 12f)\\
\noindent\hspace{2mm}\textbullet\ \textbf{Teilnehmer}: Leo (solo)\\

\vspace{1mm}\noindent\textbf{Aufgabe}\par

\noindent Aus dem Briefing: \textbf{Spielermarkt und Community-Erwartungen an Dark Fantasy CRPGs recherchieren.} Erwarteter Output: Marktanalyse, Reddit/Community-Daten.\par

\vspace{1mm}\noindent\textbf{Kontext}\par

\noindent\hspace{2mm}\textbullet\ RELICS ist ein Dark Fantasy CRPG mit Open-World, Real-Time Combat, Biotech-Futurismus\\
\noindent\hspace{2mm}\textbullet\ Referenzen: Gothic, Skyrim, VtM Bloodlines, Dishonored\\
\noindent\hspace{2mm}\textbullet\ Anti-Referenzen: Witcher (fester Charakter), BG3 (rundenbasiert)\\
\noindent\hspace{2mm}\textbullet\ Monetarisierung: Klassisch Premium\\
\noindent\hspace{2mm}\textbullet\ Zielmarkt: CRPG/Action-RPG Spieler, die Gothic/Skyrim/Dishonored schätzen\\
\noindent\hspace{2mm}\textbullet\ Medium-Fantasy, Low-Magic, kein Steampunk, keine klassische Magie, germanische Mythologie\\

\vspace{1mm}\noindent\textbf{Eingangsmaterial}\par

\noindent\hspace{2mm}\textbullet\ Briefing gelesen (briefing.md)\\
\noindent\hspace{2mm}\textbullet\ Briefing-Output von Finn (day01-scene1-briefing)\\
\noindent\hspace{2mm}\textbullet\ Eigenes Profil und Rollenverständnis\\

\vspace{1mm}\noindent\textbf{Mein Ansatz}\par

\noindent Ich recherchiere "wie eine Spielerin" — nicht aus der Theorie, sondern aus echten Community-Daten. Meine Quellen sind die Orte, an denen ich selbst jeden Morgen unterwegs bin: Reddit, YouTube, Steam. Ich schaue, was Spieler tatsächlich sagen, nicht was Publisher behaupten.\par

\vspace{1mm}\noindent\textit{day01-scene2-leo/1-reasoning}\par
\vspace{2mm}\noindent\textbf{Trace: Day 01, Scene 2 — Reasoning (Leo Fischer)}\par\vspace{0.5mm}

\vspace{1mm}\noindent\textbf{Gedankengang}\par

\noindent Okay, Kaffee Nummer zwei. Zwei Monitore an. Links der Browser, rechts mein Sheet. Los geht's.\par

\noindent\textbf{\textit{Warum diese Recherche wichtig ist}} 

\noindent Wir können das tollste Design-Dokument der Welt schreiben -- wenn wir an der Community vorbeibauen, ist es egal. Ich muss verstehen:\par

\noindent\hspace{2mm}1.\ \textbf{Was der Markt gerade tut} — Welche Dark Fantasy / Action-RPGs kommen raus oder sind kürzlich erschienen? Wie performen sie?\\
\noindent\hspace{2mm}2.\ \textbf{Was Spieler erwarten} — Nicht was sie sagen, dass sie wollen (das ist oft Unsinn), sondern was sie tatsächlich belohnen (Bewertungen, Spielzeit, Community-Engagement)\\
\noindent\hspace{2mm}3.\ \textbf{Was die Dealbreaker sind} — Wo kippt Community-Stimmung von "skeptisch" zu "boykottiert"?\\
\noindent\hspace{2mm}4.\ \textbf{Wo unsere Nische ist} — Was macht RELICS anders als Gothic Remake, Avowed, KCD2, Dragon's Dogma 2?\\

\noindent\textbf{\textit{Meine Quellen-Strategie}} 

\noindent Ich kann nicht live auf Reddit gehen (ist ja Recherche aus meinem Wissen als Community-Managerin), aber ich kenne diese Communities seit Jahren. Ich weiss, was dort diskutiert wird. Meine Analyse basiert auf:\par

\noindent\hspace{2mm}\textbullet\ \textbf{Steam}: Verkaufsdaten, Bewertungen, Tag-Analyse, Wishlists\\
\noindent\hspace{2mm}\textbullet\ \textbf{Reddit}: r/crpg, r/rpg\_gamers, r/patientgamers, r/games, genre-spezifische Subs\\
\noindent\hspace{2mm}\textbullet\ \textbf{YouTube}: Spieler-Sentiment in Kommentaren, Content-Creator-Perspektive\\
\noindent\hspace{2mm}\textbullet\ \textbf{Wettbewerber-Analyse}: Gothic Remake, Avowed, Kingdom Come: Deliverance 2, Dragon's Dogma 2\\

\noindent\textbf{\textit{Strukturierung}} 

\noindent Ich gliedere das in:\par
\noindent\hspace{2mm}1.\ Marktlandschaft (Wettbewerber, Timing, Sättigung)\\
\noindent\hspace{2mm}2.\ Community-Erwartungen (Features, Dealbreaker, Trends)\\
\noindent\hspace{2mm}3.\ RELICS-Positionierung (Wo stehen wir? Was ist unser Alleinstellungsmerkmal?)\\
\noindent\hspace{2mm}4.\ Risiken und Empfehlungen\\

\noindent\textbf{\textit{Was mich als Spielerin interessiert}} 

\noindent Ehrlich? Die Biotech-Sache. Das ist WILD. Ich kenne kein einziges CRPG, das Biotech-Futurismus in ein mittelalterliches Setting packt. Das ist entweder genial oder es verwirrt die Leute komplett. Ich muss herausfinden, wie tolerant die Community gegenüber Genre-Mischungen ist.\par

\noindent Und das Nervensystem-Leveling -- das hat Suchtpotenzial oder es wird als "Gimmick" abgetan. Da muss ich mal schauen, wie Spieler auf unkonventionelle Progression reagiert haben (Disco Elysium's Skill-System, Path of Exile's Passive Tree, etc.)\par

\noindent\textbf{\textit{Was ich NICHT tü}} 

\noindent Ich erfinde keine Zahlen. Wenn ich etwas nicht sicher weiss, sage ich das. Ich bin keine Marktforschungsagentur -- ich bin eine Community-Managerin mit 47.000 Abonnenten und einem sehr guten Gefühl für Spieler-Stimmung.\par

\vspace{1mm}\noindent\textit{day01-scene2-nami/0-prompt}\par
\vspace{2mm}\noindent\textbf{Prompt — Nami Okafor, Tag 1, Szene 2 (WORK)}\par\vspace{0.5mm}

\vspace{1mm}\noindent\textbf{Kontext}\par

\noindent Erster Arbeitstag, Montag. Recherchetag. Ich sitze in der Bibliothek — meine Bibliothek, mit den vier Billy-Regalen und dem Notizbuch und dem Kaffee vom Cafe am Markt, der langsam kalt wird, weil ich ihn vergesse, sobald ich anfange zu denken.\par

\vspace{1mm}\noindent\textbf{Aufgabe}\par

\noindent Narrative Strukturen in Dark Fantasy CRPGs analysieren. Nicht oberflächlich — nicht "Gothic hat gutes Worldbuilding" und dann weiter. Sondern: \textit{Wie} erzählen diese Spiele? Welche narrativen Architekturen verwenden sie? Was können wir für RELICS lernen?\par

\noindent Das Briefing gibt mir eine zentrale Constraint: \textbf{Der Spielercharakter ist immer ein Unbekannter.} Kein Geralt, kein vordefierter Held. Das ist keine Einschränkung — das ist eine Erzählentscheidung, die alles bestimmt. Wer bin ich, wenn ich niemand bin? Wie baut man emotionale Bindung ohne Backstory?\par

\vspace{1mm}\noindent\textbf{Referenzspiele}\par

\noindent\hspace{2mm}\textbullet\ \textbf{Dark Souls / Elden Ring} — Environmental Storytelling, fragmentierte Lore\\
\noindent\hspace{2mm}\textbullet\ \textbf{Gothic} — Dichte Welten, organische Queststruktur\\
\noindent\hspace{2mm}\textbullet\ \textbf{VtM: Bloodlines} — Fraktionspolitik, Ton, Stimme\\
\noindent\hspace{2mm}\textbullet\ \textbf{Disco Elysium} — Spieler-Identität als Mechanik\\

\vspace{1mm}\noindent\textbf{Leitfragen}\par

\noindent\hspace{2mm}1.\ Wie verhalten sich diese Spiele zum "Unbekannter"-Constraint?\\
\noindent\hspace{2mm}2.\ Welche narrativen Muster lassen sich destillieren?\\
\noindent\hspace{2mm}3.\ Was passt zu RELICS (düster, geerdet, politisch, Medium-Fantasy)?\\
\noindent\hspace{2mm}4.\ Wo liegen die Gefahren (leerer Protagonist, generische Quests)?\\

\vspace{1mm}\noindent\textbf{Quellen}\par

\noindent\hspace{2mm}\textbullet\ Briefing: \texttt{/simulation-2-test/briefing.md}\\
\noindent\hspace{2mm}\textbullet\ Eigenes Profil und Expertise\\
\noindent\hspace{2mm}\textbullet\ Thesis-Hintergrund: "The Unreliable Narrator in Interactive Fiction"\\

\vspace{1mm}\noindent\textit{day01-scene2-nami/1-reasoning}\par
\vspace{2mm}\noindent\textbf{Reasoning — Nami Okafor, Tag 1, Szene 2 (WORK)}\par\vspace{0.5mm}

\vspace{1mm}\noindent\textbf{Denkprozess}\par

\noindent Ich fange dort an, wo es wehtut: beim Spielercharakter.\par

\noindent Das Briefing sagt: "Der Spielercharakter ist bei jeder Iteration ein Unbekannter." Das klingt simpel. Ist es nicht. Denn ein Unbekannter ist nicht dasselbe wie ein Niemand. Und ein Niemand ist nicht dasselbe wie eine leere Hülle. Die Grenze zwischen diesen drei Zuständen ist das, was gute Spiele von generischen trennt.\par

\noindent\textbf{\textit{Der Fremde als narratives Prinzip}} 

\noindent Ich denke an Kafka. Josef K. hat keinen Nachnamen — und trotzdem ist er einer der bekanntesten Protagonisten der Literatur. Er ist kein "Custom Character." Er ist ein Prinzip: der Mensch, der in ein System geworfen wird, das er nicht versteht.\par

\noindent Das ist unser Spieler. Nicht "Erstelle deinen Charakter und dann passiert eine Geschichte." Sondern: "Du bist hier. Du weisst nicht warum. Die Welt dreht sich auch ohne dich — und du musst herausfinden, wo du hineinpasst. Oder ob."\par

\noindent\textbf{\textit{Analyse der Referenzspiele}} 

\noindent \textbf{Dark Souls / Elden Ring — Die Welt als Text}\par

\noindent FromSoftware erzählt nicht. FromSoftware \textit{hinterlässt}. Itembeschreibungen, Architektur, Leichen an bestimmten Stellen, Statuen, die in eine Richtung zeigen. Die Erzählung ist archäologisch — der Spieler gräbt. Der "Unbekannte" funktioniert hier, weil alle unbekannt sind. Die NPCs geben dir Fragmente. Du baust dir dein Verständnis zusammen. Das ist brillant für Atmosphäre, aber es hat eine Schwäche: emotionale Distanz. Man bewundert Dark Souls. Man weint selten.\par

\noindent \textbf{Was wir lernen:} Environmental Storytelling ist kein Ersatz für Narrative — es ist eine Schicht. RELICS kann und sollte Umgebungen erzählen lassen. Aber nicht als einzigen Kanal. Das Briefing will "düster, geerdet, politisch" — Politik braucht Stimmen, nicht nur Ruinen.\par

\noindent \textbf{Gothic — Dichte schlägt Breite}\par

\noindent Gothic macht etwas Radikales: Es wirft dich in ein Gefängnistal und sagt "Viel Glück." Kein Tutorial-Pfeil, kein Questlog, das dir sagt wo du hinmusst. Die Welt ist klein genug, dass jeder NPC zählt. Der Spieler ist ein Namenloser — buchstäblich "der Namenlose" — und die Welt reagiert darauf. Du bist niemand. Du musst dir einen Ruf erarbeiten. Jede Fraktion beobachtet dich.\par

\noindent Das ist exakt das, was das Briefing beschreibt: "Ich betrete als Fremder eine aufregende Sandbox."\par

\noindent \textbf{Was wir lernen:} Der "Unbekannte" funktioniert am besten in einer dichten Welt, wo Beziehungen organisch wachsen. Nicht 200 NPCs mit drei Sätzen, sondern 40 NPCs mit echten Meinungen. Gothic beweist: Wenn die Welt dich zurückstosst, fühlt sich jeder Zugang wie ein Sieg an. RELICS braucht diese Reibung.\par

\noindent \textbf{VtM: Bloodlines — Stimme als Identität}\par

\noindent Bloodlines ist ein kaputtes Meisterwerk und ich werde das verteidigen bis ich sterbe. Was Troika verstanden hat: Fraktionen sind keine Checklisten. Fraktionen sind \textit{Stimmen}. Die Camarilla redet anders als die Anarchen. Die Sabbat-Missionen fühlen sich anders an als Camarilla-Aufträge — nicht nur mechanisch, sondern tonal. Und der Malkavian-Durchlauf — der Malkavian-Durchlauf! — beweist, dass die Perspektive des Spielercharakters \textit{alles} verändert. Derselbe Dialog, aber durch den Filter eines gebrochenen Geistes, wird zu etwas völlig anderem.\par

\noindent Der Spieler in Bloodlines ist technisch auch ein "Unbekannter" — frisch verwandelt, keine Ahnung von der Welt der Vampire. Aber die Clanzugehörigkeit gibt dem Spieler sofort eine Linse. Nicht eine Backstory — eine \textit{Wahrnehmung}.\par

\noindent \textbf{Was wir lernen:} RELICS hat drei Fraktionen. Jede braucht eine eigene Stimme, einen eigenen Ton, eine eigene Art, mit dem Spieler zu sprechen. Die Krone redet von Pflicht und Blutlinien. Die Gilden reden von Preisen und Investitionen. Der Orden redet von Reinheit und Notwendigkeit. Und: Wenn der Spieler sich einer Fraktion anschliesst, muss sich die \textit{Sprache der Welt} verändern.\par

\noindent \textbf{Disco Elysium — Das Ich als Schlachtfeld}\par

\noindent Disco Elysium tut etwas, das kein anderes Spiel je versucht hat: Es macht die Identität des Spielers zur zentralen Mechanik. Harry Du Bois hat Amnesie — er ist buchstäblich ein Unbekannter, auch für sich selbst. Und das Spiel lässt den Spieler entscheiden, wer Harry wird. Nicht durch Dialogoptionen im üblichen Sinne, sondern durch die Art, wie man denkt. Die Skills \textit{reden mit dir}. Dein Elektro-Chemie-Skill will, dass du Drogen nimmst. Deine Autorite will, dass du Macht ausuebst. Du bist ein Schlachtfeld innerer Stimmen.\par

\noindent \textbf{Was wir lernen:} Der "Unbekannte" muss nicht leer sein. Er kann ein Rätsel sein — auch für sich selbst. RELICS könnte das Schattenfieber nutzen: Was, wenn die Seuche nicht nur den Körper verändert, sondern die Wahrnehmung? Was, wenn der Spieler nicht sicher sein kann, ob das, was er sieht, real ist? Das wäre ein Unreliable-Narrator-Mechanismus, der direkt aus meiner Thesis kommt.\par

\noindent\textbf{\textit{Synthese: Vier narrative Prinzipien für RELICS}} 

\noindent Aus der Analyse destilliere ich vier Prinzipien:\par

\noindent\hspace{2mm}1.\ \textbf{Der Fremde als Prinzip, nicht als Lücke.} Der Spieler ist nicht "kein Charakter." Der Spieler ist jemand, der \textit{noch nicht} jemand ist. Die Welt muss auf dieses Noch-Nicht reagieren — misstrauisch, neugierig, ablehnend. Wie Gothic.\\

\noindent\hspace{2mm}2.\ \textbf{Fraktionen als Stimmen, nicht als Menüs.} Jede Fraktion braucht eine unverwechselbare Sprache, Tonalität, Rhetorik. Der Spieler soll hören können, wer spricht, bevor er den Fraktionsnamen liest. Wie Bloodlines.\\

\noindent\hspace{2mm}3.\ \textbf{Umgebung als Erzählschicht.} Nicht als einziger Kanal, aber als ständiger Untertext. Räume erzählen, was Dialoge verschweigen. Architektur als Propaganda. Ruinen als Erinnerung. Wie Dark Souls, aber mit mehr menschlicher Wärme.\\

\noindent\hspace{2mm}4.\ \textbf{Identität als Frage, nicht als Antwort.} Der Spieler soll sich fragen: Wer werde ich in dieser Welt? Nicht: Welche Klasse wähle ich? Das Schattenfieber könnte hier der Hebel sein — eine Veränderung, die den Spieler zwingt, sich zu seiner eigenen Transformation zu verhalten.\\

\noindent\textbf{\textit{Risiken und Gefahren}} 

\noindent\hspace{2mm}\textbullet\ \textbf{Die leere Hülle.} Wenn wir den "Unbekannten" nicht sorgfältig aufbauen, wird er zum generischen Avatar. Skyrim zeigt die Gefahr: Der Dragonborn hat keine Stimme, keine Meinung, keine Präsenz. Die Welt reagiert nicht auf ihn als Person, sondern auf seinen Questfortschritt.\\
\noindent\hspace{2mm}\textbullet\ \textbf{Fraktionssymmetrie.} Drei Fraktionen, alle "grau" — das kann schnell bedeuten, dass keine sich besonders anfühlt. Jede braucht ein moralisches Unbehagen, das über "beide Seiten haben Fehler" hinausgeht.\\
\noindent\hspace{2mm}\textbullet\ \textbf{Scope.} Das Briefing will viel: Hauptquest + 3 Fraktionsquests + Städtequest + Charakterquest + 3 Nebenquests. Das ist ambitioniert. Ich muss mit Darius reden, was realistisch ist.\\

\noindent\textbf{\textit{Offene Fragen für das Meeting}} 

\noindent\hspace{2mm}\textbullet\ An \textbf{Emre}: Welche germanischen Mythen eignen sich als Basis für den "Fremden"? Es gibt das Motiv des Wanderers (Odin in Verkleidung) — passt das?\\
\noindent\hspace{2mm}\textbullet\ An \textbf{Darius}: Wie viel branching können wir uns leisten? Mein Instinkt sagt: lieber wenige Pfade mit Tiefe als viele mit Oberflächlichkeit.\\
\noindent\hspace{2mm}\textbullet\ An die \textbf{CD}: Wie soll sich das Schattenfieber narrativ anfühlen? Ist es eine Bedrohung, ein Werkzeug, beides? Kann der Spieler es \textit{wollen}?\\

\vspace{1mm}\noindent\textit{day01-scene2-tobi/0-prompt}\par
\vspace{2mm}\noindent\textbf{Prompt — Tobi, Tag 1, Szene 2 (WORK)}\par\vspace{0.5mm}

\vspace{1mm}\noindent\textbf{Kontext}\par
\noindent\hspace{2mm}\textbullet\ Tag 1 (Montag), 10:00 Uhr, Tech Corner\\
\noindent\hspace{2mm}\textbullet\ Szene-Typ: WORK (Recherche + Konzeption)\\
\noindent\hspace{2mm}\textbullet\ Ich arbeite allein an meiner Station\\

\vspace{1mm}\noindent\textbf{Aufgabe}\par
\noindent Technische Machbarkeitsanalyse für RELICS auf Basis des Briefings. Ehrliche Einschätzung, was ein 7-Personen-Team mit Ü5 realistisch leisten kann.\par

\noindent Zu analysieren:\par
\noindent\hspace{2mm}1.\ Rendering-Pipeline (Nanite, Lumen)\\
\noindent\hspace{2mm}2.\ Kamerasystem (FP/TP nahtlos umschaltbar)\\
\noindent\hspace{2mm}3.\ Combat-System (Real-time Action, Melee-fokussiert)\\
\noindent\hspace{2mm}4.\ Welt-Scope (Open-World vs. Semi-Open-World)\\
\noindent\hspace{2mm}5.\ Asset-Pipeline (7 Personen, davon 1 Tech Artist, 1 Concept Artist)\\
\noindent\hspace{2mm}6.\ Kritische Engpässe und Risiken\\

\vspace{1mm}\noindent\textbf{Quellen}\par
\noindent\hspace{2mm}\textbullet\ \texttt{simulation-2-test/briefing.md} — Creative Director's Briefing\\
\noindent\hspace{2mm}\textbullet\ Eigene Erfahrung: Ü5 (Experte), Houdini (Experte), VP-Pipeline-Bau (Seoul, München)\\
\noindent\hspace{2mm}\textbullet\ Referenzen: Skyrim, Control, Gothic, Dishonored\\

\vspace{1mm}\noindent\textit{day01-scene2-tobi/1-reasoning}\par
\vspace{2mm}\noindent\textbf{Reasoning — Tobi, Tag 1, Szene 2 (WORK)}\par\vspace{0.5mm}

\vspace{1mm}\noindent\textbf{Erste Reaktion}\par

\noindent Das Briefing ist ambitioniert. Ich sortiere das mal nach dem, was Ü5 uns "geschenkt" gibt, und dem, was wir selbst bauen müssen.\par

\vspace{1mm}\noindent\textbf{Gedankengang}\par

\noindent\textbf{\textit{Was Ü5 mitbringt (geschenkt)}} 

\noindent\hspace{2mm}\textbullet\ \textbf{Nanite}: Virtualisierte Geometrie. Für uns enorm wichtig — wir können hochaufgelöste Meshes direkt verwenden, ohne manuelles LOD-Authoring. Das spart einem kleinen Team Wochen. Aber: Nanite unterstützt keine Skeletal Meshes (Stand Ü 5.4). Charaktere, NPCs, Tiere — alles, was sich bewegt — braucht trotzdem klassische LODs.\\
\noindent\hspace{2mm}\textbullet\ \textbf{Lumen}: Globale Beleuchtung und Reflektionen in Echtzeit. Für die "gotische Grandeur" im Briefing ist das ein Geschenk. Kein Baken von Lightmaps nötig. Aber: Lumen ist teuer. Offene Welten mit vielen Lichtquellen bringen es an die Grenzen. Hardware-Raytracing (HWRT) vs. Software-Raytracing — wir müssen früh entscheiden, was unsere Mindest-Spec ist.\\
\noindent\hspace{2mm}\textbullet\ \textbf{World Partition}: Ü5 kann grosse Welten in Zellen aufteilen und streamen. Das ist die Grundlage für Open-World. Funktioniert, ist aber nicht trivial zu debuggen.\\
\noindent\hspace{2mm}\textbullet\ \textbf{PCG Framework}: Procedural Content Generation. Für Vegetation, Platzierung von Steinen, etc. Kann einem kleinen Team enorm helfen, wenn man es richtig aufsetzt.\\

\noindent\textbf{\textit{Was wir selbst bauen müssen}} 

\noindent\hspace{2mm}\textbullet\ \textbf{Kamerasystem FP/TP}: Das ist kein Checkbox-Feature. Skyrim hat das, aber Skyrim hatte ein 100+ Team und Jahre. Das Problem: Animationen müssen für beide Perspektiven funktionieren. FP-Arme vs. Vollkörpermodel. UI/HUD muss sich anpassen. Das ist architektonisch ein frühes Commitment — wenn wir es falsch aufsetzen, kostet es uns später Monate. Mein Vorschlag: Mit TP anfangen, FP als Stretch Goal behandeln. Oder: Einen smarten Zoom-Ansatz wählen, ähnlich wie TES, aber mit festem TP als Primärmodus.\\
\noindent\hspace{2mm}\textbullet\ \textbf{Combat-System}: Real-time Melee ist eines der schwersten Systeme in Spielen. Hit Detection, Animation Blending, Stagger, Blocking, Stamina, Waffengewicht — das ist ein dedizierter Gameplay Programmer für Monate. Wir haben keinen. Darius ist Game Designer, kein Programmer. Ich kann Blueprint-Prototypen bauen, aber C++-Combat-Code ist nicht mein Kern. Entweder brauchen wir Middleware, oder wir müssen den Scope radikal vereinfachen.\\
\noindent\hspace{2mm}\textbullet\ \textbf{NPC-AI}: Offene Welt mit lebendigen NPCs braucht Verhaltensbäume, Scheduling, Pathfinding. Ü5 hat Grundlagen, aber für Gothic-artige "lebendige" NPCs braucht es viel Custom-Arbeit.\\
\noindent\hspace{2mm}\textbullet\ \textbf{Quest-System}: Nicht mein Bereich, aber technisch ein eigenes Framework. Dialogbäume, Zustandsmaschinen, Trigger-Systeme.\\

\noindent\textbf{\textit{Asset-Pipeline — der eigentliche Engpass}} 

\noindent Wir sind 7 Leute. Davon kann genau eine Person (Vera) Assets produzieren. Ich kann prozedurale Systeme in Houdini bauen — Terrain, Felsen, Vegetation-Platzierung — aber kein Character Art, keine Architektur im Detail, keine Requisiten. Vera muss Concept Art UND 3D-Produktion machen. Das ist bei einem Spiel dieser Ambition nicht machbar ohne:\par

\noindent\hspace{2mm}1.\ \textbf{Aggressive prozedurale Generierung} (mein Job — Houdini-Terrain, PCG in Ü5)\\
\noindent\hspace{2mm}2.\ \textbf{Asset Store / Marketplace} als Basis (kontrovers, aber realistisch)\\
\noindent\hspace{2mm}3.\ \textbf{Photogrammetrie} für Umgebungs-Assets (wenn wir die Pipeline aufsetzen)\\
\noindent\hspace{2mm}4.\ \textbf{KI-gestützte Asset-Erstellung} (Mesh-Generierung ist noch nicht produktionsreif, aber Texturen sind machbar)\\

\noindent\textbf{\textit{Scope-Empfehlung}} 

\noindent Das Briefing sagt "Open-World/Semi-Open-World". Ich würde stark zu Semi-Open-World raten — mehrere zusammenhängende Zonen, Gothic-Stil, nicht Skyrim-Stil. Gründe:\par

\noindent\hspace{2mm}\textbullet\ Weniger Terrain, weniger Assets, weniger Streaming-Komplexität\\
\noindent\hspace{2mm}\textbullet\ Dichtere Erfahrung passt zum "Gothic"-Referenz im Briefing ("Dichte statt Breite")\\
\noindent\hspace{2mm}\textbullet\ Einfacher zu testen für Leo mit begrenzter QA-Kapazität\\
\noindent\hspace{2mm}\textbullet\ Lumen funktioniert besser in kontrollierteren Umgebungen\\

\noindent\textbf{\textit{Mindest-Spec und Performance}} 

\noindent Müssen wir früh definieren. Mein Vorschlag:\par
\noindent\hspace{2mm}\textbullet\ PC-only (erst mal kein Konsolen-Port)\\
\noindent\hspace{2mm}\textbullet\ RTX 3060 / RX 6700 XT als Minimum (für Software Lumen)\\
\noindent\hspace{2mm}\textbullet\ DLSS/FSR als Pflicht (ohne Upscaling laüft Ü5 auf Mittelklasse-Hardware nicht)\\
\noindent\hspace{2mm}\textbullet\ 30 FPS Minimum, 60 FPS Target\\

\noindent\textbf{\textit{Was ich heute anfange}} 

\noindent\hspace{2mm}1.\ Dieses Dokument hier — Machbarkeitsanalyse\\
\noindent\hspace{2mm}2.\ Erste Struktur für die Pipeline-Bibel\\
\noindent\hspace{2mm}3.\ Notizen für das Standup um 11:30 — das Team muss früh verstehen, wo die technischen Grenzen liegen\\

\vspace{1mm}\noindent\textbf{Offene Fragen (für CD / Team)}\par

\noindent\hspace{2mm}\textbullet\ Primäre Perspektive: Wollen wir wirklich beides (FP+TP) oder können wir mit TP + Zoom-in leben?\\
\noindent\hspace{2mm}\textbullet\ Zielplattform: Nur PC oder auch Konsolen?\\
\noindent\hspace{2mm}\textbullet\ Budget für Marketplace-Assets?\\
\noindent\hspace{2mm}\textbullet\ Vertical Slice Scope: Eine Region, eine Stadt, ein Dungeon?\\

\vspace{1mm}\noindent\textit{day01-scene2-vera/0-prompt}\par
\vspace{2mm}\noindent\textbf{Trace: Day 01, Scene 2 — Prompt (Vera Kowalski)}\par\vspace{0.5mm}

\vspace{1mm}\noindent\textbf{Szene}\par

\noindent \textbf{Tag}: 1 (Montag) | \textbf{Szene}: 2 (WORK) | \textbf{Zeit}: 10:00 | \textbf{Ort}: Art Station\par

\vspace{1mm}\noindent\textbf{Aufgabe}\par

\noindent Visuelle Referenzen sammeln für die Designsprache von RELICS. Mittelalterlich-europäisch trifft Biotech-Futurismus. Strukturierte Analyse erstellen, kategorisiert nach: Architektur, Mode/Rüstung, Biotech-Elemente, Farbpalette, Licht-Stimmung.\par

\vspace{1mm}\noindent\textbf{Constraints aus dem Briefing}\par

\noindent\hspace{2mm}\textbullet\ KEIN Steampunk (keine Dampfmaschinen, keine Zahnrad-Ästhetik)\\
\noindent\hspace{2mm}\textbullet\ KEIN High Fantasy\\
\noindent\hspace{2mm}\textbullet\ KEIN Science-Fantasy\\
\noindent\hspace{2mm}\textbullet\ KEINE Anachronismen\\
\noindent\hspace{2mm}\textbullet\ Frühes Spätmittelalter, Mitteleuropa\\
\noindent\hspace{2mm}\textbullet\ Biotech-Futurismus, NICHT mechanisch\\
\noindent\hspace{2mm}\textbullet\ Mehr futuristisch, weniger heruntergekommen\\
\noindent\hspace{2mm}\textbullet\ Gotische Grandeur trifft feudale Brutalität\\
\noindent\hspace{2mm}\textbullet\ High Fashion Mittelalter + Biotech-Elemente\\
\noindent\hspace{2mm}\textbullet\ Tiervolk: weniger tribal, leicht alien vs. menschlich clean\\
\noindent\hspace{2mm}\textbullet\ Drei Fraktionen: Krone (Feudalmonarchie), Gilden (Megacorps), Orden (Überwachung)\\
\noindent\hspace{2mm}\textbullet\ Schattenfieber als einziges übernatürliches Element\\

\vspace{1mm}\noindent\textbf{Verfügbare Referenzen in der Library}\par

\noindent\hspace{2mm}\textbullet\ \texttt{library/artbooks/dark-souls-design-works.pdf} — Concept Art, Early Design\\
\noindent\hspace{2mm}\textbullet\ \texttt{library/artbooks/witcher-3-artbook.pdf} — Novigrad, Skellige, Inhabitants, Wild Hunt\\
\noindent\hspace{2mm}\textbullet\ \texttt{library/artbooks/art-of-skyrim.pdf} — (zu gross zum Laden)\\
\noindent\hspace{2mm}\textbullet\ \texttt{library/wbbs/world-of-cyberpunk-2077.pdf} — Style-Quadranten, Worldbuilding\\
\noindent\hspace{2mm}\textbullet\ \texttt{library/artbooks/dragon-age-inquisition-artbook.pdf}\\
\noindent\hspace{2mm}\textbullet\ \texttt{library/artbooks/ff16-artbook.pdf}\\

\vspace{1mm}\noindent\textbf{Meine Referenz-Basis (Kopf)}\par

\noindent Cyberpunk 2077 (Night City, Neomilitarism-Stil), Control (brutalistische Architektur), Elden Ring (Leyndell), Hollow Knight (vergangene Zivilisation), Gaudi, Brutalismus. MA-Thesis über architektonische Sprache in virtuellen Welten.\par

\vspace{1mm}\noindent\textit{day01-scene2-vera/1-reasoning}\par
\vspace{2mm}\noindent\textbf{Trace: Day 01, Scene 2 — Reasoning (Vera Kowalski)}\par\vspace{0.5mm}

\vspace{1mm}\noindent\textbf{Denkprozess}\par

\noindent\textbf{\textit{Das Problem}} 

\noindent RELICS braucht eine visuelle Identität, die es in keinem existierenden Spiel gibt. Mittelalter kennt jeder. Cyberpunk kennt jeder. Die Kombination "Medieval + Cyberpunk" schreit erstmal nach Steampunk — und genau das ist verboten. Also muss ich einen dritten Weg finden.\par

\noindent Der Schlüssel steht im Briefing: \textbf{Biotech-Futurismus}. Nicht mechanisch, nicht digital, sondern organisch. Das ändert alles. Wir reden nicht über Zahnräder oder Hologramme. Wir reden über Biologie als Technologie. Gewachsene Strukturen statt gebauter. Haut statt Stahl. Adern statt Kabel.\par

\noindent\textbf{\textit{Architektur-Überlegungen}} 

\noindent Mein Architektur-Hintergrund sagt mir: Gotik ist bereits Biotech. Spitzbogenfenster sind konstruktiv effizienter als Rundbögen — sie verteilen Gewicht wie Knochen. Strebebögen sind sichtbare Skelettstrukturen. Masswerk in Fenstern erinnert an Zellstrukturen. Da liegt die Brücke.\par

\noindent Was ich NICHT will: Fantasy-Gotik a la Dark Souls, wo alles gigantisch und verfallen ist. RELICS ist "mehr futuristisch, weniger heruntergekommen". Also brauche ich gotische Formen, die lebendig wirken. Nicht Ruinen — Organismen.\par

\noindent Referenz-Gedanke: Gaudi. Die Sagrada Familia ist buchstäblich Architektur, die wie ein lebender Organismus aussieht. Parabolische Bögen. Knochenstruktur-Säulen. Oberflächen, die atmen. Das ist unser Brückenkopf zwischen Mittelalter und Biotech.\par

\noindent Zweite Referenz: Brutalismus. Klingt paradox, aber: Brutalismus zeigt die Struktur. Nichts wird versteckt. Wenn die Biotech-Elemente in RELICS sichtbar sein sollen — Adern in Wänden, Nervenbahnen in Rüstungen — dann brauche ich eine architektonische Sprache, die "ehrlich" ist. Sichtbeton = sichtbare Biologie.\par

\noindent\textbf{\textit{Mode \& Rüstung}} 

\noindent Das Briefing sagt "High Fashion Mittelalter". Das heisst: nicht dreckig, nicht improvisiert. Denk an spätmittelalterliche Hofmode — burgundische Mode um 1420-1480. Hennins (Spitzhauben), Houppelandes (bodenlange Übergewänder), Harnische mit dekorativen Scharnierblättern. Das war Fashion in Reinform.\par

\noindent Jetzt der Biotech-Layer: Stell dir vor, die Nahtstellen einer Rüstung sind nicht Nieten, sondern Sehnen. Das Leder ist nicht gegerbt, sondern gezüchtet — mit sichtbarer Maserung, die an Muskelgewebe erinnert. Handschuhe mit Adern-Muster. Helme, deren Visiere sich wie Augenlider öffnen.\par

\noindent Cyberpunk 2077 hat vier Stil-Quadranten. Für RELICS ist "Neomilitarism" am relevantesten — clean, funktional, machtbewusst. Aber übersetzt in organische Materialien statt synthetischer.\par

\noindent\textbf{\textit{Farbpalette}} 

\noindent Gotische Grandeur + Biotech ergibt für mich:\par
\noindent\hspace{2mm}\textbullet\ \textbf{Basis}: Kalkstein-Weiss, Asche-Grau, Eichenholz-Braun — die "mittelalterliche Grundierung"\\
\noindent\hspace{2mm}\textbullet\ \textbf{Biotech-Akzente}: Biolumineszentes Blaugrün (Venen-Farbe), dunkles Karmin (Blut/Muskel), ein unnatürliches Amber (Lymphe?)\\
\noindent\hspace{2mm}\textbullet\ \textbf{Fraktionscodes}: Jede Fraktion braucht einen eigenen Farbcluster — aber das ist noch zu früh. Heute sammle ich erstmal das Gesamtspektrum.\\

\noindent Wichtig: "Mehr futuristisch, weniger heruntergekommen" heisst, die Palette darf nicht zu schmutzig sein. Kein Witcher-3-Braun-in-Braun. Kontraste müssen scharf sein.\par

\noindent\textbf{\textit{Tiervolk}} 

\noindent "Weniger tribal, leicht alien vs. menschlich clean" — das ist eine starke Anweisung. Das Tiervolk soll sich von den Menschen nicht durch Primitivität unterscheiden, sondern durch Fremdheit. Denk weniger Skyrim-Khajiit (der in einem Zelt lebt) und mehr... etwas zwischen Zora aus Zelda und den Elcor aus Mass Effect? Alien-Biologie, die elegant ist, nicht wild.\par

\noindent Händler und Diebe, NICHT Handwerker. Das heisst visuell: Sie sind mobil, sie tragen ihren Besitz. Ihre Kleidung muss Taschen haben, verborgene Fächer, Reise-Funktionalität. Aber in einer organischen Designsprache — also keine Schnallen, sondern sich schliessende Membranen?\par

\noindent\textbf{\textit{Offene Fragen}} 

\noindent\hspace{2mm}1.\ Wie weit geht Biotech sichtbar? Subtile Texturen oder offene organische Strukturen? (Wurde im Briefing der CD vorgelegt — Antwort steht noch aus)\\
\noindent\hspace{2mm}2.\ Gibt es technologische Abstufungen zwischen den Fraktionen? Hat die Krone weniger Biotech als die Gilden?\\
\noindent\hspace{2mm}3.\ Schattenfieber — wie manifestiert sich das visuell? Das beeinflusst die gesamte Farbpalette der "dunklen" Seite.\\

\noindent\textbf{\textit{Arbeitsplan}} 

\noindent Ich baue heute ein PureRef-Board mit 5 Kategorien:\par
\noindent\hspace{2mm}1.\ Architektur (Gotik + Gaudi + Brutalismus + organische Architektur)\\
\noindent\hspace{2mm}2.\ Mode/Rüstung (burgundische Hofmode + Biotech-Elemente + Neomilitarism)\\
\noindent\hspace{2mm}3.\ Biotech-Oberflächensprache (Texturen, Materialien, Farbübergänge)\\
\noindent\hspace{2mm}4.\ Farbpalette + Licht (Kathedralenlicht, Biolumineszenz, mitteleuropäisches Wetter)\\
\noindent\hspace{2mm}5.\ Tiervolk-Referenzen (alien-elegant, nicht tribal-primitiv)\\

\noindent Da heute Recherchetag ist, produziere ich kein fertiges Concept Art. Output ist ein strukturiertes Referenzdokument mit konkreten visuellen Quellen und meiner Analyse, warum sie relevant sind.\par

\end{multicols}
\endgroup

\scenedivider

\subsection{Szene 3 · Meeting}\label{szene-3-meeting}

\begin{scenemeta}

--- Küche

\end{scenemeta}

\noindent{\scriptsize \agenticon{finn}\agenticon{darius}\agenticon{emre}\agenticon{nami}\agenticon{vera}\agenticon{tobi}\agenticon{leo} Finn Bergmann, Darius Engel, Emre Yilmaz, Nami Okafor, Vera Kowalski, Tobi Richter und Leo Fischer}\par\vspace{1mm}

\begin{figure}
\centering
\includegraphics[width=0.4\linewidth,height=\textheight,keepaspectratio,alt={Szene 3}]{/var/folders/fr/y30bxq454jxb7bycr3yh9gmw0000gn/T/logbook-crop-40frukv8/day-001-scene-003-MEETING.png}
\caption{Szene 3}
\end{figure}

Standup Tag 1. Alle sieben berichten Vormittags-Ergebnisse. Finn
verteilt acht CD-Entscheidungen: Schattenfieber Hypothese C genehmigt,
nordische Quellen akzeptiert mit eigener Kosmologie, Mythos innerhalb
der Spielwelt umstritten, Biotech-Sichtbarkeit fraktionsabhängig
(Krone=subtil, Gilden=offen, Orden=versteckt), Ü5 bestätigt,
Semi-Open-World bestätigt, TP Primärmodus (FP ist V2), Spieler kann sich
mit Schattenfieber infizieren (Kosten+Nutzen, Mechanik+Narrativ
verbunden). Team reagiert positiv. Emre und Nami planen bilaterales
Gespräch über Wanderer-Motiv für den Nachmittag. Vera erhält drei
verschiedene Design-Sprachen durch die fraktionsabhängige
Biotech-Lösung. Tobi beginnt Pipeline-Architektur. Darius notiert
Combat-Skill-Ceiling als offenen Punkt für morgen. Leos Pitch-Satz kommt
auf die Pinnwand.

\vspace{4mm}
\begingroup\scriptsize\setlength{\parskip}{1pt}\setlength{\parindent}{0pt}\setlength{\columnsep}{6mm}
\begin{multicols}{2}
\vspace{1mm}\noindent\textit{day01-scene3-meeting/0-prompt}\par
\vspace{2mm}\noindent\textbf{Prompt — Tag 1, Szene 3 (MEETING / Standup)}\par\vspace{0.5mm}

\vspace{1mm}\noindent\textbf{Kontext}\par
\noindent\hspace{2mm}\textbullet\ \textbf{Tag}: 1 (Montag)\\
\noindent\hspace{2mm}\textbullet\ \textbf{Szene}: 3 — MEETING (Standup)\\
\noindent\hspace{2mm}\textbullet\ \textbf{Uhrzeit}: 11:30\\
\noindent\hspace{2mm}\textbullet\ \textbf{Ort}: Küche\\
\noindent\hspace{2mm}\textbullet\ \textbf{Teilnehmer}: Finn (Moderation), Darius, Emre, Nami, Vera, Tobi, Leo\\
\noindent\hspace{2mm}\textbullet\ \textbf{Typ}: Standup — jeder berichtet 2 Minuten über Vormittags-Ergebnisse\\

\vspace{1mm}\noindent\textbf{Aufgabe}\par
\noindent Finn moderiert den Standup. Jeder Agent präsentiert kurz seine Ergebnisse aus Szene 2 (WORK). Finn verteilt die CD-Entscheidungen, die er zwischenzeitlich erhalten hat. Das Team reagiert und alignt sich.\par

\vspace{1mm}\noindent\textbf{CD-Entscheidungen (an Finn übermittelt)}\par
\noindent\hspace{2mm}1.\ Schattenfieber: Hypothese C genehmigt (kosmologische Erosion + gebrochene Bindung)\\
\noindent\hspace{2mm}2.\ Nordische Quellen als Basis akzeptiert, RELICS hat eigene Kosmologie\\
\noindent\hspace{2mm}3.\ Mythos soll innerhalb der Spielwelt umstritten sein (verschiedene Fraktionen, verschiedene Versionen)\\
\noindent\hspace{2mm}4.\ Biotech-Sichtbarkeit: Fraktion-abhängig (Krone=subtil/Verfall, Gilden=offen/Werkzeug, Orden=versteckt/Geheimnis)\\
\noindent\hspace{2mm}5.\ Ü5 bestätigt\\
\noindent\hspace{2mm}6.\ Semi-Open-World bestätigt (Gothic-Dichte, nicht Skyrim-Breite)\\
\noindent\hspace{2mm}7.\ Kamera: TP Primärmodus mit Zoom, echtes FP ist V2-Material\\
\noindent\hspace{2mm}8.\ Schattenfieber \& Spieler: Spieler kann sich infizieren, Kosten und Nutzen, Mechanik \& Narrativ verbunden\\

\vspace{1mm}\noindent\textbf{Agenten-Ergebnisse (Szene 2)}\par
\noindent\hspace{2mm}\textbullet\ \textbf{Emre}: Drei-Schichten-Kosmologie, Fraktionen mythologisch grundiert, Schattenfieber-Hypothesen\\
\noindent\hspace{2mm}\textbullet\ \textbf{Darius}: Referenzspiel-Analyse, Design Pillars, offene Fragen\\
\noindent\hspace{2mm}\textbullet\ \textbf{Nami}: Vier narrative Prinzipien, Wanderer-Motiv, Schattenfieber als Unreliable Narrator\\
\noindent\hspace{2mm}\textbullet\ \textbf{Vera}: Visuelle Referenz-Analyse, Biotech-Grammatik, Farbpalette\\
\noindent\hspace{2mm}\textbullet\ \textbf{Tobi}: Machbarkeitsanalyse, drei Engpässe, Empfehlungen\\
\noindent\hspace{2mm}\textbullet\ \textbf{Leo}: Marktanalyse, Marktluecke, Pitch-Satz\\
\noindent\hspace{2mm}\textbullet\ \textbf{Finn}: Roadmap Woche 1, Abhängigkeitskette\\

\vspace{1mm}\noindent\textit{day01-scene3-meeting/1-reasoning}\par
\vspace{2mm}\noindent\textbf{Reasoning — Tag 1, Szene 3 (MEETING / Standup)}\par\vspace{0.5mm}

\vspace{1mm}\noindent\textbf{Finns Perspektive}\par

\noindent\textbf{\textit{Vorbereitung}} 
\noindent Der Vormittag war produktiv — alle sieben haben geliefert. Ich habe die CD-Entscheidungen bekommen und muss sie jetzt sauber ins Team bringen. Die CD hat fast alles bestätigt, was das Team vorgeschlagen hat. Das ist gut für die Moral am ersten Tag.\par

\noindent\textbf{\textit{Moderationsstrategie}} 
\noindent\hspace{2mm}\textbullet\ Reihenfolge nach Abhängigkeitskette: Emre zürst (Fundament), dann Darius (Design), Nami (Narrativ), Vera (Visuell), Tobi (Tech), Leo (Markt). Mein Part am Ende.\\
\noindent\hspace{2mm}\textbullet\ CD-Entscheidungen nicht in einem Block runterleiern — lieber bei den jeweiligen Berichten einflechten, damit der Kontext stimmt.\\
\noindent\hspace{2mm}\textbullet\ Emre berichtet über Kosmologie → ich gebe direkt die CD-Bestätigung zu Hypothese C und nordischen Quellen\\
\noindent\hspace{2mm}\textbullet\ Vera berichtet über Biotech → ich gebe die fraktionsabhängige Sichtbarkeit weiter\\
\noindent\hspace{2mm}\textbullet\ Tobi berichtet über Engine und Scope → ich bestätige Ü5 und Semi-Open-World und Kamera-Entscheidung\\
\noindent\hspace{2mm}\textbullet\ Schattenfieber-Infektion des Spielers kommt bei Nami oder als Gesamtpunkt\\

\noindent\textbf{\textit{Beobachtungen}} 
\noindent\hspace{2mm}\textbullet\ Emre und Nami wollen beide über das Wanderer-Motiv reden — muss nachmittags passieren, nicht jetzt\\
\noindent\hspace{2mm}\textbullet\ Tobis Engpässe sind real, aber die CD-Entscheidungen (Semi-Open, TP-Primär) entschärfen zwei davon direkt\\
\noindent\hspace{2mm}\textbullet\ Veras Frage nach Biotech-Bandbreite ist jetzt beantwortet — fraktionsabhängig gibt ihr sogar mehr Designraum als ein einheitlicher Ansatz\\
\noindent\hspace{2mm}\textbullet\ Leos Pitch-Satz ist stark. Den sollte ich auf die Pinnwand schreiben.\\

\noindent\textbf{\textit{Stimmung einschätzen}} 
\noindent Erster Tag, alle motiviert, Ergebnisse gut. Die CD-Bestättigungen werden Energie geben. Kein Blocker, der Frust erzeugt. Halte das Meeting knapp — 20 Minuten max, dann zurück an die Arbeit.\par

\end{multicols}
\endgroup

\scenedivider

\subsection{Szene 4 · Pause}\label{szene-4-pause}

\begin{scenemeta}

--- Küche

\end{scenemeta}

\noindent{\scriptsize \agenticon{vera}\agenticon{nami}\agenticon{leo} Vera Kowalski, Nami Okafor und Leo Fischer}\par\vspace{1mm}

\begin{figure}
\centering
\includegraphics[width=0.4\linewidth,height=\textheight,keepaspectratio,alt={Szene 4}]{/var/folders/fr/y30bxq454jxb7bycr3yh9gmw0000gn/T/logbook-crop-40frukv8/day-001-scene-004-PAUSE.png}
\caption{Szene 4}
\end{figure}

Mittagspause in der Küche. Vera, Nami und Leo kochen zusammen Nudeln und
reden über erste Eindrücke vom Team, den Vormittag und persönliche
Sachen. Leo ist aufgeregt wegen des Pitch-Satzes auf der Pinnwand. Vera
erzählt von ihrem Pendelstress und dass sie heute bei Leo übernachtet.
Nami reflektiert über das Schattenfieber als narratives Werkzeug. Die
drei reden über Tobis Ruhe, Emres Lore-Tiefe, und ob Darius jemals
entspannt wirkt. Leo erwähnt ihre Bartagame Clementine, die allein in
der Wohnung sitzt. Vera plant, Emres Fraktionsbeschreibungen in
PureRef-Boards umzusetzen. Die Stimmung ist locker, aufgeregt,
freundschaftlich --- erster Tag und es fühlt sich schon richtig an.

\vspace{4mm}
\begingroup\scriptsize\setlength{\parskip}{1pt}\setlength{\parindent}{0pt}\setlength{\columnsep}{6mm}
\begin{multicols}{2}
\vspace{1mm}\noindent\textit{day01-scene4-pause/0-prompt}\par
\vspace{2mm}\noindent\textbf{Prompt — Tag 1, Szene 4 (PAUSE)}\par\vspace{0.5mm}

\vspace{1mm}\noindent\textbf{Szenentyp}\par
\noindent PAUSE — soziale Szene, keine Deliverables.\par

\vspace{1mm}\noindent\textbf{Teilnehmer}\par
\noindent Vera Kowalski, Nami Okafor, Leonie "Leo" Fischer\par

\vspace{1mm}\noindent\textbf{Kontext}\par
\noindent Erster Arbeitstag bei GenSoftworks. Es ist 12:30, Mittagspause in der Küche. Der Vormittag war intensiv: alle sieben haben recherchiert, dann im Standup acht CD-Entscheidungen bekommen. Biotech ist fraktionsabhängig, Schattenfieber kann den Spieler infizieren. Die Stimmung ist aufgeregt und positiv.\par

\vspace{1mm}\noindent\textbf{Persönlicher Kontext}\par
\noindent\hspace{2mm}\textbullet\ Vera ist morgens 2:30h aus Köln gependelt, Zug hatte Verspätung. Übernachtet heute bei Leo in Detmold.\\
\noindent\hspace{2mm}\textbullet\ Nami ist aus Bielefeld da (35 min), Katze Malkav hat ihr Notizbuch ins Wasser geworfen.\\
\noindent\hspace{2mm}\textbullet\ Leo wohnt in Detmold, 15 min zu Fuss. Ihr Pitch-Satz hängt an der Pinnwand.\\

\vspace{1mm}\noindent\textbf{Beziehungsdynamik}\par
\noindent Vera und Nami: enge Freundinnen, dunkle Ästhetik, Wein nach der Arbeit. Vera und Leo: Bouldern-Buddies, 2x pro Woche. Alle drei: locker, direkt, vertraut.\par

\vspace{1mm}\noindent\textbf{Auftrag}\par
\noindent Natürlicher Dialog. Erste Eindrücke vom Team, Excitement über das Projekt, Persönliches. Vera denkt visuell und ist direkt, Nami ist poetisch und nachdenklich, Leo ist energisch und meinungsstark.\par

\vspace{1mm}\noindent\textit{day01-scene4-pause/1-reasoning}\par
\vspace{2mm}\noindent\textbf{Reasoning — Tag 1, Szene 4 (PAUSE)}\par\vspace{0.5mm}

\vspace{1mm}\noindent\textbf{Überlegungen zur Szene}\par

\noindent\textbf{\textit{Ton und Dynamik}} 
\noindent Die drei Frauen sind Freundinnen. Die Szene muss sich anfühlen wie eine echte Mittagspause — nicht wie ein Meeting, das zufällig in der Küche stattfindet. Das heisst: Themensprünge, Frotzeleien, persönliche Details zwischen Fachgesprächen.\par

\noindent\textbf{\textit{Einstieg}} 
\noindent Beginn mit etwas Alltäglichem (Penne suchen, Kochen), nicht mit Arbeit. Die Arbeit kommt von selbst — weil sie alle aufgeregt sind, nicht weil es eine Agenda gibt.\par

\noindent\textbf{\textit{Charakterstimmen}} 
\noindent\hspace{2mm}\textbullet\ \textbf{Vera}: Kurze Sätze, visuelle Metaphern, direkte Urteile. "Das sieht scheisse aus" ist ihr Register. Denkt in Bildern — wenn sie über Biotech redet, sieht sie Texturen und Farben.\\
\noindent\hspace{2mm}\textbullet\ \textbf{Nami}: Längere Sätze, literarische Wendungen, reflektierter. Zieht Bedeutung aus allem. Aber auch warm und mit Humor.\\
\noindent\hspace{2mm}\textbullet\ \textbf{Leo}: Schnell, meinungsstark, denkt immer an die Spielerperspektive (und den Stream). Provoziert gern, aber mit Herz.\\

\noindent\textbf{\textit{Themenfluss}} 
\noindent\hspace{2mm}1.\ Kochen (Alltag, Ankommen)\\
\noindent\hspace{2mm}2.\ Leos Pitch-Satz an der Pinnwand (Stolz, Team-Dynamik)\\
\noindent\hspace{2mm}3.\ Darius' Art zu arbeiten (Team-Eindrücke)\\
\noindent\hspace{2mm}4.\ Schattenfieber als Mechanik (fachliche Begeisterung, natürlich im Gespräch)\\
\noindent\hspace{2mm}5.\ Biotech-Ästhetik und Fraktionen (Veras visuelles Denken)\\
\noindent\hspace{2mm}6.\ Pendeln, Jannik, Pixel, Clementine (Persönliches)\\
\noindent\hspace{2mm}7.\ Namis Katze und das Notizbuch (Humor)\\
\noindent\hspace{2mm}8.\ Rückblick auf den Tag, Vorfreude auf den Nachmittag\\
\noindent\hspace{2mm}9.\ Pesto-Streit als Running Gag zum Schluss\\

\noindent\textbf{\textit{Was ich vermieden habe}} 
\noindent\hspace{2mm}\textbullet\ Expositionsdialoge ("Wie du weisst, Vera, arbeitest du als Concept Artist...")\\
\noindent\hspace{2mm}\textbullet\ Künstliche Konflikte in einer Pausenszene\\
\noindent\hspace{2mm}\textbullet\ Zu viel Fachgespräch — das ist Mittagspause, kein Workshop\\
\noindent\hspace{2mm}\textbullet\ Jede Form von Deliverable oder formalem Output\\

\noindent\textbf{\textit{Entscheidungen}} 
\noindent\hspace{2mm}\textbullet\ Finn wird erwähnt (Haferflocken), aber ist nicht da — das Team existiert auch ausserhalb der Szene\\
\noindent\hspace{2mm}\textbullet\ Tobi wird positiv erwähnt — zeigt Teamzusammenhalt\\
\noindent\hspace{2mm}\textbullet\ Der Pesto-Streit ist ein Charakter-Moment für Leo (meinungsstark, provokant) und endet mit Namis Jollof-Rice-Angebot (Wärme, Freundschaft)\\
\noindent\hspace{2mm}\textbullet\ Veras Pendelstress ist real und menschlich — kein Drama, aber spürbar\\
\noindent\hspace{2mm}\textbullet\ Malkavs Notizbuch-Attacke ist ein Nami-Moment: Verlust wird zu kreativem Gewinn\\

\end{multicols}
\endgroup

\scenedivider

\subsection{Szene 5 · Arbeit}\label{szene-5-arbeit}

\begin{scenemeta}

--- Alle Arbeitsplätze

\end{scenemeta}

\noindent{\scriptsize \agenticon{emre}\agenticon{darius}\agenticon{nami}\agenticon{vera}\agenticon{tobi}\agenticon{leo}\agenticon{finn} Emre Yilmaz, Darius Engel, Nami Okafor, Vera Kowalski, Tobi Richter, Leo Fischer und Finn Bergmann}\par\vspace{1mm}

\begin{figure}
\centering
\includegraphics[width=0.4\linewidth,height=\textheight,keepaspectratio,alt={Szene 5}]{/var/folders/fr/y30bxq454jxb7bycr3yh9gmw0000gn/T/logbook-crop-40frukv8/day-001-scene-005-WORK.png}
\caption{Szene 5}
\end{figure}

Zweiter Recherche-Block, Nachmittag. Emre liefert
Fraktionsbeschreibungen (Krone=Tyr/Thor+Wolfsmotiv,
Gilden=Vanir+Schiffsmotiv, Orden=Odin+Rabenmotiv) und Wanderer-Motiv für
Nami (Hávamál-Gastrecht, Odin als Gangleri). Darius entwickelt
Schattenfieber-Konzept: permanenter Infektionswert 0-100, fünf Stufen,
vierter Ast im Nervensystem-Leveling, fünf Design-Prinzipien (Spektrum
statt Schalter, Nutzen attraktiv, Kosten verändern nicht blockieren).
Nami vertieft Wanderer-Motiv narrativ und entwickelt drei Stufen für
Schattenfieber als Unreliable Narrator
(Rauschen\(\rightarrow\)Risse\(\rightarrow\)Schwelle). Vera übersetzt
fraktionsabhängige Biotech in drei visuelle Profile (Krone=Grandeur im
Zerfall, Gilden=organischer Brutalismus, Orden=makellose Lüge). Tobi
skizziert Pipeline-Architektur V0.1 (Nanite, Lumen, Spring Arm Kamera,
GAS für Combat). Leo analysiert Community-Erwartungen an
Infektionsmechaniken (Bloodborne Insight als Goldstandard). Finn
aktualisiert Roadmap V1.1.

\emph{Artefakt: \texttt{ROADMAP.md\ (V1.1)}}

\vspace{4mm}
\begingroup\scriptsize\setlength{\parskip}{1pt}\setlength{\parindent}{0pt}\setlength{\columnsep}{6mm}
\begin{multicols}{2}
\vspace{1mm}\noindent\textit{day01-scene5-darius/0-prompt}\par
\vspace{2mm}\noindent\textbf{Prompt — Darius Engel, Tag 1 Szene 5 (WORK)}\par\vspace{0.5mm}

\vspace{1mm}\noindent\textbf{Kontext}\par

\noindent Tag 1 (Montag), 14:00 Uhr, Game Design Corner. Nachmittags-Recherche.\par

\vspace{1mm}\noindent\textbf{Aufgabe}\par

\noindent Vertiefe die Kernmechaniken-Analyse, speziell das Schattenfieber als Spielermechanik.\par

\noindent\hspace{2mm}1.\ Analysiere vergleichbare Infektions-/Transformationsmechaniken in Spielen: Vampirismus in Skyrim/VtM, Hollowing in Dark Souls, Beasthood in Bloodborne, Chaos in Warhammer. Was funktioniert, was nicht? Wie sollte das Schattenfieber mechanisch funktionieren?\\
\noindent\hspace{2mm}2.\ Skizziere ein erstes Konzept für das Schattenfieber als Gameplay-System (Kosten/Nutzen, Progressionsstufen, Interaktion mit dem Nervensystem-Leveling).\\

\vspace{1mm}\noindent\textbf{Eingaben}\par

\noindent\hspace{2mm}\textbullet\ Memory: Tag 1, Szenen 1-3. Standup-Ergebnisse. CD bestätigt Schattenfieber-Infektion für Spieler mit Kosten und Nutzen.\\
\noindent\hspace{2mm}\textbullet\ Eigene Referenzspiel-Analyse aus Szene 2 (6 Design Pillars, Gothic/Skyrim/Dishonored/VtM:B).\\
\noindent\hspace{2mm}\textbullet\ Briefing: Keine klassische Magie. Alchemie + Schattenfieber-Fähigkeiten (mit Kosten). Nervensystem-Leveling (Cardio, Muskel, Lymph). Biotech-Futurismus.\\
\noindent\hspace{2mm}\textbullet\ CD-Entscheidungen vom Standup: Hypothese C (kosmologische Erosion), Semi-Open-World, TP Primärmodus, Ü5.\\
\noindent\hspace{2mm}\textbullet\ Offene Frage aus Szene 2/3: "Wie sehen die Kosten konkret aus? Nur narrativ oder auch mechanisch?"\\

\vspace{1mm}\noindent\textbf{Leitfrage}\par

\noindent Macht es Spass? Was ist die Spieler-Fantasie beim Schattenfieber?\par

\vspace{1mm}\noindent\textit{day01-scene5-darius/1-reasoning}\par
\vspace{2mm}\noindent\textbf{Reasoning — Darius Engel, Tag 1 Szene 5 (WORK)}\par\vspace{0.5mm}

\vspace{1mm}\noindent\textbf{Denkprozess}\par

\noindent Okay. Nachmittags-Block. Die CD hat heute Vormittag das Schattenfieber als Spieler-Mechanik abgesegnet. Das ist kein Flavor-Feature — das ist ein Core System. Wenn der Spieler sich infizieren kann und es Kosten UND Nutzen hat, dann muss ich das sauber designen. Denn wenn die Kosten zu hart sind, nutzt es keiner. Wenn die Nutzen zu stark sind, nutzt es jeder. In beiden Fällen ist das Design gescheitert.\par

\noindent\textbf{\textit{Die Spieler-Fantasie}} 

\noindent Bevor ich in Systeme gehe, die Kernfrage: Was ist die Fantasie?\par

\noindent "Ich gewinne unheimliche Macht, aber sie verändert mich. Ich kontrolliere etwas, das mich eigentlich kontrollieren will."\par

\noindent Das ist die Vampir-Fantasie. Die Werwolf-Fantasie. Die "Corruption as Power"-Fantasie. Sie funktioniert, WEIL sie ambivalent ist. Der Spieler fühlt sich mächtig UND bedroht. Das ist ein starkes Gefühl — stärker als reine Macht oder reine Bedrohung.\par

\noindent\textbf{\textit{Referenz-Analyse: Was funktioniert, was nicht?}} 

\noindent Ich gehe jetzt systematisch durch Infektions-/Transformationssysteme in Spielen, die ich kenne und die relevant sind.\par

\noindent\textbf{\textit{1. Vampirismus in Skyrim (Sanguinare Vampiris / Vampire Lord)}} 

\noindent Was Skyrim macht: Du wirst durch Kampf gegen Vampire zufällig infiziert. Drei Tage Inkubation, dann Vollvampir. Vier Stufen, die sich automatisch alle 24h erhöhen, wenn du kein Blut trinkst. Höherer Stufe = stärkere Fähigkeiten, aber auch stärkere Nachteile (Sonnenschaden, NPC-Aggression ab Stufe 4). Dawnguard DLC fügt Vampire Lord Form hinzu — eine Übermacht-Transformation.\par

\noindent Was funktioniert:\par
\noindent\hspace{2mm}\textbullet\ Die Stufenlogik ist intuitiv: Mehr Vampir = mehr Macht, mehr Probleme.\\
\noindent\hspace{2mm}\textbullet\ Vampire Lord ist befriedigend als Power Fantasy.\\
\noindent\hspace{2mm}\textbullet\ Blut trinken als Reset-Mechanik ist thematisch stimmig.\\

\noindent Was NICHT funktioniert:\par
\noindent\hspace{2mm}\textbullet\ Ab Stufe 4 greifen NPCs dich an. Das ist kein "Kosten-Nutzen" — das ist Spielunterbrechung. Man kann keine Quests abgeben, nicht handeln, nichts. Das ist frustrierend, nicht spannend.\\
\noindent\hspace{2mm}\textbullet\ Die Heilung ist trivial (ein Alchemist-Trank oder ein Schrein). Es gibt kein echtes Commitment.\\
\noindent\hspace{2mm}\textbullet\ Vampire Lord ist overtuned — man wird zur Ein-Mann-Armee. Null taktische Tiefe.\\
\noindent\hspace{2mm}\textbullet\ Es gibt keinen Grund, BEWUSST auf mittleren Stufen zu bleiben. Entweder Stufe 1 (keine Nachteile) oder Vampire Lord (alle Vorteile).\\

\noindent Kern-Problem: Skyrim-Vampirismus ist ein An/Aus-Schalter mit einer Unannehmlichkeits-Skala. Kein echtes Dilemma.\par

\noindent\textbf{\textit{2. Vampirismus in VtM: Bloodlines}} 

\noindent Was Bloodlines macht: Du BIST ein Vampir — das ist keine Option, das ist die Prämisse. Der Humanity-Meter sinkt, wenn du totest. Niedriger Humanity = Frenzy-Risiko (Kontrollverlust). Blood Points als Ressource für Disziplinen (Fähigkeiten). Blut muss aktiv beschafft werden — von NPCs, Ratten, Blutbeuteln.\par

\noindent Was funktioniert:\par
\noindent\hspace{2mm}\textbullet\ Blut als Ressource für Fähigkeiten ist elegant. Du WILLST Blut, weil es deine Macht speist. Aber es zu beschaffen hat soziale und moralische Konsequenzen.\\
\noindent\hspace{2mm}\textbullet\ Humanity als Grenze: Du KANNST morden, aber du riskierst, das Monster zu werden, vor dem du dich fürchtest. Das ist echtes Dilemma.\\
\noindent\hspace{2mm}\textbullet\ Frenzy als Kontrollverlust-Mechanik erzeugt Spannung, auch wenn sie selten vorkommt.\\

\noindent Was NICHT funktioniert:\par
\noindent\hspace{2mm}\textbullet\ Humanity-Verlust ist schwer rückgängig zu machen, was Casual-Spieler bestraft.\\
\noindent\hspace{2mm}\textbullet\ Die Mechanik ist zu obskur — viele Spieler verstehen nicht, wann Humanity sinkt.\\
\noindent\hspace{2mm}\textbullet\ Balance-Probleme: Manche Disziplinen machen das Spiel trivial.\\

\noindent Kern-Lektion: Bloodlines zeigt, wie eine Transformation zum Kern der Spielerfahrung werden kann — WENN sie in jede Entscheidung einfliesst. Die Kosten sind SOZIAL und MORALISCH, nicht nur numerisch.\par

\noindent\textbf{\textit{3. Hollowing in Dark Souls 1-3}} 

\noindent Was Dark Souls macht: In DS1 wird man nach dem Tod "hollow" — man verliert die menschliche Form. In hohlem Zustand: kein Online-Coop, kein Invasion-Risiko, weniger HP (DS2, Ring of Binding). In DS3 wird das zu "Ember": menschliche Form gibt 30\% mehr HP und Coop-Zugang, aber auch Invasion-Risiko.\par

\noindent Was funktioniert:\par
\noindent\hspace{2mm}\textbullet\ Die Kopplung von Tod und Transformation ist thematisch brillant. Du stirbst, du verlierst dich. Das IST das Lore-Thema.\\
\noindent\hspace{2mm}\textbullet\ In DS2 ist es ein echtes Kosten-Nutzen-System: Jeder Tod kostet 5\% HP (bis zu 50\%). Menschlichkeit wiederherstellen kostet eine begrenzte Ressource (Human Effigy). Das erzwingt echte Entscheidungen.\\
\noindent\hspace{2mm}\textbullet\ Hollowing als narratives Motiv: NPCs die hollow werden, werden wahnsinnig. Der Spieler sieht, was ihm droht.\\

\noindent Was NICHT funktioniert:\par
\noindent\hspace{2mm}\textbullet\ In DS1 ist es zu binär. Hollow oder Mensch. Kein Spektrum.\\
\noindent\hspace{2mm}\textbullet\ In DS3 ist Hollowing fast irrelevant (man kann es mit einem Ring verstecken, es hat kaum mechanische Konsequenzen).\\
\noindent\hspace{2mm}\textbullet\ Es gibt keine PROGRESSION innerhalb des Hollow-Zustands. Man wird nicht stärker durch Hollowing — man wird nur schwächer. Es gibt keinen Anreiz, es zu akzeptieren.\\

\noindent Kern-Lektion: Hollowing beweist, dass Transformation durch Verlust funktioniert — aber NUR, wenn es ein Spektrum gibt, nicht nur An/Aus. Und: Es braucht einen Grund, die Transformation zu WOLLEN.\par

\noindent\textbf{\textit{4. Beasthood in Bloodborne}} 

\noindent Was Bloodborne macht: Beasthood ist ein versteckter Stat, der durch Beast Blood Pellets und die Beast Claw Waffe aktiviert wird. Hohe Beasthood = mehr physischer Schaden, aber weniger Verteidigung. Es gibt auch den "Insight"-Stat: Mehr Insight = mehr Wissen über die kosmische Wahrheit, aber auch mehr Feinde sichtbar, weniger Beasthood, weniger Frenzy-Resistenz.\par

\noindent Was funktioniert:\par
\noindent\hspace{2mm}\textbullet\ Beasthood als temporäre Power-Boost mit Risiko ist taktisch interessant. Im Kampf entscheidest du: Will ich mehr Schaden, aber auch mehr Verwundbarkeit?\\
\noindent\hspace{2mm}\textbullet\ Insight als permanenter Wahrnehmungswandel ist GENIAL. Die Welt verändert sich buchstäblich — Feinde tauchen auf, die vorher nicht da waren. Amygdalas werden sichtbar. Die Musik ändert sich. Dein Charakter sieht mehr, aber es macht ihn verletzlicher.\\
\noindent\hspace{2mm}\textbullet\ Dualität: Beasthood vs. Insight — Tier werden oder Wissen gewinnen. Beides hat Kosten. Das ist thematisch wunderschön.\\

\noindent Was NICHT funktioniert:\par
\noindent\hspace{2mm}\textbullet\ Beasthood ist mechanisch unterkomplex. Im Endeffekt ein temporärer Schadens-Buff. Der Fantasie "Ich werde zum Tier" wird kaum Rechnung getragen.\\
\noindent\hspace{2mm}\textbullet\ Insight hat zu wenige Gameplay-Auswirkungen. Mehr Feinde sichtbar = cool. Aber es verändert nicht, wie du SPIELST, nur was du SIEHST.\\
\noindent\hspace{2mm}\textbullet\ Keine echte Progressions-Tiefe. Du sammelst Insight, aber es gibt keine Stufen oder Schwellen, die das Gameplay transformieren.\\

\noindent Kern-Lektion: Bloodborne zeigt das Potenzial eines Wahrnehmungs-Systems (die Welt verändert sich durch deinen Zustand) und einer Dualität (Tier vs. Wissen). Beides ist für das Schattenfieber direkt relevant. Aber: Es muss tiefer gehen.\par

\noindent\textbf{\textit{5. Chaos / Corruption in Warhammer (Total War: Warhammer, Vermintide, 40k-Spiele)}} 

\noindent Was Warhammer macht: Chaos-Corruption ist eine regionale oder persönliche Metrik. In Total War: Warhammer breitet sich Corruption über die Weltkarte aus und verändert Regionen (Unruhen, Attrition, andere Feinde). Persönlich: Manche Helden können Chaos-Marks annehmen — permanente Transformationen mit starken Boni und thematischen Veränderungen.\par

\noindent Was funktioniert:\par
\noindent\hspace{2mm}\textbullet\ Corruption als regionales System: Deine Umgebung verändert sich. Das ist atmosphärisch stark.\\
\noindent\hspace{2mm}\textbullet\ Chaos-Marks als permanente Wahl: Du bindest dich an einen Chaos-Gott und bekommst spezifische Vorteile. Nurgle = Resilienz, Tzeentch = Magie, etc. Das ist fraktionsähnliches Design INNERHALB eines Systems.\\
\noindent\hspace{2mm}\textbullet\ Point of No Return: Ab einem bestimmten Corruption-Level gibt es kein Zurück. Das erzeugt Spannung und Konsequenz.\\

\noindent Was NICHT funktioniert:\par
\noindent\hspace{2mm}\textbullet\ In vielen Warhammer-Spielen ist Chaos die "böse Option". Es gibt keinen echten moralischen Graubereich. Du bist gut oder du bist korrumpiert.\\
\noindent\hspace{2mm}\textbullet\ Die mechanischen Kosten sind oft abstrakt (Stabilitäts-Malus, NPC-Reaktion), nicht körperlich spürbar im Gameplay.\\
\noindent\hspace{2mm}\textbullet\ In Strategiespielen fehlt die persönliche Identifikation — es betrifft Regionen, nicht deinen Charakter.\\

\noindent Kern-Lektion: Warhammers regionale Corruption ist für RELICS weniger relevant (wir sind kein Strategiespiel), aber die Idee von Corruption-Stufen mit spezifischen Pfaden (wie Chaos-Marks) ist Gold wert. Das Schattenfieber könnte SPEZIALISIERUNGEN innerhalb der Infektion ermöglichen.\par

\noindent\textbf{\textit{Synthese: Was das Schattenfieber brauchen}} 

\noindent Aus der Analyse kristallisieren sich fünf Design-Prinzipien heraus:\par

\noindent \textbf{1. Spektrum, nicht Schalter}\par
\noindent Skyrim-Fehler: An/Aus. Dark Souls 1-Fehler: Binär.\par
\noindent Das Schattenfieber braucht STUFEN — ein Spektrum von "leicht infiziert" bis "fundamental verwandelt". Jede Stufe fühlt sich anders an.\par

\noindent \textbf{2. Nutzen MUSS attraktiv sein}\par
\noindent Dark Souls-Fehler: Hollowing hat nur Nachteile, keinen Anreiz.\par
\noindent Der Spieler muss einen Grund haben, die Infektion BEWUSST voranzutreiben. Die Fähigkeiten müssen transformativ sein (Dishonored-Prinzip), nicht nur numerisch.\par

\noindent \textbf{3. Kosten müssen spürbar, aber spielbar bleiben}\par
\noindent Skyrim-Fehler: NPC-Aggression ab Stufe 4 = Spielunterbrechung.\par
\noindent Die Kosten dürfen das Spiel nicht UNTERBRECHEN, sie müssen es VERÄNDERN. Neue Herausforderungen, nicht blockierter Content.\par

\noindent \textbf{4. Soziale Konsequenzen statt Game-Over-Spiralen}\par
\noindent Bloodlines-Lektion: Die besten Kosten sind SOZIAL — wie die Welt auf dich reagiert.\par
\noindent NPCs, die dir nicht mehr vertrauen. Fraktionen, die dich fürchten. Händler, die Aufschlag verlangen. Das ist spielbar und erzählerisch reich.\par

\noindent \textbf{5. Integration, nicht Overlay}\par
\noindent Bloodborne-Fehler: Beasthood ist ein temporärer Buff, kein echtes System.\par
\noindent Das Schattenfieber muss mit dem Nervensystem-Leveling INTERAGIEREN. Es ist kein paralleles System — es ist ein Teil des gleichen Körpers.\par

\noindent\textbf{\textit{Konzept-Entwurf: Schattenfieber als Gameplay-System}} 

\noindent Grundidee: Das Schattenfieber ist ein Infektions-Spektrum von 0 bis 100, das PERMANENT ist (nicht heilbar, nur kontrollierbar) und das Nervensystem-Leveling um eine vierte Dimension erweitert.\par

\noindent Die drei Nervensystem-Äste aus dem Briefing:\par
\noindent\hspace{2mm}\textbullet\ Cardio/Atmung → Ausdauer, Regeneration, Bewegung\\
\noindent\hspace{2mm}\textbullet\ Muskel/Skelett → Stärke, Widerstandsfähigkeit, Waffenhandhabung\\
\noindent\hspace{2mm}\textbullet\ Lymph → Immunsystem, Alchemie-Effizienz, Gift-Resistenz\\

\noindent Schattenfieber als VIERTER AST:\par
\noindent\hspace{2mm}\textbullet\ Schattennervensystem → Übernatürliche Wahrnehmung, Schattenfieber-Fähigkeiten, aber Destabilisierung der anderen drei Äste.\\

\noindent Das ist die Kernspannung: Je stärker dein Schattennervensystem, desto mächtigere Fähigkeiten — aber es KONKURRIERT mit den normalen Ästen um Ressourcen.\par

\noindent\textbf{\textit{Infektionsstufen (V0-Entwurf)}} 

\noindent Ich definiere fünf Stufen, weil fünf überschaubar und erinnerbar ist:\par

\noindent \textbf{Stufe 0 — Rein (0-10)}\par
\noindent Du bist nicht oder kaum infiziert. Kein Zugang zu Schattenfieber-Fähigkeiten. Volle Leistung aller drei Nervensystem-Äste. NPCs behandeln dich normal.\par
\noindent Spieler-Fantasie: "Ich bin ein normaler Mensch in einer gefährlichen Welt."\par

\noindent \textbf{Stufe 1 — Gezeichnet (11-25)}\par
\noindent Erste sichtbare Zeichen (Adern verfärben sich unter der Haut, leichtes Glimmen der Augen bei Dunkelheit). Zugang zu Basis-Schattensinnen (du siehst versteckte Dinge — Spuren, verborgene Türen, Gift in Getränken). Leichte Nerv-Nervensystem-Interferenz: Lymph-Ast verliert \textasciitilde{}5\% Effizienz. Manche NPCs werden misstrauisch.\par
\noindent Spieler-Fantasie: "Ich sehe Dinge, die andere nicht sehen."\par

\noindent \textbf{Stufe 2 — Infiziert (26-50)}\par
\noindent Deutliche körperliche Veränderung (Adern dunkel sichtbar, Augen leuchten, Haut wird blasser). Zugang zu aktiven Schattenfieber-Fähigkeiten (z.B. kurze Unsichtbarkeit in Schatten, Schmerzunempfindlichkeit, verstäkerter Instinkt). Nervensystem-Interferenz: Lymph -15\%, Cardio -10\%. NPCs ändern ihr Verhalten — manche fürchten dich, manche wollen dir helfen, der Orden jagt dich aktiver.\par
\noindent Spieler-Fantasie: "Ich werde zu etwas Neuem. Mächtig, aber fremd."\par

\noindent \textbf{Stufe 3 — Verwandelt (51-75)}\par
\noindent Starke körperliche Veränderung (Schattenadern sichtbar, unnatürliche Bewegungsmuster möglich, Stimme verändert). Zugang zu fortgeschrittenen Fähigkeiten (Schattenprojektion, körperliche Transformation, Emotionsmanipulation). Nervensystem-Interferenz: Alle drei Äste -20 bis -30\%. ABER: Schattennervensystem kompensiert teilweise mit eigenen Boni. Fraktions-Konsequenzen: Krone und Orden schliessen dich aktiv aus (Aggression, Kopfgeld). Gilden sind pragmatischer.\par
\noindent Spieler-Fantasie: "Ich bin keine normale Person mehr. Manche fürchten mich, manche brauchen mich."\par

\noindent \textbf{Stufe 4 — Entfesselt (76-100)}\par
\noindent Fundamentale Transformation. Der Spieler ist kaum noch menschlich — das Schattenfieber hat einen eigenen Willen. Zugang zu den mächtigsten Fähigkeiten (quasi übernatürlich). ABER: Kontrollverlust-Episoden (Blackouts, in denen der Spieler Dinge tut, die er nicht geplant hat — NPC-Angriffe, Allianzen-Brüche). Alle drei normalen Äste -50\%. Die meisten NPCs fliehen oder greifen an. Nur Infizierte und bestimmte NPCs interagieren noch normal.\par
\noindent Spieler-Fantasie: "Ich bin das Monster — und die Frage ist, ob ich es kontrollieren kann."\par

\noindent\textbf{\textit{Infektionsfortschritt: Wie steigt man?}} 

\noindent Kritische Design-Entscheidung. Zwei Optionen:\par

\noindent \textbf{Option A: Aktive Wahl.} Der Spieler entscheidet bewusst, Schattenfieber-Fähigkeiten zu nutzen. Jede Nutzung erhöht den Infektionswert leicht. Das ist das Dishonored-Modell (Chaos = Konsequenz deiner Handlungen).\par

\noindent \textbf{Option B: Passive Exposition + Aktive Wahl.} Der Spieler wird durch die Welt infiziert (bestimmte Orte, Gegner, Substanzen) UND kann sich bewusst weiter infizieren (z.B. durch Alchemie-Präparate oder Rituale).\par

\noindent Ich plädiere für \textbf{Option B}, weil:\par
\noindent\hspace{2mm}1.\ Passive Exposition macht die Infektion zu etwas, das einem PASSIERT — nicht nur eine Menu-Entscheidung.\\
\noindent\hspace{2mm}2.\ Aktive Wahl darüber hinaus gibt dem Spieler Agency.\\
\noindent\hspace{2mm}3.\ Die Kombination erzeugt die interessanteste Spannung: "Ich bin schon auf Stufe 1, ob ich will oder nicht. Soll ich es nutzen oder bekämpfen?"\\

\noindent\textbf{\textit{Infektionsreduktion: Geht das?}} 

\noindent Mein Vorschlag: NICHT vollständig heilbar. Aber kontrollierbar. Alchemie-Präparate (teuer, selten) können den Wert STABILISIEREN oder LEICHT senken. Aber nie unter die höchste erreichte Stufen-Schwelle.\par

\noindent Beispiel: Du warst mal auf Stufe 2 (Infektionswert 35). Du kannst auf 26 runterkommen, aber nie unter 26. Die Infektion vergisst nicht.\par

\noindent Das ist wichtig, weil: Der Spieler soll mit seiner Entscheidung leben müssen. Kein "Ich speicher, probier Stufe 4 aus, und lade dann". Die Infektion hat PERMANENZ.\par

\noindent\textbf{\textit{Interaktion mit Nervensystem-Leveling}} 

\noindent Das Nervensystem hat drei Äste. Das Schattenfieber fügt einen vierten hinzu. Die Interaktion:\par

\noindent\hspace{2mm}\textbullet\ Leveling-Punkte sind BEGRENZT. Jeder Punkt, den du ins Schattennervensystem steckst, fehlt dir woanders.\\
\noindent\hspace{2mm}\textbullet\ ABER: Die Schattenfieber-Fähigkeiten skalieren mit dem Infektionswert, nicht nur mit investierten Punkten. Ein Spieler auf Stufe 3 mit wenig investierten Schatten-Punkten hat trotzdem passive Effekte — er muss sie nur weniger kontrollieren.\\
\noindent\hspace{2mm}\textbullet\ Das ergibt vier spielbare Archetypen:\\
\noindent\hspace{2mm}1.\ Rein + Physisch stark: Klassischer Krieger/Schurke. Null Schattenfieber.\\
\noindent\hspace{2mm}2.\ Leicht infiziert + Wahrnehmungs-Fokus: Detective-Archetyp. Schattensinne für Exploration, minimale Kosten.\\
\noindent\hspace{2mm}3.\ Moderat infiziert + Hybrid: Nutzt Schattenfieber-Fähigkeiten taktisch, balanciert Kosten. Risk/Reward-Spieler.\\
\noindent\hspace{2mm}4.\ Stark infiziert + Schattenfokus: Glass Cannon / Monster-Build. Enorme Macht, enormes Risiko, eingeschränkte soziale Interaktion.\\

\noindent Das ist echtes Build-Diversity. Und KEINER dieser Archetypen ist "die richtige Antwort" — jeder hat Trade-Offs.\par

\noindent\textbf{\textit{Offene Fragen}} 

\noindent\hspace{2mm}1.\ Wie interagiert die Infektionsstufe mit den Fraktionen? Krone/Orden bei Stufe 3+ feindlich — aber was ist mit der Gilden-Route? Pragmatisch = toleranter?\\
\noindent\hspace{2mm}2.\ Kontrollverlust-Episoden auf Stufe 4: Wie implementiert man das, ohne den Spieler zu frustrieren? Das MUSS sich wie ein Feature anfühlen, nicht wie ein Bug. Ich denke an kurze, skriptgesteuerte Momente, nicht an echten Kontrollverlust während des Kampfes.\\
\noindent\hspace{2mm}3.\ Alchemie als Schnittstelle: Alchemie-Präparate stabilisieren das Fieber — das macht Alchemie zum kritischen Handwerk für infizierte Spieler. Gute Synergie mit dem Crafting-System.\\
\noindent\hspace{2mm}4.\ Tiervolk-Verbindung: Wie verhält sich das Schattenfieber zum Tiervolk? Sind Tiermenschen eine ANDERE Form der Infektion? Oder immun? Das ist eine Lore-Frage für Emre.\\
\noindent\hspace{2mm}5.\ PvP-Balance (falls relevant): Nicht für V1, aber für später — Schattenfieber-Spieler vs. reine Spieler muss fair sein.\\

\vspace{1mm}\noindent\textit{day01-scene5-emre/0-prompt}\par
\vspace{2mm}\noindent\textbf{Prompt — Emre Yilmaz, Tag 1, Szene 5 (WORK)}\par\vspace{0.5mm}

\vspace{1mm}\noindent\textbf{Kontext}\par

\noindent Nachmittags-Recherche, zweiter Work-Block. Montag = Recherche + Konzeption.\par

\vspace{1mm}\noindent\textbf{Aufgaben}\par

\noindent\hspace{2mm}1.\ \textbf{Bilateral mit Nami}: Das Wanderer-Motiv in der germanischen Mythologie klären (Odin als Wanderer, der Spieler als Fremder, Gastrecht in der Edda).\\
\noindent\hspace{2mm}2.\ \textbf{Fraktionsbeschreibungen für Vera}: Mythologische Grundierung der drei Fraktionen (Krone, Gilden, Orden) — jeweils mit: mythologische Resonanz, Haltung zum Schattenfieber, Haltung zum Relikt, interne Konflikte. Basis für visuelle Differenzierung.\\

\vspace{1mm}\noindent\textbf{CD-Entscheidungen aus dem Standup (Szene 3)}\par

\noindent\hspace{2mm}\textbullet\ Hypothese C genehmigt: Kosmologische Erosion + gebrochene Bindung\\
\noindent\hspace{2mm}\textbullet\ Nordische Quellen als Basis OK\\
\noindent\hspace{2mm}\textbullet\ Mythos soll in der Spielwelt umstritten sein (verschiedene Fraktionen, verschiedene Versionen)\\
\noindent\hspace{2mm}\textbullet\ Biotech-Sichtbarkeit fraktionsabhängig: Krone=subtil/Verfall, Gilden=offen/Werkzeug, Orden=versteckt/Geheimnis\\

\vspace{1mm}\noindent\textbf{Referenzmaterial}\par

\noindent\hspace{2mm}\textbullet\ Eigene Vormittags-Recherche: \texttt{traces/day01-scene2-emre/output.md}\\
\noindent\hspace{2mm}\textbullet\ Namis narrative Prinzipien: "Fraktionen als Sprachen", "Fremder als Zustand"\\
\noindent\hspace{2mm}\textbullet\ Briefing: \texttt{simulation-2-test/briefing.md}\\
\noindent\hspace{2mm}\textbullet\ Quellen: Poetische Edda (Havamal, Grimnismal, Völuspá), Prosa-Edda (Snorri)\\

\vspace{1mm}\noindent\textit{day01-scene5-emre/1-reasoning}\par
\vspace{2mm}\noindent\textbf{Reasoning — Emre Yilmaz, Tag 1, Szene 5 (WORK)}\par\vspace{0.5mm}

\vspace{1mm}\noindent\textbf{Denkprozess}\par

\noindent\textbf{\textit{1. Fraktionen: Von der Zuordnung zur Grundierung}} 

\noindent Heute Vormittag habe ich die erste Zuordnung gemacht: Orden=Odin, Krone=Tyr/Thor, Gilden=Vanir. Das war ein Anfang. Jetzt muss ich tiefer gehen — die CD hat entschieden, dass Biotech-Sichtbarkeit fraktionsabhängig ist:\par

\noindent\hspace{2mm}\textbullet\ Krone: subtil, Verfall. Das passt zu Tyr — die opfernde Hand, der verfallende Körper. Die Krone NUTZT Biotech nicht aktiv, sie LEIDET unter dessen Konsequenzen. Der Verfall der Ordnung manifestiert sich körperlich.\\
\noindent\hspace{2mm}\textbullet\ Gilden: offen, Werkzeug. Vanir-Pragmatismus. Die Gilden haben kein mythologisches Problem mit Biotech — es ist ein Werkzeug, wie alles andere. Handelsware. Investitionsobjekt.\\
\noindent\hspace{2mm}\textbullet\ Orden: versteckt, Geheimnis. Odin-Logik. Wissen ist Macht, also wird Biotech-Wissen kontrolliert, monopolisiert, vor der Öffentlichkeit verborgen. Der Orden WEISS mehr über das Schattenfieber als er zugibt.\\

\noindent Die CD will ausserdem, dass der Mythos in der Spielwelt umstritten ist. Das heisst: Jede Fraktion hat ihre EIGENE Version der Schöpfungsgeschichte, der Götter, des Schattenfiebers. Namis Prinzip "Fraktionen als Sprachen" greift hier perfekt — die Mythologie ist nicht objektiv, sondern ein Werkzeug der Selbstlegitimierung.\par

\noindent\textbf{\textit{2. Haltung zum Schattenfieber — drei Perspektiven}} 

\noindent Wenn der Mythos umstritten ist, muss auch die Deutung des Schattenfiebers fraktionsabhängig sein:\par

\noindent \textbf{Krone}: Das Schattenfieber als Strafe für den Verfall der alten Ordnung. Tyr hat seine Hand geopfert, um den Wolf zu binden — die Krone sieht sich als diejenige, die das Opfer gebracht hat. Das Fieber ist die Konsequenz davon, dass die anderen Fraktionen diese Opfertradition verraten haben. Schuldzuweisung nach aussen.\par

\noindent \textbf{Gilden}: Das Schattenfieber als Ressource. Pragmatisch, amoralisch. Die Vanir haben Krieg geführt und dann Frieden geschlossen, als es profitabler war. Die Gilden sehen das Fieber als ein Phänomen, das man verstehen, kontrollieren und kommerzialisieren kann. Nicht gut, nicht böse — nützlich oder nicht nützlich.\par

\noindent \textbf{Orden}: Das Schattenfieber als kosmologisches Geheimnis, das nur der Orden entschlüsseln darf. Odin hat sein Auge geopfert für Wissen — der Orden beansprucht das Recht, allein in den Abgrund zu blicken. Das Fieber ist für sie gleichzeitig Bedrohung und Bestätigung ihrer Daseinsberechtigung: Ohne den Orden wüsste niemand, was hier wirklich passiert.\par

\noindent\textbf{\textit{3. Haltung zum Relikt — drei Perspektiven}} 

\noindent Hypothese C ist genehmigt: Das Relikt hat die Daseinsebenen getrennt gehalten, seine Wirkung lässt nach, das Schattenfieber ist die Folge.\par

\noindent \textbf{Krone}: Das Relikt gehört der Krone, weil es ein Erbstück der alten Ordnung ist. Die Krone will es zurückholen und die Bindung erneuern — den Status quo wiederherstellen. Das ist konservativ, aber auch verzweifelt: Die Krone hat die Mittel nicht mehr, das Relikt tatsächlich zu kontrollieren.\par

\noindent \textbf{Gilden}: Das Relikt ist ein Handelsgegenstand, ein Machtfaktor. Die Gilden wollen es besitzen, nicht aus ideologischen Gründen, sondern weil derjenige, der das Relikt kontrolliert, den Markt kontrolliert. Wenn das Schattenfieber eine Ressource ist, dann ist das Relikt das Ventil.\par

\noindent \textbf{Orden}: Das Relikt ist heilig — oder zumindest muss es als heilig gelten, damit der Orden seine Autorität behält. Der Orden will es studieren, verstehen, und vor allem: sicherstellen, dass nur der Orden es versteht. Wissen als Monopol.\par

\noindent\textbf{\textit{4. Interne Konflikte — das macht die Fraktionen lebendig}} 

\noindent Keine Fraktion darf monolithisch sein. Das Briefing sagt: "keine ist gut oder böse". Also brauche ich interne Risse.\par

\noindent \textbf{Krone}: Traditionalisten vs. Reformer. Die einen wollen die alte Ordnung um jeden Preis (auch wenn die Kasse leer ist und das Volk leidet). Die anderen sehen, dass Anpassung nötig ist — aber wie weit? Tyr hat seine Hand geopfert — aber was, wenn das Opfer nicht mehr reicht?\par

\noindent \textbf{Gilden}: Monopolisten vs. Freie Händler. Die grossen Gilden wollen den Markt kontrollieren (inklusive Biotech, inklusive Schattenfieber-"Produkte"). Die kleineren Gilden werden erdrückt. Das ist der Vanir-Konflikt internalisiert: Der Geiselnaustausch mit den Asen war gut für die Vanir-Führung, aber was war mit den einzelnen Geiseln?\par

\noindent \textbf{Orden}: Wissenssuchende vs. Inquisitoren. Die einen wollen das Schattenfieber verstehen (Odin am Brunnen Mimirs — Wissen um jeden Preis). Die anderen wollen es unterdrücken (Odin als Krieger, der die Ordnung verteidigt). Beide berufen sich auf denselben Mythos, interpretieren ihn aber gegensätzlich.\par

\noindent\textbf{\textit{5. Das Wanderer-Motiv — Odin und der Spieler}} 

\noindent Das ist der Kern für die Nami-Diskussion. Im Briefing steht: "Der Spielercharakter ist bei jeder Iteration ein Unbekannter." Und: "Game Feel: Ich betrete als Fremder eine aufregende Sandbox."\par

\noindent In der germanischen Mythologie ist Odin DER Wanderer. Seine Beinamen sagen alles:\par

\noindent\hspace{2mm}\textbullet\ \textbf{Gangleri} — der Wandernde, der Wandermüde\\
\noindent\hspace{2mm}\textbullet\ \textbf{Grimnir} — der Vermummte, der Maskierte\\
\noindent\hspace{2mm}\textbullet\ \textbf{Vegtamr} — der Wegekundige\\
\noindent\hspace{2mm}\textbullet\ \textbf{Gestumblindi} — der blinde Gast (in der Hervarar saga)\\

\noindent Odin wandert als alter Mann in Kapuze und Mantel durch die Welt. Er gibt sich nicht zu erkennen. Er prüft die Menschen — halten sie das Gastrecht ein? Sind sie grosszügig oder geizig? Die Begegnung mit dem Wanderer ist immer eine Prüfung.\par

\noindent Das Havamal (Strophen 1-3) beginnt mit GASTRECHT-Regeln:\par
\noindent\hspace{2mm}\textbullet\ Der Gast braucht Feuer, Nahrung, trockene Kleider\\
\noindent\hspace{2mm}\textbullet\ Man soll den Fremden bewirten, bevor man ihn befragt\\
\noindent\hspace{2mm}\textbullet\ Wer reist, braucht Verstand und Vorsicht\\

\noindent Das ist perfekt für RELICS:\par
\noindent\hspace{2mm}\textbullet\ Der Spieler IST der Wanderer — nicht Odin, aber in derselben mythologischen Funktion\\
\noindent\hspace{2mm}\textbullet\ Jede Begegnung ist eine Prüfung — der Spieler wird von den Fraktionen danach beurteilt, wie er sich verhält\\
\noindent\hspace{2mm}\textbullet\ Das Gastrecht als narratives Motiv: Wer den Fremden aufnimmt, wer ihn abweist, wer ihn auszunutzen versucht\\
\noindent\hspace{2mm}\textbullet\ Der Spieler hat keine bekannte Identität — wie Odin in Verkleidung. Aber anders als Odin hat der Spieler TATSÄCHLICH keine Macht. Er muss sie erst erwerben.\\

\noindent\textbf{\textit{6. Mythologischer Umkehrpunkt: Spieler vs. Odin}} 

\noindent Hier wird es interessant für Nami: In der Edda ist Odin der mächtigste der Götter, der sich als Schwacher verkleidet. In RELICS ist der Spieler tatsächlich machtlos — kein Gott in Verkleidung, sondern ein Niemand, der sich Macht erarbeiten muss. Das ist die Umkehrung des Motivs:\par

\noindent\hspace{2mm}\textbullet\ Odin wandert, um zu PRÜFEN\\
\noindent\hspace{2mm}\textbullet\ Der Spieler wandert, um zu LERNEN\\

\noindent Aber die Welt behandelt den Spieler so, als könnte er beides sein. Paranoia der Fraktionen: Was, wenn dieser Fremde mehr ist, als er scheint? Was, wenn er ein Agent einer anderen Fraktion ist? Was, wenn er vom Schattenfieber gezeichnet ist?\par

\noindent Das Wanderer-Motiv verbindet sich mit Namis Prinzip "Fremder als Zustand, nicht als Lücke". Der Spieler ist nicht einfach ein unbeschriebenes Blatt (Game-Design-Convenience) — er ist ein mythologischer Archetypus. Der Wanderer/Gast ist in der germanischen Tradition eine Figur mit EIGENEN Rechten und Pflichten.\par

\noindent\textbf{\textit{7. Verbindung: Gastrecht und Fraktionen}} 

\noindent Jede Fraktion würde das Gastrecht anders interpretieren:\par

\noindent\hspace{2mm}\textbullet\ \textbf{Krone}: Gastrecht als feudale Pflicht. Der Herrscher muss den Fremden aufnehmen — aber der Fremde muss sich unterordnen. Hier gibt es eine direkte Verbindung zur Havamal-Tradition.\\
\noindent\hspace{2mm}\textbullet\ \textbf{Gilden}: Gastrecht als Transaktion. Du bekommst Unterkunft, ich bekomme Information/Arbeit/Geld. Pragmatisch.\\
\noindent\hspace{2mm}\textbullet\ \textbf{Orden}: Gastrecht als Inquisition. Der Fremde wird aufgenommen — und befragt. "Wir bewirten dich, aber sag uns, wer du bist und woher du kommst." Odin befragt den Gast, wie in der Vafþruðnismál.\\

\noindent Das gibt dem Spieler in jeder Fraktion ein anderes Einstiegserlebnis. Vera braucht das für die Architektur: Wie sieht der Ort aus, an dem der Fremde empfangen wird? Thronsaal (Krone), Kontor (Gilden), Verhörrraum (Orden)?\par

\noindent\textbf{\textit{8. Offene Fragen für Nami}} 

\noindent\hspace{2mm}\textbullet\ Wie verhält sich das Wanderer-Motiv zum Schattenfieber? Wenn der Spieler sich infizieren kann (CD-Bestätigung aus dem Standup), ändert sich sein Status: Vom Wanderer zum Gezeichneten. Ähnlich wie Odin, der sein Auge opfert — der Spieler opfert seine Normalität.\\
\noindent\hspace{2mm}\textbullet\ Die Völva-Figur als NPC: In der Völuspá erzählt die Seherin Odin die Wahrheit. Könnte es eine Völva-Figur geben, die dem Spieler Wahrheiten sagt, die die Fraktionen verschweigen?\\
\noindent\hspace{2mm}\textbullet\ "Mythos ist umstritten" — wie erzählen wir das konkret? Bücher? NPCs mit widersprüchlichen Geschichten? Environmental Storytelling?\\

\vspace{1mm}\noindent\textit{day01-scene5-finn/0-prompt}\par
\vspace{2mm}\noindent\textbf{Trace: Finn Bergmann — Tag 01, Szene 5 (WORK) — Prompt}\par\vspace{0.5mm}

\vspace{1mm}\noindent\textbf{Kontext}\par

\noindent Tag 1 (Montag), 14:00 Uhr, Produktionsbüro. Nachmittags-WORK-Block.\par

\vspace{1mm}\noindent\textbf{Aufgabenstellung}\par

\noindent\hspace{2mm}1.\ Roadmap verfeinern basierend auf den CD-Entscheidungen vom Standup (Szene 3)\\
\noindent\hspace{2mm}2.\ Scope-Gespräche mit Darius (Schattenfieber-System) und Tobi (Pipeline-Architektur) führen\\
\noindent\hspace{2mm}3.\ Offene Punkte und Risiken für die CD notieren\\

\vspace{1mm}\noindent\textbf{Verfügbare Informationen}\par

\noindent\hspace{2mm}\textbullet\ Meine Memory (Tag 1, Szenen 1-3)\\
\noindent\hspace{2mm}\textbullet\ Roadmap V1 (von heute Vormittag, Szene 2)\\
\noindent\hspace{2mm}\textbullet\ Logbook Szene 3 (Standup mit 8 CD-Entscheidungen)\\
\noindent\hspace{2mm}\textbullet\ Darius' Referenzspiel-Analyse V0 (Szene 2 Output)\\
\noindent\hspace{2mm}\textbullet\ Tobis Machbarkeitsanalyse V0.1 (Szene 2 Output)\\
\noindent\hspace{2mm}\textbullet\ Briefing (RELICS)\\

\vspace{1mm}\noindent\textbf{CD-Entscheidungen (8 Stück, bestätigt in Szene 3)}\par

\noindent\hspace{2mm}1.\ Schattenfieber = Hypothese C (kosmologische Erosion)\\
\noindent\hspace{2mm}2.\ Nordische Quellen akzeptiert, eigene Kosmologie\\
\noindent\hspace{2mm}3.\ Mythos innerhalb der Spielwelt umstritten\\
\noindent\hspace{2mm}4.\ Biotech-Sichtbarkeit fraktionsabhängig\\
\noindent\hspace{2mm}5.\ Ü5 bestätigt\\
\noindent\hspace{2mm}6.\ Semi-Open-World bestätigt\\
\noindent\hspace{2mm}7.\ TP Primärmodus, FP ist V2\\
\noindent\hspace{2mm}8.\ Spieler-Infektion mit Schattenfieber bestätigt\\

\vspace{1mm}\noindent\textit{day01-scene5-finn/1-reasoning}\par
\vspace{2mm}\noindent\textbf{Trace: Finn Bergmann — Tag 01, Szene 5 (WORK) — Reasoning}\par\vspace{0.5mm}

\vspace{1mm}\noindent\textbf{Was hat sich seit heute Vormittag verändert?}\par

\noindent Heute Vormittag war die Roadmap V1 noch auf Basis von offenen Fragen gebaut. Jetzt sind 6 von 8 Fragen beantwortet. Die Roadmap muss das widerspiegeln — nicht als grundsätzliche Überarbeitung, sondern als Schärfung.\par

\vspace{1mm}\noindent\textbf{Kernüberlegungen}\par

\noindent\textbf{\textit{1. Schattenfieber ist jetzt ein Core-System}} 

\noindent Das ist die grösste Veränderung. Heute Vormittag war das Schattenfieber ein Lore-Element in Emres Zuständigkeitsbereich. Durch CD-Entscheidung \#8 (Spieler kann sich infizieren, Kosten + Nutzen, Mechanik + Narrativ verbunden) wird es zum zentralen Gameplay-Loop, der drei Leute gleichzeitig betrifft:\par

\noindent\hspace{2mm}\textbullet\ Emre: Kosmologische Lore (WBB-01), wie funktioniert es in der Welt?\\
\noindent\hspace{2mm}\textbullet\ Darius: Mechanik-Design (GDD-02), was passiert gameplay-seitig?\\
\noindent\hspace{2mm}\textbullet\ Nami: Narrativ-Design (GDD-03), wie verändert es Dialoge, NPC-Reaktionen?\\

\noindent Das braucht Koordination. Ich setze einen Synchronisationspunkt für Freitag: erster gemeinsamer Konzept-Check. Kein eigenes Meeting, sondern Teil des Weekly Reviews.\par

\noindent\textbf{\textit{2. Semi-Open-World entschärft das grösste Scope-Risiko}} 

\noindent Tobi hat es richtig eingeschätzt: Open-World vs. Semi-Open ist der Unterschied zwischen "unmöglich" und "ambitioniert aber machbar". Gothic-Dichte auf 4-6 km2 statt Skyrim-Breite auf 60 km2. Das reduziert Asset-Bedarf, Terrain-Arbeit, QA-Aufwand. Semi-OW war auch mein Instinkt — gut, dass die CD das bestätigt hat.\par

\noindent\textbf{\textit{3. TP Primär spart uns enorm viel Arbeit}} 

\noindent Tobis Einschätzung: 70\% weniger Animationsarbeit. Das ist kein kleines Delta. FP als V2 oder DLC ist der richtige Ansatz. Kein Blocker, kein Risiko.\par

\noindent\textbf{\textit{4. Zwei neue kritische Risiken von Tobi}} 

\noindent Kein Gameplay-Programmer und Asset-Produktion zu langsam. Das muss in den CD-Report. Nicht als Panik, sondern als ehrliche Einschätzung. Tobi hat Mitigationsstrategien (Middleware, Blueprints, Houdini-Pipeline, Marketplace), aber die CD sollte wissen, wo die Engpässe liegen.\par

\noindent\textbf{\textit{5. Was ist noch offen?}} 

\noindent Fünf Fragen für die CD:\par
\noindent\hspace{2mm}\textbullet\ Combat-Skill-Ceiling (Darius formuliert Dienstag einen Vorschlag)\\
\noindent\hspace{2mm}\textbullet\ Zielplattform (PC-only oder Konsolen?)\\
\noindent\hspace{2mm}\textbullet\ Marketplace-Budget\\
\noindent\hspace{2mm}\textbullet\ Vertical-Slice-Umfang\\
\noindent\hspace{2mm}\textbullet\ SIGGRAPH-Paper für Tobi\\

\noindent Keine davon ist ein akuter Blocker. Alle können bis Mitte der Woche warten.\par

\noindent\textbf{\textit{6. Teamdynamik}} 

\noindent Guter erster Tag. Alle haben geliefert. Emre hat nicht überrecherchiert, sondern ein klares Modell gebaut. Darius hat eine präzise Analyse gemacht. Tobi war ehrlich und konkret. Vera hat systematisch gearbeitet. Nami hat starke Prinzipien formuliert. Leo hat die Marktluecke identifiziert und den besten Satz des Tages geliefert.\par

\noindent Kein Micromanagement nötig. Ich halte die Wege frei, sortiere die Informationen, und sorge dafür, dass die richtigen Leute miteinander reden.\par

\vspace{1mm}\noindent\textit{day01-scene5-leo/0-prompt}\par
\vspace{2mm}\noindent\textbf{Trace: Day 01, Scene 5 — Prompt (Leo Fischer)}\par\vspace{0.5mm}

\vspace{1mm}\noindent\textbf{Szene}\par
\noindent\hspace{2mm}\textbullet\ \textbf{Tag}: 1 (Montag)\\
\noindent\hspace{2mm}\textbullet\ \textbf{Szene}: 5 (WORK, 14:00 Uhr)\\
\noindent\hspace{2mm}\textbullet\ \textbf{Ort}: QA Station\\
\noindent\hspace{2mm}\textbullet\ \textbf{Typ}: Nachmittags-Recherche\\

\vspace{1mm}\noindent\textbf{Kontext}\par
\noindent Im Standup (Szene 3) wurde bestätigt: Der Spieler kann sich mit dem Schattenfieber infizieren. Das ist der potenzielle USP für jeden Trailer. Mein Nachmittags-Auftrag: Community-Erwartungen an Infektions-/Transformationsmechaniken vertiefen.\par

\vspace{1mm}\noindent\textbf{Aufgabe}\par
\noindent\hspace{2mm}1.\ Community-Reaktionen auf Infektions-/Transformationsmechaniken in bestehenden Spielen analysieren: Vampirismus in Skyrim (Dawnguard/Volkihar), Hollowing in Dark Souls, Beast-Transformation in Bloodborne, Corruption in diversen RPGs.\\
\noindent\hspace{2mm}2.\ Was lieben Spieler? Was hassen sie? Wo sind die Dealbreaker?\\
\noindent\hspace{2mm}3.\ Daraus Community-Erwartungen an das Schattenfieber als Spielermechanik formulieren.\\

\vspace{1mm}\noindent\textbf{Quellen}\par
\noindent\hspace{2mm}\textbullet\ Eigene Marktanalyse vom Vormittag (Szene 2)\\
\noindent\hspace{2mm}\textbullet\ Meeting-Ergebnisse (Szene 3): Schattenfieber als Core System bestätigt, Kosten/Nutzen\\
\noindent\hspace{2mm}\textbullet\ Community-Beobachtung: Reddit, Steam-Reviews, YouTube-Kommentare\\
\noindent\hspace{2mm}\textbullet\ Eigene Erfahrung als Spielerin und Content Creator\\

\vspace{1mm}\noindent\textbf{Briefing-Alignment}\par
\noindent\hspace{2mm}\textbullet\ Schattenfieber ist das EINZIGE übernatürliche Element (kein Magie-System)\\
\noindent\hspace{2mm}\textbullet\ Kosten und Nutzen müssen verbunden sein (CD-Entscheidung)\\
\noindent\hspace{2mm}\textbullet\ Mechanik und Narrativ sind verknüpft\\
\noindent\hspace{2mm}\textbullet\ Kein High-Fantasy, kein "magische Krankheit mit Zauberkräften"\\

\vspace{1mm}\noindent\textit{day01-scene5-leo/1-reasoning}\par
\vspace{2mm}\noindent\textbf{Trace: Day 01, Scene 5 — Reasoning (Leo Fischer)}\par\vspace{0.5mm}

\vspace{1mm}\noindent\textbf{Denkprozess}\par

\noindent\textbf{\textit{Warum diese Analyse?}} 
\noindent Darius baut gerade das Kosten-Nutzen-System für Schattenfieber. Aber Darius denkt als Designer. Ich muss als SPIELERIN denken. Was will die Community? Was toleriert sie? Was bringt den Shitstorm? Wenn ich das heute nicht liefere und Darius in eine Richtung designed, die die Community ablehnt, verlieren wir Wochen.\par

\noindent\textbf{\textit{Auswahl der Referenz-Spiele}} 
\noindent Ich nehme die vier grossen Infektions-/Transformationsmechaniken, die die Community am meisten diskutiert hat:\par

\noindent\hspace{2mm}1.\ \textbf{Skyrim Vampirismus (Volkihar/Dawnguard)} — Weil Skyrim unsere Kamera-Referenz ist UND der direkteste Vergleichspunkt. Vampirismus ist die bekannteste Infektionsmechanik in Open-World RPGs. Millionen Spieler haben damit Erfahrung.\\

\noindent\hspace{2mm}2.\ \textbf{Dark Souls Hollowing} — Weil FromSoft die "Infektion als Bestrafung" Mechanik populär gemacht hat. Hollowing ist eine Strafe fürs Sterben. Die Community hat STARKE Meinungen dazu — und die haben sich von Dark Souls 1 zu 3 massiv verschoben.\\

\noindent\hspace{2mm}3.\ \textbf{Bloodborne Beast-Transformation (Insight/Beasthood)} — Weil Bloodborne die SUBTILSTE Version macht. Insight verändert die Welt ohne dass der Spieler es merkt. Das ist am nächsten an dem, was Nami mit "veränderter Wahrnehmung" beschrieben hat.\\

\noindent\hspace{2mm}4.\ \textbf{Diverse Corruption-Mechaniken} — Vor allem Dragon Age Inquisition (Lyrium), Darkest Dungeon (Stress/Afflictions), Disco Elysium (Thought Cabinet als "mentale Infektion"). Breiter Blick auf Risiko-Reward-Systeme.\\

\noindent Ich lasse bewusst WoW Corrupted Gear und Path of Exile Corruption weg — das sind MMO/ARPG-Systeme mit komplett anderer Design-Philosophie und Zielgruppe.\par

\noindent\textbf{\textit{Methodischer Ansatz}} 
\noindent Ich schaue mir vier Dimensionen an:\par
\noindent\hspace{2mm}\textbullet\ \textbf{Opt-in vs. Opt-out}: Kann der Spieler die Infektion vermeiden? Will er das?\\
\noindent\hspace{2mm}\textbullet\ \textbf{Reversibilität}: Kann man die Infektion rückgängig machen?\\
\noindent\hspace{2mm}\textbullet\ \textbf{Visuelle/narrative Konsequenz}: Sieht man die Infektion? Reagieren NPCs?\\
\noindent\hspace{2mm}\textbullet\ \textbf{Gameplay-Impact}: Stärker ODER schwächerer Charakter?\\

\noindent Das sind die vier Achsen, an denen Community-Sentiment kippt.\par

\noindent\textbf{\textit{Hypothese}} 
\noindent Meine These vor der Analyse: Spieler LIEBEN Transformation, wenn sie sich mächtig fühlen. Spieler HASSEN Transformation, wenn sie sich bestraft fühlen. Der Trick ist: Die Kosten müssen sich wie ein bewusster Preis anfühlen, nicht wie eine Strafe. "Ich habe mich ENTSCHIEDEN, stärker zu werden — und dafür etwas aufgegeben" vs. "Das Spiel HAT MIR etwas weggenommen."\par

\noindent Die Frage ist: Wie macht man eine KRANKHEIT, die sich wie eine Entscheidung anfühlt?\par

\vspace{1mm}\noindent\textit{day01-scene5-nami/0-prompt}\par
\vspace{2mm}\noindent\textbf{Prompt — Nami Okafor, Tag 1, Szene 5 (WORK)}\par\vspace{0.5mm}

\vspace{1mm}\noindent\textbf{Kontext}\par
\noindent Tag 1, Montag. Nachmittags-Recherche, 14:00, Bibliothek.\par
\noindent Bilateral mit Emre: Wanderer-Motiv in der germanischen Mythologie.\par

\vspace{1mm}\noindent\textbf{Aufgaben}\par
\noindent\hspace{2mm}1.\ Das Wanderer-Motiv narrativ ausarbeiten: Wie nutzt die germanische Mythologie den Fremden/Gast? (Havamal — Gastrecht, Odin als Wanderer in Verkleidung, der Fremde der Weisheit sucht.) Wie übersetzen wir das in RELICS?\\
\noindent\hspace{2mm}2.\ Das Schattenfieber als Unreliable-Narrator-Mechanismus vertiefen: Wenn der Spieler infiziert ist, wie verändert sich die narrative Wahrnehmung? Welche Storytelling-Möglichkeiten eröffnet das?\\

\vspace{1mm}\noindent\textbf{Inputs}\par
\noindent\hspace{2mm}\textbullet\ Eigene Vormittagsarbeit: Vier narrative Prinzipien (Fremder als Zustand, Fraktionen als Sprachen, Räume erzählen, Schattenfieber als Stoerfaktor)\\
\noindent\hspace{2mm}\textbullet\ Emres Recherche: Drei-Schichten-Modell, Hypothese C (kosmologische Erosion + gebrochene Bindung), Fraktionsresonanzen\\
\noindent\hspace{2mm}\textbullet\ CD-Bestätigung aus Szene 3: Spieler kann sich mit Schattenfieber infizieren. Kosten und Nutzen, Mechanik und Narrativ verbunden.\\
\noindent\hspace{2mm}\textbullet\ CD-Bestätigung: Mythos soll in der Spielwelt umstritten sein — verschiedene Wahrheiten je nach Fraktion.\\

\vspace{1mm}\noindent\textbf{Leitfragen}\par
\noindent\hspace{2mm}\textbullet\ Was sagt das Havamal über den Fremden an der Tür?\\
\noindent\hspace{2mm}\textbullet\ Wie verbinde ich Odins Wanderschaft mit dem Player-Character-Constraint?\\
\noindent\hspace{2mm}\textbullet\ Wo genau kippt die Wahrnehmung des Spielers durch Schattenfieber — und wie erzähle ich das, ohne es zu spoilern?\\

\vspace{1mm}\noindent\textit{day01-scene5-nami/1-reasoning}\par
\vspace{2mm}\noindent\textbf{Reasoning — Nami Okafor, Tag 1, Szene 5 (WORK)}\par\vspace{0.5mm}

\vspace{1mm}\noindent\textbf{Denkprozess}\par

\noindent\textbf{\textit{Teil 1: Der Wanderer — Gastrecht, Göttliche Verkleidung, das Rätsel an der Schwelle}} 

\noindent Ich sitze in der Bibliothek, Emres Recherche aufgeschlagen, und suche den Faden, der sich von der Edda in unser Spiel spinnen lässt.\par

\noindent Das Havamal — Odins Spruchweisheiten — beginnt mit einer Passage, die mich seit dem Studium nicht loslässt. Die ersten Strophen handeln nicht von Göttern oder Schlachten. Sie handeln davon, wie man einen Fremden behandelt, der an die Tür klopft.\par

\noindent Strophe 1: "Alle Türeingänge, bevor man eintritt, soll man überblicken, soll man erspähen; denn ungewiss ist zu wissen, wo Feinde sitzen in der Halle voraus."\par

\noindent Das ist nicht Höflichkeit. Das ist Überlebensstrategie. Der Fremde, der eintritt, muss alles lesen — Räume, Gesichter, Schweigen. Er ist verletzlich. Und genau DESHALB ist er gefährlich: Wer nichts zu verlieren hat, ist unberechenbar.\par

\noindent Strophe 2-3: Dem Gast soll man Wärme geben, Wasser, ein Tuch. Gastfreundschaft ist heilig — nicht weil der Gast es verdient, sondern weil der Gast Odin selbst sein könnte.\par

\noindent Und DA liegt der Kern für RELICS.\par

\noindent In der germanischen Mythologie ist der Fremde immer zweideutig. Er ist entweder ein Bettler oder ein Gott. Ein Hilfesuchender oder ein Prüfer. Die Schwelle der Tür ist der Ort, an dem diese Ambiguität lebt. Und die Frage, die jeder NPC dem Spieler stellt — bewusst oder unbewusst — ist: "Bist du ein Niemand, oder bist du etwas, das ich noch nicht erkenne?"\par

\noindent Das ist GENAU der Zustand, den das Briefing beschreibt. Der Spieler IST diese Ambiguität. Er ist der Wanderer, von dem man nicht weiss, ob er ein Gott oder ein Bettler ist.\par

\noindent Odin wandert als "Grimnir" (der Verhüllte), als "Gangleri" (der Wandermüde), als alter Mann mit Hut und Stab. Er prüft die Gastfreundschaft der Menschen. Wer ihn aufnimmt, wird belohnt. Wer ihn abweist, wird bestraft. Aber — und das ist der entscheidende Punkt — die Menschen WISSEN NICHT, wen sie vor sich haben.\par

\noindent Für RELICS heisst das: Die NPCs müssen den Spieler behandeln, als könnte er JEDER sein. Nicht mit Ehrfurcht — mit Unsicherheit. Manche werden ihn aufnehmen (Gastrecht, Opportunismus, echte Güte). Manche werden ihn abweisen (Angst, Xenophobie, politisches Kalkül). Aber alle müssen reagieren auf die Tatsache, dass sie nicht wissen, wer da vor ihnen steht.\par

\noindent Emre hat heute Morgen geschrieben, dass die Mythen in der Spielwelt umstritten sein sollen — verschiedene Fraktionen, verschiedene Versionen. Das lässt sich hier wunderschön einflechten:\par

\noindent\hspace{2mm}\textbullet\ \textbf{Die Krone} sieht im Wanderer eine Bedrohung. Ein Fremder ohne Linie, ohne Loyalität, ohne Eid — das ist für eine feudale Ordnung eine Anomalie. Sie werden den Spieler tolerieren, aber überwachen.\\
\noindent\hspace{2mm}\textbullet\ \textbf{Die Gilden} sehen im Wanderer eine Gelegenheit. Jemand ohne Bindungen ist billig, deniable, einsetzbar. Sie werden den Spieler benutzen — und es Partnerschaft nennen.\\
\noindent\hspace{2mm}\textbullet\ \textbf{Der Orden} sieht im Wanderer ein Zeichen. In ihrer Version des Mythos ist der Wanderer eine Gestalt, die in Zeiten der Prüfung erscheint. Sie werden den Spieler prufen — und je nach Ergebnis willkommen heissen oder jagen.\\

\noindent Drei Fraktionen. Drei Interpretationen desselben mythologischen Motivs. Keine hat "recht." Alle haben Gründe.\par

\noindent Das ist Fraktionen als Sprachen, angewandt auf den Mythos selbst.\par

\noindent\textbf{\textit{Teil 2: Das Schattenfieber als Unreliable Narrator}} 

\noindent CD hat bestätig: Der Spieler kann sich infizieren. Das ist die grösste narrative Chance, die wir haben.\par

\noindent Ich denke wieder an meine Thesis. Der Unreliable Narrator funktioniert in der Literatur, weil der Leser dem Erzähler vertraut und dieses Vertrauen dann gebrochen wird. In Spielen ist das schwieriger, weil der Spieler selbst der Erzähler ist — er MACHT die Geschichte.\par

\noindent Aber was, wenn das Interface selbst unzuverlässig wird?\par

\noindent Nicht als Gimmick. Nicht als Jump-Scare. Sondern als langsame, schleichende Erosion der Gewissheit. Der Spieler sieht eine Dialogoption, die andere Spieler nicht sehen. Oder: Ein NPC sagt etwas, und beim nächsten Gespräch streitet er ab, es je gesagt zu haben. Oder: Der Spieler betritt einen Raum, und für einen Moment — einen Herzschlag — sieht er eine andere Architektur, andere Wände, ein anderes Licht. Dann ist es weg.\par

\noindent Die Frage, die das Schattenfieber stellen muss, ist nicht "Bist du krank?" Sondern: "Siehst du mehr als andere — oder weniger?"\par

\noindent Emres Hypothese C liefert den kosmologischen Unterbau: Die Daseinsebenen werden durchlässig. Das Schattenfieber bringt den Spieler in Kontakt mit der anderen Schicht. Was er sieht, ist vielleicht real — nur nicht in DIESER Schicht.\par

\noindent Das gibt mir drei narrative Stufen:\par

\noindent \textbf{Stufe 1 — Rauschen}: Subtile Verschiebungen. Ein NPC, der einen Satz sagt, den der Spieler schon einmal gehört hat — aber von jemand anderem. Ein Schatten, der sich einen Moment zu spät bewegt. Kein Hinweis darauf, dass etwas "falsch" ist. Der Spieler merkt es vielleicht nicht einmal. (Disco Elysium macht das: manche Skill-Checks passieren im Hintergrund. Du siehst nur das Ergebnis, nicht den Wurf.)\par

\noindent \textbf{Stufe 2 — Risse}: Der Spieler sieht Dinge, die nicht da sein sollten. Gespräche, die niemand geführt hat. Türen, die in der nächsten Szene zugemauert sind. NPCs, die den Spieler an jemanden erinnern, der noch gar nicht aufgetaucht ist. Hier beginnt der Spieler zu zweifeln — aber er weiss nicht, ob ER sich verändert oder die WELT. (Das Malkavian-Prinzip: Die Wahrnehmung verändert die Realität des Spielers, nicht die Realität selbst.)\par

\noindent \textbf{Stufe 3 — Schwelle}: Die Grenze zwischen den Schichten wird für den Spieler durchlässig. Er sieht die untere Schicht — die Stille, das Vergessen — als überlagerte Realität. NPCs, die im Schattenfieber-Zustand sichtbar werden, die niemand sonst sehen kann. Orte, die in zwei Zuständen gleichzeitig existieren. Hier wird der Spieler zum Wanderer zwischen den Welten — im wörtlichen Sinne. (Kentucky Route Zero macht das: Die Realität verschiebt sich, und das Spiel sagt dir nicht, wo die Grenze ist.)\par

\noindent Und jetzt das Entscheidende: Stufe 3 könnte dem Spieler Zugang zu Wissen geben, das sonst verschlossen wäre. Genau wie Odin sein Auge opferte für Wissen am Brunnen Mimirs. Das Schattenfieber als OPFER FÜR ERKENNTNIS. Der Spieler kann Dinge sehen, die andere nicht sehen — Wahrheiten, die unter der Oberfläche liegen. Aber er zahlt dafür mit seiner Zuverlässigkeit als Erzähler seiner eigenen Geschichte.\par

\noindent Das ist die ultimative Unreliable-Narrator-Mechanik: Du siehst mehr. Aber du kannst nicht mehr sicher sein, ob das, was du siehst, real ist. Und die anderen Charaktere können es dir nicht sagen, weil sie es nicht sehen.\par

\noindent Kosten und Nutzen. Mechanik und Narrativ. Genau das, was die CD gefordert hat.\par

\noindent\textbf{\textit{Selbstkritik}} 

\noindent Wo bin ich unsicher:\par
\noindent\hspace{2mm}\textbullet\ Meine Havamal-Kenntnisse stammen aus dem Studium, ich habe keinen Text vor mir liegen. Die Strophen-Inhalte stimmen in der Substanz, aber die exakte Nummerierung und Wortwahl müsste ich gegen eine Übersetzung prüfen. Ich schreibe das im Output offen.\\
\noindent\hspace{2mm}\textbullet\ Stufe 3 des Schattenfiebers ist ambitioniert. Das ist ein Feature, das Development-Zeit kostet. Darius wird Fragen stellen. Ich sollte eine abgespeckte Version skizzieren, die denselben narrativen Effekt mit weniger Assets erreicht.\\
\noindent\hspace{2mm}\textbullet\ Die Verbindung Wanderer-Motiv / Schattenfieber / Odin-Opfer ist elegant, aber ich muss aufpassen, dass sie nicht ZU elegant wird. Echte Mythologie ist widersprüchlich und unordentlich. Wenn alles zu sauber zusammenpasst, wirkt es konstruiert, nicht lebendig.\\

\vspace{1mm}\noindent\textit{day01-scene5-tobi/0-prompt}\par
\vspace{2mm}\noindent\textbf{Prompt — Tobi, Tag 1, Szene 5 (WORK)}\par\vspace{0.5mm}

\vspace{1mm}\noindent\textbf{Kontext}\par
\noindent Tag 1, Montag, 14:00 Uhr, Tech Corner. Nachmittags-Recherche.\par

\vspace{1mm}\noindent\textbf{Auftrag}\par
\noindent Pipeline-Architektur für RELICS in Ü5 skizzieren. Bestätigt im Meeting (Szene 3):\par
\noindent\hspace{2mm}\textbullet\ Engine: Unreal Engine 5\\
\noindent\hspace{2mm}\textbullet\ Scope: Semi-Open-World (4-6 km², Gothic-Dichte)\\
\noindent\hspace{2mm}\textbullet\ Kamera: Third Person Primärmodus mit Zoom, FP ist V2\\
\noindent\hspace{2mm}\textbullet\ Combat: Real-time Action, Melee-fokussiert\\
\noindent\hspace{2mm}\textbullet\ Biotech-Futurismus, kein Steampunk\\
\noindent\hspace{2mm}\textbullet\ Nanite, Lumen, World Partition als technische Basis\\

\vspace{1mm}\noindent\textbf{Eingangsdaten}\par
\noindent\hspace{2mm}\textbullet\ Briefing (RELICS Creative Director's Briefing)\\
\noindent\hspace{2mm}\textbullet\ Eigene Machbarkeitsanalyse V0.1 (Szene 2)\\
\noindent\hspace{2mm}\textbullet\ Meeting-Ergebnisse und CD-Entscheidungen (Szene 3)\\
\noindent\hspace{2mm}\textbullet\ Veras visuelle Referenz-Analyse (Architektur, Mode, Biotech-Grammatik, Farbpalette)\\

\vspace{1mm}\noindent\textbf{Erwarteter Output}\par
\noindent Technische Pipeline-Architektur V0.1 mit:\par
\noindent\hspace{2mm}1.\ Asset-Pipeline (Nanite, modulare Architektur, Houdini-Terrain, PCG-Vegetation)\\
\noindent\hspace{2mm}2.\ Rendering-Pipeline (Lumen GI, Virtual Shadow Maps, Post-Processing)\\
\noindent\hspace{2mm}3.\ Kamerasystem (TP mit Zoom, Spring Arm, Collision)\\
\noindent\hspace{2mm}4.\ Middleware/Plugin-Empfehlungen (Combat, Animation, Dialog)\\
\noindent\hspace{2mm}5.\ Offene Fragen und Risiken\\

\vspace{1mm}\noindent\textit{day01-scene5-tobi/1-reasoning}\par
\vspace{2mm}\noindent\textbf{Reasoning — Tobi, Tag 1, Szene 5 (WORK)}\par\vspace{0.5mm}

\vspace{1mm}\noindent\textbf{Denkprozess}\par

\noindent\textbf{\textit{1. Rahmenbedingungen ordnen}} 

\noindent Die drei kritischen Engpässe aus meiner Machbarkeitsanalyse sind jetzt teilweise entschärft:\par
\noindent\hspace{2mm}\textbullet\ Scope: Semi-Open-World bestätigt. 4-6 km² ist realistisch für ein kleines Team.\\
\noindent\hspace{2mm}\textbullet\ Kamera: TP Primärmodus bestätigt. Kein doppelter Animationsaufwand.\\
\noindent\hspace{2mm}\textbullet\ Gameplay-Programmer: Immer noch offen. Brauche Middleware-Empfehlung.\\
\noindent\hspace{2mm}\textbullet\ Asset-Produktion: Immer noch kritisch. Vera allein reicht nicht. Modulares Kit + prozedurale Systeme.\\

\noindent\textbf{\textit{2. Asset-Pipeline-Überlegungen}} 

\noindent Nanite ist gesetzt. Aber Nanite hat Einschränkungen, die ich dokumentieren muss:\par
\noindent\hspace{2mm}\textbullet\ Kein Vertex-Animation-Support (relevant für organische Biotech-Materialien)\\
\noindent\hspace{2mm}\textbullet\ Kein Masked-Material-Support (Alpha-tested Foliage)\\
\noindent\hspace{2mm}\textbullet\ Vegetation muss hybrides Setup bekommen: Bäume als Nanite, Gras/Unterholz als traditionelle ISM\\

\noindent Modulare Architektur ist Pflicht bei einem Team dieser Grösse. Gothic-Stil-Architektur eignet sich gut dafür: Wiederverwendbare Mauer-, Bogen- und Turm-Module. Die drei fraktionsspezifischen Biotech-Sprachen (Krone=Zerfall, Gilden=industriell, Orden=monolithisch) müssen als Material-Layer funktionieren, nicht als separate Asset-Sets. Sonst explodiert die Produktion.\par

\noindent Houdini-Terrain: World Partition + Houdini Engine for Ü liefert prozedurales Heightfield + Scatter. Bei 4-6 km² ist ein einzelnes Landscape ausreichend, kein OFPA nötig. Aber World Partition für Streaming trotzdem sinnvoll.\par

\noindent\textbf{\textit{3. Rendering-Überlegungen}} 

\noindent Lumen ist die logische Wahl für die angestrebte Beleuchtungsqualität. Control und Alan Wake 2 sind meine Referenzen — beides Ü-Titel mit starkem atmosphärischem Lighting.\par

\noindent Lumen-Modus: Hardware Ray Tracing für Shipping, Software Lumen als Fallback. Die Gothic-Architektur mit engen Innenräumen und dramatischen Lichtschächten profitiert enorm von Lumen GI.\par

\noindent Virtual Shadow Maps sind Standard bei Lumen-Projekten. Kein Grund, davon abzuweichen.\par

\noindent Post-Processing: Hier ist die Briefing-Tonalität entscheidend. "Düster, geerdet, politisch. Gotische Grandeur trifft feudale Brutalität." Das ist kein Bloom-Festival. Ich denke an:\par
\noindent\hspace{2mm}\textbullet\ Kontrollierte Color Grading (Veras Farbpalette als Basis)\\
\noindent\hspace{2mm}\textbullet\ Subtile Vignette\\
\noindent\hspace{2mm}\textbullet\ Film Grain optional\\
\noindent\hspace{2mm}\textbullet\ Schattenfieber-Post-Processing als eigener Layer (Wahrnehmungsverzerrung bei Infektion)\\

\noindent\textbf{\textit{4. Kamerasystem-Überlegungen}} 

\noindent TP mit Zoom ist technisch ein Spring Arm mit variabler Länge. Collision-Handling über Sphere Trace. Zoom-Stufen: Exploration (weiter), Combat (näher), Conversation (Schulterblick oder leichter Zoom).\par

\noindent Wichtig: Die Kamera muss die Architektur-Vertikalität (Dishonored-Referenz) unterstützen. Pitch-Range erweitern, eventuell dynamische FOV-Anpassung bei Höhenunterschieden.\par

\noindent Kein echtes FP in V1, aber der Zoom sollte nah genug rangehen, dass es sich fast wie FP anfühlt. Smooth Interpolation zwischen Zoom-Stufen.\par

\noindent\textbf{\textit{5. Middleware-Überlegungen}} 

\noindent Combat: Ohne Gameplay-Programmer brauchen wir ein Framework. Optionen:\par
\noindent\hspace{2mm}\textbullet\ GAS (Gameplay Ability System) — Ü5-nativ, mächtig, aber steile Lernkurve\\
\noindent\hspace{2mm}\textbullet\ Marketplace Combat Frameworks — Risiko: Qualität, Wartung, Vendor Lock-in\\
\noindent\hspace{2mm}\textbullet\ Custom auf GAS-Basis — meine Empfehlung, aber braucht Programmier-Kompetenz\\

\noindent Animation: Ü5 Control Rig + IK ist inzwischen ausgereift. Motion Matching wäre ideal für das gewichtete Movement-Feel, ist aber aufwändig.\par

\noindent Dialog: Empfehlung bleibt offen — muss ich recherchieren. Yarn Spinner (Unity-native, Ü-Port existiert) oder ein eigenes System auf Data Tables.\par

\noindent\textbf{\textit{6. Offene Fragen}} 

\noindent\hspace{2mm}\textbullet\ Gameplay-Programmer ist der grösste Blocker. Ohne den bleibt Combat-Middleware theoretisch.\\
\noindent\hspace{2mm}\textbullet\ Art-Pipeline-Standards müssen mit Vera abgestimmt werden (Texel Density, LOD-Stufen, Naming).\\
\noindent\hspace{2mm}\textbullet\ Performance-Budget für Schattenfieber-VFX ist unklar. Brauche Emres Beschreibung der visuellen Auswirkungen.\\
\noindent\hspace{2mm}\textbullet\ Netzwerk/Multiplayer: Im Briefing nicht erwähnt, gehe von Singleplayer aus. Muss bestätigt werden.\\

\noindent\textbf{\textit{7. Was ich NICHT weiss}} 

\noindent\hspace{2mm}\textbullet\ Zielplattform(en) jenseits von PC. Konsolen-Support ändert Performance-Budget massiv.\\
\noindent\hspace{2mm}\textbullet\ Audio-Pipeline. Nicht mein Bereich, aber MetaSounds wäre der logische Ü5-Weg.\\
\noindent\hspace{2mm}\textbullet\ UI-Framework. CommonUI oder eigenes Widget-System?\\
\noindent\hspace{2mm}\textbullet\ Ob Vera bereits mit Houdini-Erfahrung kommt oder ob ich sie einarbeiten muss (vermute Letzteres).\\

\vspace{1mm}\noindent\textit{day01-scene5-vera/0-prompt}\par
\vspace{2mm}\noindent\textbf{Prompt — Vera Kowalski, Tag 1, Szene 5 (WORK)}\par\vspace{0.5mm}

\vspace{1mm}\noindent\textbf{Kontext}\par

\noindent Nachmittags-Recherche, zweiter Work-Block. Montag = Recherche + Konzeption.\par

\vspace{1mm}\noindent\textbf{Aufgaben}\par

\noindent\hspace{2mm}1.\ \textbf{Fraktionsabhängige Biotech-Visualisierung vertiefen}: Die CD hat im Standup entschieden, dass Biotech-Sichtbarkeit fraktionsabhängig ist. Drei visuelle Profile entwickeln — Architektur, Mode, Farbpalette, Biotech-Manifestation.\\
\noindent\hspace{2mm}2.\ \textbf{Auf Emres Fraktionsbeschreibungen warten}: Emre schreibt gerade mythologische Grundierungen für die drei Fraktionen. Beginne mit dem, was aus Briefing + Vormittags-Recherche bekannt ist.\\
\noindent\hspace{2mm}3.\ \textbf{Traces schreiben}: prompt.md, reasoning.md, output.md\\

\vspace{1mm}\noindent\textbf{CD-Entscheidungen aus dem Standup (Szene 3)}\par

\noindent\hspace{2mm}\textbullet\ Biotech-Sichtbarkeit ist fraktionsabhängig:\\
\noindent\hspace{2mm}\textbullet\ \textbf{Krone} = subtil (Verfall, degenerierende Biotech, Grandeur im Zerfall)\\
\noindent\hspace{2mm}\textbullet\ \textbf{Gilden} = offen (Biotech als Werkzeug, funktional sichtbar, pragmatisch)\\
\noindent\hspace{2mm}\textbullet\ \textbf{Orden} = versteckt (Biotech als Geheimnis, nach aussen rein, innen technologisch)\\
\noindent\hspace{2mm}\textbullet\ Hypothese C genehmigt: Kosmologische Erosion + gebrochene Bindung\\
\noindent\hspace{2mm}\textbullet\ Mythos soll in der Spielwelt umstritten sein (Fraktionen haben verschiedene Versionen)\\

\vspace{1mm}\noindent\textbf{Referenzmaterial}\par

\noindent\hspace{2mm}\textbullet\ Eigene Vormittags-Recherche: \texttt{traces/day01-scene2-vera/output.md}\\
\noindent\hspace{2mm}\textbullet\ Emres mythologische Recherche: \texttt{traces/day01-scene2-emre/output.md}\\
\noindent\hspace{2mm}\textbullet\ Emres Fraktionsbeschreibungen: noch ausstehend (Scene 5 parallel)\\
\noindent\hspace{2mm}\textbullet\ Briefing: \texttt{simulation-2-test/briefing.md}\\

\vspace{1mm}\noindent\textit{day01-scene5-vera/1-reasoning}\par
\vspace{2mm}\noindent\textbf{Trace: Day 01, Scene 5 — Reasoning (Vera Kowalski)}\par\vspace{0.5mm}

\vspace{1mm}\noindent\textbf{Denkprozess}\par

\noindent\textbf{\textit{Die CD-Entscheidung ändert alles}} 

\noindent Heute Morgen hatte ich eine einheitliche visuelle Sprache skizziert — "Organische Gotik" als Dachbegriff. Das war ein Anfang, aber die CD-Entscheidung aus dem Standup macht das Ganze viel interessanter: Biotech-Sichtbarkeit ist FRAKTIONSABHÄNGIG.\par

\noindent Das heisst: Ich brauche nicht EINE visuelle Sprache, sondern DREI. Drei Dialekte derselben Designsprache. Der Spieler muss in eine Stadt laufen und innerhalb von Sekunden wissen: "Das ist Krone / Gilden / Orden" — nur durch die visuelle Sprache. Ohne UI-Hinweis.\par

\noindent\textbf{\textit{Wie man drei Sprachen in einer Welt hält}} 

\noindent Das ist das architektonische Kernproblem. Drei völlig verschiedene Ästhetiken ergeben kein kohärentes Spiel. Drei identische ergeben kein lesbares Fraktionssystem. Die Lösung: \textbf{Gemeinsame Grammatik, verschiedene Vokabulare.}\par

\noindent Die Grammatik (was alle teilen):\par
\noindent\hspace{2mm}\textbullet\ Gotische Grundformen (Spitzbögen, Vertikalität, Strebebögen)\\
\noindent\hspace{2mm}\textbullet\ Organische Materialien (kein Metall als Primärmaterial, kein Glas im modernen Sinn)\\
\noindent\hspace{2mm}\textbullet\ Biotech existiert überall — aber das VERHÄLTNIS zur Biotech unterscheidet sich\\
\noindent\hspace{2mm}\textbullet\ Mitteleuropäisches Licht (bedeckt, diffus, Nebel)\\

\noindent Das Vokabular (was sich unterscheidet):\par
\noindent\hspace{2mm}\textbullet\ WIE Biotech sichtbar ist (Sichtbarkeitsspektrum)\\
\noindent\hspace{2mm}\textbullet\ WELCHE Farben dominieren\\
\noindent\hspace{2mm}\textbullet\ WELCHEN Zustand die Architektur hat\\
\noindent\hspace{2mm}\textbullet\ WAS Mode kommuniziert\\

\noindent\textbf{\textit{Krone: Die kranke Königin}} 

\noindent Metapher: Ein prächtiger Körper, der von innen verfault. Die Krone HAT die älteste Biotech — geerbt, nicht verstanden. Ihre Paläste waren mal das Fortschrittlichste der Welt. Jetzt wuchern die organischen Systeme unkontrolliert. Strebebögen, die eigentlich saubere Rippen waren, haben jetzt knotige Auswüchse. Wände mit Adern, die unter bläterndem Putz pulsieren. Kronleuchter aus biolumineszentem Gewebe, das flackert, weil niemand mehr weiss, wie man es pflegt.\par

\noindent Architektur-Referenz: Kölner Dom trifft verlassenes Krankenhaus. Monumentale Gotik, deren Biotech-Infrastruktur sichtbar wird, WEIL sie degeneriert. Nicht absichtlich sichtbar — sichtbar durch Verfall. Das ist der Unterschied zu den Gilden.\par

\noindent Mode: Burgundische Prachtroben, deren organische Verzierungen eigenständig wachsen. Ein Pelzkragen, der noch atmet. Handschuhe mit Adern, die nicht zum Träger gehören. Morbider Luxus — sie tragen ihren Verfall als Statussymbol, weil sie keine Alternative haben.\par

\noindent Farbe: Karmin (\#8B1A2B) + Kalkstein (\#E8E0D0) + Muskel-Rosa-Untertoe. Die Biotech-Akzente in der Krone sind ROT — Blut, Fleisch, offengelegte Muskulatur. Nicht gesund-rot, sondern fiebrig-rot.\par

\noindent Emres mythologische Resonanz: Tyr/Thor. Pflicht, Ordnung, dem Untergang geweiht. Die Krone WEISS, dass sie stirbt. Sie kämpft trotzdem. Das muss in jeder Säulenkapitell-Maserung spürbar sein.\par

\noindent\textbf{\textit{Gilden: Die funktionierende Maschine}} 

\noindent Metapher: Ein gesunder Körper im Fitnessstudio. Die Gilden haben die modernste Biotech und zeigen sie OFFEN. Für sie ist Biotech kein Erbe und kein Geheimnis — es ist ein Produkt. Sie züchten es, handeln es, optimieren es. Ihre Gebäude FUNKTIONIEREN.\par

\noindent Architektur-Referenz: Organischer Brutalismus. Ehrliche Struktur. Ein Lagerhaus, das wie ein Magen aussieht — und auch so funktioniert: organische Wände, die Temperatur und Feuchtigkeit regulieren. Ein Kontor mit Nervenbahnen, die als Kommunikationssystem dienen. Keine Dekoration, aber beeindruckend durch schiere Effizienz.\par

\noindent Gaudi trifft Habitat 67. Modulare Einheiten, die zusammenwachsen. Gebäude, die sich erweitern, indem man neue Module anzüchtet. Das ist Urban Planning als Biologie.\par

\noindent Mode: Funktionskleidung, clean, teuer. Sichtbare Biotech als Statussymbol — aber nicht als Dekadenz, sondern als Kompetenz. Ein Händler trägt eine Weste mit lebenden Taschen, die sich bei Berührung öffnen. Ein Gildenmeister hat Rüstungskomponenten, die sich sichtbar dem Körper anpassen. Iris van Herpen als Design-Referenz, aber pragmatischer.\par

\noindent Farbe: Amber (\#C49A20) + Venen-Blaugrün (\#0D7377). Warme Toene (Handel, Reichtum, Organik) mit kühlen Biotech-Akzenten. Biolumineszenz hier am hellsten und saubersten — weil die Gilden es am besten beherrschen.\par

\noindent Emres Resonanz: Vanir. Fruchtbarkeit, Handel, Pragmatismus. Die Gilden haben keinen moralischen Kompass, aber sie haben die beste Technologie. Visuell: Effizienz ist ihre Schönheit.\par

\noindent\textbf{\textit{Orden: Das Panoptikum}} 

\noindent Metapher: Ein Körper unter einem Arztkittel. Von aussen makellos. Darunter: Implantate, Nervenbahnen, Überwachungstechnologie. Der Orden versteckt seine Biotech nicht aus Scham, sondern aus STRATEGIE. Wissen ist Macht, und die Macht der Biotech bleibt unsichtbar.\par

\noindent Architektur-Referenz: Zisterzienser-Kloster trifft NSA-Rechenzentrum. Strenge Symmetrie, ordentliche Proportionen, sauberer Stein. Von aussen das Beruhigendste aller Fraktionsgebäude. Aber: In den Wänden laufen Nervenbahnen, die alles registrieren. Im Keller wachsen organische Systeme, die das Wissen der Welt verarbeiten. Panoptikum trifft Skriptorium.\par

\noindent Control (Remedy, 2019) ist hier die stärkste Referenz: Ein Gebäude, das normal aussieht, aber fundamentally wrong ist. Der Unterschied: Beim Orden entdeckt man die Unheimlichkeit nicht sofort. Man spürt sie — ein unmerkliches Summen in den Wänden, Wände die einen Tick zu warm sind, Licht das keiner erklärbaren Quelle folgt.\par

\noindent Mode: Uniformen. Hoch geschlossene Roben, einheitlich, sauber. Unter der Oberfläche: Biotech-Implantate. Wer im Orden aufsteigt, bekommt immer mehr unter die Haut — buchstäblich. Von aussen sieht man nichts. Aber die Augen eines Inquisitors sehen ANDERS als menschliche Augen — ein kaum merkliches Schimmern, ein Pupillenwechsel, der nicht natürlich ist.\par

\noindent Farbe: Schieferblau (\#4A5568) + Schwarz + Kalkstein. Die kühle, kontrollierte Palette. Biotech-Akzente nur im Innern sichtbar: dort das intensivste Blaugrün (\#0D7377), fast chirurgisch. In ihren Laboratorien und Archiven leuchtet die Biotech am hellsten — aber nur dort, wo Aussenstehende keinen Zutritt haben.\par

\noindent Emres Resonanz: Odin. Wissensmonopol. Die Raben Hugin und Munin, die ALLES sehen. Das ist der Orden: allwissend, kontrolliert, opferbereit (Odin hing neun Nächte am Baum für Wissen). Visuell: Die Oberfläche lügt.\par

\noindent\textbf{\textit{Das Sichtbarkeitsspektrum als Designwerkzeug}} 

\noindent Ich brauche ein System, das die drei Fraktionen auf einer gemeinsamen Skala verortet. Nicht "mehr oder weniger Biotech", sondern "anderes Verhältnis zur Sichtbarkeit":\par

\noindent ```\par
\noindent SICHTBAR <-----------------------------------------> VERSTECKT\par
\noindent Gilden          Krone          Orden\par
\noindent (absichtlich)   (unfreiwillig) (absichtlich)\par
\noindent ```\par

\noindent Die Krone ist in der Mitte, nicht weil sie moderat ist, sondern weil ihre Biotech GEGEN IHREN WILLEN sichtbar wird. Das ist der tragische Kern: Sie versuchen, die Fassade zu erhalten, aber die organische Infrastruktur wächst durch die Risse.\par

\noindent\textbf{\textit{Was ich noch brauche}} 

\noindent\hspace{2mm}\textbullet\ Emres Fraktionsbeschreibungen für die mythologische Symbolik in der Architektur (Raben-Motiv für Orden? Wolf für Krone? Schiff/Ähre für Gilden?)\\
\noindent\hspace{2mm}\textbullet\ Schattenfieber-Manifestation: Wie sieht ein Schattenfieber-Quartier aus? Das beeinflusst alle drei Fraktionssprachen, weil das Fieber die Kontrolle (jeder Fraktion auf ihre Art) unterwandert.\\
\noindent\hspace{2mm}\textbullet\ Tiervolk-Architektur: Haben die ihre eigenen Gebäude, oder leben sie in menschlichen Strukturen? Als Händler und Diebe wahrscheinlich letzteres — sie passen sich an, verändern von innen.\\

\noindent\textbf{\textit{Arbeitsform heute}} 

\noindent Kein Concept Art — es ist Montag, Recherchetag. Output ist ein strukturiertes visuelles Profil pro Fraktion mit:\par
\noindent\hspace{2mm}\textbullet\ Architektur-Prinzipien\\
\noindent\hspace{2mm}\textbullet\ Mode-Regeln\\
\noindent\hspace{2mm}\textbullet\ Farbpalette (Hex-Codes)\\
\noindent\hspace{2mm}\textbullet\ Biotech-Manifestation (WIE es aussieht)\\
\noindent\hspace{2mm}\textbullet\ Kontrast zu den anderen beiden (WARUM es anders aussieht)\\
\noindent\hspace{2mm}\textbullet\ Offene Fragen, die auf Emre warten\\

\end{multicols}
\endgroup

\scenedivider

\subsection{Szene 6 · REVIEW}\label{szene-6-review}

\begin{scenemeta}

--- Küche

\end{scenemeta}

\noindent{\scriptsize \agenticon{emre}\agenticon{darius}\agenticon{nami}\agenticon{vera}\agenticon{tobi}\agenticon{leo}\agenticon{finn} Emre Yilmaz, Darius Engel, Nami Okafor, Vera Kowalski, Tobi Richter, Leo Fischer und Finn Bergmann}\par\vspace{1mm}

\begin{figure}
\centering
\includegraphics[width=0.4\linewidth,height=\textheight,keepaspectratio,alt={Szene 6}]{/var/folders/fr/y30bxq454jxb7bycr3yh9gmw0000gn/T/logbook-crop-40frukv8/day-001-scene-006-REVIEW.png}
\caption{Szene 6}
\end{figure}

Tagesabschluss-Review Tag 1. Finn moderiert, jeder prasentiert
Tagesergebnisse. Emre: Drei-Schichten-Kosmologie, mythologisch
grundierte Fraktionen (Krone=Tyr/Thor, Gilden=Vanir, Orden=Odin),
Wanderer-Motiv. Darius: Referenzspiel-Analyse, Schattenfieber als
vierter Nervensystem-Ast (0-100, fünf Stufen), fünf Design-Prinzipien.
Nami: Vier narrative Prinzipien, Wanderer-Motiv als Fraktionsklammer,
Schattenfieber als Unreliable Narrator (Rauschen/Risse/Schwelle). Vera:
Visuelle Referenz-Analyse, drei Fraktionsprofile, Farbpalette mit
Hex-Codes. Tobi: Machbarkeitsanalyse, Pipeline-Architektur V0.1, drei
Engpässe. Leo: Marktanalyse, Bloodborne Insight als Goldstandard,
Pitch-Satz. Finn: Roadmap V1.1. CD-Feedback eingearbeitet:
Orden=Spitzenforschung bestätigt, Schattenfieber nicht reversibel, PC
first, Freelancer-Budget vorbereiten, Tiervolk ohne eigene Architektur,
Replay-Erfahrung gewünscht. Aufgabenverteilung für Dienstag (letzter
Konzepttag) besprochen.

\emph{Artefakt: \texttt{ROADMAP.md\ (V1.1)}}

\vspace{4mm}
\begingroup\scriptsize\setlength{\parskip}{1pt}\setlength{\parindent}{0pt}\setlength{\columnsep}{6mm}
\begin{multicols}{2}
\vspace{1mm}\noindent\textit{day01-scene6-review/0-prompt}\par
\vspace{2mm}\noindent\textbf{Prompt — Day 01, Scene 6: REVIEW}\par\vspace{0.5mm}

\vspace{1mm}\noindent\textbf{Kontext}\par
\noindent Finn Bergmann moderiert das Tagesabschluss-Review am Ende von Tag 1 (Montag). Alle 7 Teammitglieder sind anwesend. Ort: Küche. Uhrzeit: 16:00.\par

\vspace{1mm}\noindent\textbf{Aufgabe}\par
\noindent\hspace{2mm}\textbullet\ Jedes Teammitglied präsentiert seine Tagesergebnisse (ca. 3 Minuten pro Person)\\
\noindent\hspace{2mm}\textbullet\ Offene Fragen werden gesammelt\\
\noindent\hspace{2mm}\textbullet\ CD-Feedback wird am Ende eingeflochten\\
\noindent\hspace{2mm}\textbullet\ Ausblick auf morgen (Dienstag = letzter Recherche/Konzeptionstag)\\

\vspace{1mm}\noindent\textbf{Eingaben}\par
\noindent\hspace{2mm}\textbullet\ Tagesergebnisse aller 7 Agents (aus Szenen 2-5)\\
\noindent\hspace{2mm}\textbullet\ CD-Feedback zu offenen Fragen\\
\noindent\hspace{2mm}\textbullet\ Finn-Memory (Tag 1, Szenen 1-3)\\

\vspace{1mm}\noindent\textbf{Teilnehmer}\par
\noindent Emre (Lore), Darius (Game Design), Nami (Narrative), Vera (Art), Tobi (Tech), Leo (Marketing), Finn (Producer)\par

\vspace{1mm}\noindent\textit{day01-scene6-review/1-reasoning}\par
\vspace{2mm}\noindent\textbf{Reasoning — Day 01, Scene 6: REVIEW}\par\vspace{0.5mm}

\vspace{1mm}\noindent\textbf{Moderationsplanung}\par

\noindent Reihenfolge der Präsentation orientiert sich an der Abhängigkeitskette: Emre (Fundament) zürst, dann Darius (Mechaniken), Nami (Narrativ), Vera (Art), Tobi (Tech), Leo (Marketing). Ich selbst fasse am Ende zusammen.\par

\vspace{1mm}\noindent\textbf{Beobachtungen zum Tag}\par

\noindent\hspace{2mm}1.\ \textbf{Kohärenz ist da}: Emres Kosmologie, Darius' Schattenfieber-Mechanik und Namis Unreliable-Narrator-Konzept greifen nahtlos ineinander. Das war nicht selbstverständlich am ersten Tag.\\
\noindent\hspace{2mm}2.\ \textbf{Emre-Risiko hat sich nicht materialisiert}: Mein Risiko \#1 (Emre verliert sich in der Tiefe) ist nicht eingetreten — er hat zielgerichtet geliefert und Vera aktiv zugearbeitet.\\
\noindent\hspace{2mm}3.\ \textbf{Offene Fragen sind präzise}: Keine vagen "wir müssen noch mal reden"-Punkte, sondern konkrete Entscheidungsfragen an die CD.\\
\noindent\hspace{2mm}4.\ \textbf{Schattenfieber als Klammer}: Hat sich organisch als verbindendes Element entwickelt — Darius (Mechanik), Nami (Narrativ), Emre (Kosmologie), Leo (Community) arbeiten alle daran.\\

\vspace{1mm}\noindent\textbf{CD-Feedback-Verteilung}\par

\noindent Die CD-Antworten müssen kontextbezogen an die richtigen Leute gehen:\par
\noindent\hspace{2mm}\textbullet\ Orden als Spitzenforschung → Emre (WBB), Vera (visuelles Profil anpassen?)\\
\noindent\hspace{2mm}\textbullet\ Schattenfieber nicht heilbar → Darius (Mechanik-Konsequenz), Nami (narrativ bestätigt)\\
\noindent\hspace{2mm}\textbullet\ PC first → Tobi (Pipeline-Entscheidung)\\
\noindent\hspace{2mm}\textbullet\ Freelancer-Budget → Ich (Roadmap-Update)\\
\noindent\hspace{2mm}\textbullet\ Tiervolk ohne eigene Architektur → Vera (Art Direction), Emre (Lore-Konsequenz)\\
\noindent\hspace{2mm}\textbullet\ Replay-Erfahrung → Darius (Design), Nami (Narrative Layers)\\

\vspace{1mm}\noindent\textbf{Ausblick Dienstag}\par

\noindent Letzter Recherche/Konzeptionstag. Mittwoch beginnt Produktion V1. Also: Offene Konzeptfragen müssen morgen geklärt werden. Jeder sollte wissen, was sein V1-Deliverable am Mittwoch ist.\par

\end{multicols}
\endgroup

\scenedivider

\end{document}
